%\part{}
\lecture{Random Variables}{random-variables}


\title{Random Variables}
\subtitle{Quantifying Outcomes}

\author{Kelly Black}
\institute{Clarkson University}
\date{8 February 2012}

\begin{frame}
  \titlepage
\end{frame}

\begin{frame}
  \frametitle{Outline}
  \tableofcontents[pausesection,hideallsubsections]
\end{frame}


\section{Clicker Quiz}


\begin{frame}
  \frametitle{Clicker Quiz}

  I flip a coin three times. What is the probability that I get two tails?

  \vfill

  \begin{tabular}{l@{\hspace{3em}}l@{\hspace{3em}}l}
    A: 1/8 & B: 3/8 & C: 1/2
  \end{tabular}

  \vfill
  \vfill
  \vfill

\end{frame}


\section{Random Variables}

\begin{frame}
  \frametitle{Random Variables}

  \begin{definition}
    A \textbf{random variable} is a variable whose outcome is a number
    and has a random component.
  \end{definition}

  \uncover<2->
  {
    \begin{definition}
      A \textbf{discrete random variable} is a random variable whose
      outcome is one of a countable number of outcomes.
    \end{definition}
  }

  \uncover<3->
  {
    \begin{definition}
      A \textbf{continuous random variable} is a random variable whose
      outcome come from a range of values.
    \end{definition}
  }

  \uncover<4->
  {
    \begin{definition}
      A \textbf{probability distribution} is a list of probabilities of
      \textit{all possible outcomes} of a random variable.
    \end{definition}
  }

\end{frame}



\begin{frame}{Discrete Random Variables}

  \begin{columns}
    \column{.40\textwidth}

    \begin{eqnarray*}
      \begin{array}{l|l}
        \mathrm{X} & p \\ \hline
        x_1 & p_1 \\
        x_2 & p_2 \\
        x_3 & p_3 \\
        \vdots & \vdots \\
        x_n & p_n
      \end{array}
    \end{eqnarray*}

    \column{.60\textwidth}
    Properties:
    \begin{eqnarray*}
      \begin{array}{rcccl}
        0 & \leq & p_i & \leq 1
      \end{array}
      \\
      p_1 + p_2 + p_3 + \cdots + p_n & = & 1.
    \end{eqnarray*}

  \end{columns}
  
\end{frame}


\begin{frame}{Discrete Random Variables}

  \begin{columns}
    \column{.10\textwidth}
    \begin{eqnarray*}
      \begin{array}{l|l}
        \mathrm{X} & p \\ \hline
        x_1 & p_1 \\
        x_2 & p_2 \\
        x_3 & p_3 \\
        \vdots & \vdots \\
        x_n & p_n
      \end{array}
    \end{eqnarray*}
    \vfill

    \column{.90\textwidth}
    \uncover<2->
    {
      \begin{definition}
        The \textbf{mean} of a random variable is
        \begin{eqnarray*}
          \mu_X & = & x_1 p_1 + x_2 p_2 + \cdots + x_n p_n.
        \end{eqnarray*}
      \end{definition}
    }

    \uncover<3->
    {
      \begin{definition}
        The \textbf{variation} of a random variable is
        \begin{eqnarray*}
          \sigma^2_X & = & (x_1-\mu_X)^2 p_1 + (x_2-\mu_X)^2 p_2 + \cdots + (x_n-\mu_X)^2 p_n.
        \end{eqnarray*}
      \end{definition}
    }

    \uncover<4->
    {
      \begin{definition}
        The \textbf{standard deviation} of a random variable is the
        square root of the variance.
      \end{definition}
    }

    
  \end{columns}
  
\end{frame}


\section{Examples}

\begin{frame}{Example}
  \begin{columns}
    \column{.15\textwidth}
    \begin{eqnarray*}
      \begin{array}{r|l}
        \mathrm{X} & p \\ \hline
        -2 & \frac{1}{6} \\ [5pt]
         0 & \frac{3}{6} \\ [5pt]
         2 & \frac{2}{6}
      \end{array}
    \end{eqnarray*}

    \column{.90\textwidth}
    \uncover<2->
    {
      \begin{eqnarray*}
        \mu_X & = & -2 \cdot \frac{1}{6} + 0 \cdot \frac{3}{6} + 2 \cdot \frac{2}{6}, \\
        & = & \frac{1}{3}.
      \end{eqnarray*}
    }

  \end{columns}

    \uncover<3->
    {
        \begin{eqnarray*}
          \sigma^2_X & = & \lp -2-\frac{1}{3}\rp^2 \frac{1}{6} + 
          \lp 0-\frac{1}{3}\rp^2 \frac{3}{6} + \lp 2-\frac{1}{3}\rp^2 \frac{2}{6}, \\
          & = & \frac{17}{9}.
        \end{eqnarray*}
    }

    \uncover<4->
    {
      \begin{eqnarray*}
        \sigma_X & = & \sqrt{\frac{17}{9}}.
      \end{eqnarray*}
      \vfill
    }

    

\end{frame}



\begin{frame}{Clicker Quiz}

  What is the mean for the following probability distribution?
    \begin{eqnarray*}
      \begin{array}{r|l}
        \mathrm{X} & p \\ \hline
         0 & \frac{1}{8} \\ [5pt]
         1 & \frac{1}{8} \\ [5pt]
         2 & \frac{4}{8} \\ [5pt]
         3 & \frac{2}{8}
      \end{array}
    \end{eqnarray*}

    \vfill

  \begin{tabular}{l@{\hspace{3em}}l@{\hspace{3em}}l}
    A: 15/8  & B: 2 & C: 17/8
  \end{tabular}

  \vfill
  \vfill
  \vfill

\end{frame}

\begin{frame}{Example}
  \begin{columns}
    \column{.15\textwidth}
    \begin{eqnarray*}
      \begin{array}{r|l}
        \mathrm{X} & p \\ \hline
         0 & \frac{1}{8} \\ [5pt]
         1 & \frac{1}{8} \\ [5pt]
         2 & \frac{4}{8} \\ [5pt]
         3 & \frac{2}{8}
      \end{array}
    \end{eqnarray*}

    \column{.90\textwidth}
    \uncover<2->
    {
      \begin{eqnarray*}
        \mu_X & = & 0 \cdot \frac{1}{8} + 1 \cdot \frac{1}{8} + 2 \cdot \frac{4}{8} + 3 \frac{2}{8}, \\
        & = & \frac{15}{8}.
      \end{eqnarray*}
    }

  \end{columns}

    \uncover<3->
    {
        \begin{eqnarray*}
          \sigma^2_X & = & \lp 0-\frac{15}{8}\rp^2 \frac{1}{8} + 
          \lp 1-\frac{15}{8}\rp^2 \frac{1}{8} + \lp 2-\frac{15}{8}\rp^2 \frac{4}{8} + 
          \lp 3-\frac{15}{8}\rp^2 \frac{2}{8}, \\
          & = & \frac{55}{64}.
        \end{eqnarray*}
    }

    \uncover<4->
    {
      \begin{eqnarray*}
        \sigma_X & = & \sqrt{\frac{55}{64}}.
      \end{eqnarray*}
      \vfill
    }

    

\end{frame}



% LocalWords:  Clarkson pausesection hideallsubsections
