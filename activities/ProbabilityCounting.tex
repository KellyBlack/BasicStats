\documentclass[12pt]{article}

\oddsidemargin=0.0in
\evensidemargin=0.0in
\textwidth=6in
\topmargin=-1in
\textheight=9.5in

\setlength{\parindent}{0pt}

\begin{document}

 \textbf{Probability: Counting} \\
\\
\textit{You are not expected to finish this during the class period. Anything you don't finish is good to practice at home.}  \\ [5pt]

\textbf{Please define the following terms in your own words:} (1 or 2 sentences) \\ [5pt]

  \textit{Probability:} \\
\\
\textit{Certain vs. Impossible}\\

\textbf{Definittions:}  \\ [5pt]
\textit{Counting: } Determining the number of outcomes that sequence of events could result in.  Ex: There are 3 types of cones in an ice cream store and 10 flavors therefore there are 30 different types of cone and ice cream combinations if only one cone and one flavor is chosen.\\  [12pt]

\textit{Factorial: }Used to represent the number of ways that k number of objects can be ordered. k! = k x (k-1) x (k-2) x ... x 1.  Remember 0! = 1. \\ [12pt]


 \textbf{Examples:}

\begin{enumerate}
\item In how many different ways can a cross country race be completed if there are 5 runners and no ties?\\

\item If a multiple choice test has 10 questions with 5 choices for each questions in how many different ways can you complete the test?\\

\item A license plate has three letters and 3 digits. How many different license plates can be made made? \\

\item An ID number begins with a letter and is followed by 4 numbers. How many different ID numbers can be made? \\
\end{enumerate}

\textbf{Definititon:}\\

\textit{Permutation:}   An ordering of a set of distinct objects.  \\  [12pt]
Notation: nPr where \textit{n} is the number of distinct objects and \textit{r} is the number of objects taken at a time.  Order matters.\\[12pt]
Example: How many different can this set of letters be arranged if two letters are used in a group?\\[12pt]
Letter Set: ABC\\[12pt]
Possible permutations: AA, AB,  AC, BA,  BB,  BC, CA, CB,  CC\\ [12pt]

\textit{Combination:}  A combination of a a number of elements form a set where order does not matter. \\ [12pt]
Example: How many different combinations of 2 letters can be made from these 3 letters?\\ [12pt]
Letter Set: ABC\\ [12pt]
Possible combinations: AB,  AC,  BA, BC, CA, CB. \\ [12pt]

\textbf{Formulas:}

Write the formula for perambulations: \\

Write the formula for combinations: \\

\textbf{Practice:}

\begin{enumerate}

\item Evaluate the expression:  5!

\item Evaluate the expression: 9P5

\item Evaluate the expression: 4C2

\item Ten children need to line up for a picture. In how many different ways can this be done?

\item How many different five card hands can you pick from a 52 card deck?

\item A latte consists of milk, espresso, and a flavored syrup (if desired). The corner coffee shop offers Almond milk, Soy milk, whole milk and fat free milk, a person can get one, two or three shots of espresso along with vanilla syrup, mint syrup or no syrup. How many different lattes can you order with these options?

\item There are twelve kids at a birthday party. You're a magician in need of 4 volunteers. How many different groups of volunteers can you chose if order does not matter?

\item It's time you apply to graduate schools and you're narrowed it down to the top 500. Unfortunately you only have the money to apply to 10. How many different combinations of schools can you pick?

\item  :) If you have a quiz every recitation worth 10 points and there are 15 recitations what is the maximum grade that you can get if you do not attend any of the recitations and receive a 0 on each quiz? :) 

\item You are picking cards from a 52 card deck. You need a Jack followed by a queen and then a king. How many ways can this happen?

\item You are organizing teams for field day at school. There are 50 boys and 30 girls. If each team has 8 people and you want 5 boys and 3 girls on each team in how many different way can you organize the teams. 

\item  You are making 7 layer bars. The lays consist of crushed graham cracker, chocolate, fudge, coconut, sprinkles, caramel and granola. In how many different ways can you order the layers?


\end{enumerate}

\end{document}




