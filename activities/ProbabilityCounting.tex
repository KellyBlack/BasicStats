

\preClass{Counting}


\begin{problem}
\item Basic Counting examples

  \begin{subproblem}
  \item 
    \begin{eqnarray}
      y & = 3x + 4.
    \end{eqnarray}
    \vfill
  \end{subproblem}


\end{problem}


\actTitle{Counting} 

\textit{You are not expected to finish this during the class
  period. Anything you don't finish is good to practice at home.}

\begin{problem}
  \item Define the following terms in your own words: (1 or 2 sentences) 

    \begin{subproblem}
      \item \textit{Probability:}

        \vfill

      \item \textit{Certain vs. Impossible}

        \vfill

    \end{subproblem}


    \begin{definition}
      \textit{Counting} is the method used to determine the number of
      outcomes can result in a sequence of events.  
    \end{definition}

    \begin{example}
      There are 3 types of cones in an ice cream store and 10 flavors
      therefore there are 30 different types of cone and ice cream
      combinations if only one cone and one flavor is chosen.
    \end{example}

    \begin{definition}
      The \textit{factorial} represents the number of ways that $k$
       objects can be ordered. $k! = k \cdot (k-1) \cdot (k-2) \cdot
       \ldots \cdot 1$.  The value of  0! is defined to be 1.
    \end{definition}


        \clearpage

    \item Determine the number of outcomes for each of the following
      situation.

      \begin{subproblem}
      \item In how many different ways can a cross country race be
        completed if there are 5 runners and no ties?

        \vfill

      \item If a multiple choice test has 10 questions with 5 choices
        for each question in how many different ways can you complete
        the test? (Assume that there is only one answer given per
        question.)

        \vfill

      \item A license plate has three letters and 3 digits. How many
        different license plates can be made made?

        \vfill

      \item An ID number begins with a letter and is followed by 4
        numbers. How many different ID numbers can be made?

        \vfill

  \end{subproblem}


\end{problem}



\actTitle{Counting Formulas} 

We now examine the formulas used to determine the number of
occurrences for a given situation.

\begin{definition}
  The number of permutations is the number of ways to order a set of
  distinct objects. (Order matters).
\end{definition}


\begin{notation}
  The quantity $\prescript{~}{n}{P}_r$ is the number of distinct ways
  that $r$ objects can be pulled from and arranged from a set of $n$
  objects.  (Order matters.)
\end{notation}

\begin{example}
  How many different ways can two letters be arranged from the letters
  below?

  Letter Set: ABC

  Possible permutations: AB, AC, BA, BC, CA, CB.

\end{example}

\begin{definition}
  The number of combinations is the number of ways to determine
  collections of distinct objects where the order does not matter.
\end{definition}

\begin{notation}
  The quantity $\prescript{~}{n}{C}_r$ is the number of combinations
  that $r$ objects can be pulled from a set of $n$ objects where order
  does not matter.
\end{notation}


\begin{example}
  How many different ways can two letters be arranged from the letters
  below?

  Letter Set: ABC

  Possible combinations: AB, AC, BC.
\end{example}

\clearpage

\begin{problem}

  \item Write the formula for the number of ways to arrange $k$
    objects drawn from a set of $n$ distinct objects where the order
    matters.

    \vfill

  \item Write the formula for the number of ways to arrange $k$
    objects drawn from a set of $n$ distinct objects where the order
    does not matter. 

    \vfill

  \item Determine each of the following values.

    \begin{subproblem}

    \item Evaluate the expression: 5!

      \vfill

    \item Evaluate the expression: $\prescript{~}{9}{P}_5$

      \vfill


    \clearpage

    \item Evaluate the expression: $\prescript{~}{4}{C}_2$

      \vfill

    \item Ten children will be lined up in groups of 3 to take a
      picture. How many different ways can this be done?

      \vfill

    \item Ten children will have their picture taken in groups of
      3. How many different combinations are there?

      \vfill


    \item How many different five card hands can you pick from a 52
      card deck?

      \vfill

    \item A latte consists of milk, espresso, and a flavored syrup (if
      desired). The corner coffee shop also offers Almond milk, Soy
      milk, whole milk and fat free milk. A person can get one, two or
      three shots of espresso. Finally, the person can order vanilla
      syrup, mint syrup or no syrup. How many different lattes can you
      order with these options?

      \vfill

      \clearpage

    \item There are twelve kids at a birthday party. You are a
      magician in need of 4 volunteers. How many different groups of
      volunteers can you chose if order does not matter?

      \vfill

    \item It's time you apply to graduate schools and you're narrowed
      it down to the top 500. Unfortunately you only have the money to
      apply to 10. How many different combinations of schools can you
      pick?

      \vfill

    \item If you have a quiz every recitation worth 10 points and
      there are 15 recitations what is the maximum average grade that
      you can get if you only attend two recitations?

      \vfill

      \clearpage

    \item You are picking cards from a 52 card deck. You need a Jack
      followed by a queen and then a king. How many ways can this
      happen?

      \vfill

    \item You are organizing teams for field day at school. There are
      50 boys and 30 girls. If each team has 8 people and you want 5
      boys and 3 girls on each team in how many different way can you
      organize the teams.

      \vfill

    \item You are making 7 layer bars. The layers consist of crushed
      graham cracker, chocolate, fudge, coconut, sprinkles, caramel or
      granola. In how many different ways can you order the layers?

      \vfill

  \end{subproblem}

\end{problem}



