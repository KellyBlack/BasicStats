\preClass{Calculating Probabilities Associated with Sample Means}


\begin{problem}
\item Examples of calculating probabilities associated with sample means.

  \begin{subproblem}
  \item 
    \begin{eqnarray}
      y & = 3x + 4.
    \end{eqnarray}
    \vfill
  \end{subproblem}


\end{problem}


\actTitle{The Confidence Interval for the Sample Mean} 

\begin{problem}
  \item A random variable has a mean of 5.0 and a standard deviation
    of 17.0. The random variable is sampled twenty times, and the
    sample mean, $\bar{x}$, is calculated. Answer each of the
    following questions:
    \begin{subproblem}
      \item What is the probability that $\bar{x}$ is more than 6.0.

        \vfill

      \item What is the probability that $\bar{x}$ is more than 6.0 if
        it were sampled forty times instead?

        \vfill

      \item Determine a value, $x^*$, where there is a probability of
        0.025 that $\bar{x}$ with twenty samples will be smaller than
        $x^*$.

        \vfill

    \end{subproblem}

    \clearpage

  \item The equation for a confidence interval is 
    \begin{eqnarray*}
      \left( \bar{x} \mp \frac{t_{\frac{\alpha}{2},n-1}s}{\sqrt{n}}\right).
    \end{eqnarray*} 

    \begin{subproblem}
    \item $\bar{x}$ is the \rule{3cm}{0.2mm}   
    \item $t_{\frac{\alpha}{2},n-1}$ is the \rule{3cm}{0.2mm}  
    \item $\frac{s}{\sqrt{n}}$ is the \rule{3cm}{0.2mm} 
    \item A t-critical point is used when you don't know the
      \rule{3cm}{0.2mm}
    \item If you do know it then a critical
      point of the \rule{3cm}{0.2mm} distribution is used.
    \end{subproblem}

  \item Supposed you have a data set with sample mean $\bar{x} = 3.5$,
    $s=0.25$, and $n=40$. Calculate the 90\%, 95\%, and 99\%
    confidence internvals, then explain what happens to the length of
    the interval as the confidence level increases.

    \vfill

\clearpage

\item The quality control manager at a tire plant is asked to estimate
  the mean tread life for one of the plant's products. A random sample
  of 64 tires is made. The sample mean is 55,100 miles with a sample
  standard deviation of 2,050 miles. 
  \begin{subproblem}
  \item Determine the 95\% confidence interval for the mean tread
    life.

    \vfill

  \item Can the manufacturer claim that the tread life is 56,000
    miles?

    \vfill

  \item What is the probability that a randomly selected tire will
    last longer than 56,000 miles?

    \vfill

  \item Discuss whether or not a tread life of 60,000 miles is unusual.

    \vfill

  \item How would your answers to the previous questions change if the
    sample standard deviation were 3,500 miles?

    \vfill

  \end{subproblem}

\clearpage

\item The following data set shows the U.S. GDP from 2000 to 2010 in
  billions of dollars. Construct a 95\% confidence interval for the
  mean GDP in billions of dollars.

    \begin{tabular}{l|l}
      Year & GDP (Billions of USD) \\ \hline
      2000 & 12565.2 \\
      2001 & 12684.4 \\
      2002 & 12909.7 \\
      2003 & 13270.0 \\
      2004 & 13774.0 \\
      2005 & 14235.6 \\
      2006 & 14615.2 \\
      2007 & 14876.6 \\
      2008 & 14833.6 \\
      2009 & 14417.9 \\
      2010 & 14779.4
    \end{tabular}


\end{problem}


\actTitle{The Confidence Interval for the Sample Proportion} 

\begin{problem}
\item What is the confidence interval for the sample proportion? Write
  out the formula for finding the confidence interval for $\hat{p}$.
  \vspace{3em}

  \item A company can stay in business if they are profitable at the
    end of the year (assume break-even coutns as profitable). The
    Mayor of Beaverdam claims that at least 65\% of companies stay in
    business each year. A sample of 40 companies are checked for
    profitability at the end of the year. Of these 30 show to be
    profitable. Construct a 95\% confidence interval and decide
    whether or not the mayor is accurate in saying that at least 65\%
    stay in business.

    \vfill

    \clearpage

  \item Using your equation for $\hat{p}$ on the previous page to solve for $n$.

    \vfill

  \item Suppose you want to know if dropping a phone into water will
    damage it. You're looking for the proportion of phones that stop
    working after being dropped in the water. If you wanted a 95\%
    confidence interval with a margin of error of $\pm .07\%$, how
    many phones do you test? (Assume you have no idea what the true
    value of $p$ is.)

    \vfill

  \item Redo your calculations on the previous question assuming that
    $\hat{p}=0.05$

    \vfill
    \vfill


    \clearpage

  \item A group of University of San Antonio professors compared the
    sale price with the estimated market prices for homes that were
    published by zillow.com. They estimate that roughly half of the
    sale prices for homes on zillow.com were overestimated. They based
    this estimate on a sample of 2,045 homes sampled on the web site.

    \begin{subproblem}
    \item Determine the confidence interval for the proportion of homes
      whose market value is overestimated by more than 10\% on
      zillow.com. 

      \vfill

    \item How likely is it that the real proportion is less than 45\%?

      \vfill

    \item Describe the variables used in the study and how they were
      used. 

      \vfill

    \item Describe the population used in this study. Should they use
      the same number, 2,045 homes, if they wanted to repeat the study
      for the whole United States?

      \vfill

    \end{subproblem}

\end{problem}
