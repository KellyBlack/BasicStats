\preClass{Calculating Confidence Intervals}


\begin{problem}
\item Examples of calculating confidence intervals.

  \begin{subproblem}
  \item 
    \begin{eqnarray}
      y & = 3x + 4.
    \end{eqnarray}
    \vfill
  \end{subproblem}


\end{problem}


\actTitle{An Example of Determining Confidence Intervals} 

Your company is building a new website to offer reviews of
restaurants. The website has a mobile app associated with it that can
be used to provide ratings and also estimate waiting times.

\begin{problem}
\item In the beta testing for the app a small number of people are
  allowed the app to calculate the waiting time required to obtain a
  seat.  The app reported the following wait times for one restaurant in minutes: \\
  \begin{tabular}{rrrrrrrr}
    3.3 & 4.1 & 1.5 & 3.5 & 1 & 3.5 & 0.4 & 2.2
  \end{tabular}
    \begin{subproblem}
    \item Determine the 95\% confidence interval for the mean waiting time
      for this restaurant.

      \vfill

    \item Determine the 99\% confidence interval for the mean waiting time
      for this restaurant.

      \vfill

      \clearpage

    \item The restaurant claims that the mean waiting time is 2.0
      minutes with a standard deviation of 1.3 minutes. Assuming that
      their claim is correct what is the probability that you would
      have obtained the sample mean calculated above or higher? Is
      their claim plausible?

      \vfill

    \item Suppose that the restaurant claims that the mean waiting
      time is 1.0 minutes with a standard deviation of 1.3
      minutes. Assuming that their claim is correct what is the
      probability that you would have obtained the sample mean
      calculated above or higher? Is their claim plausible?

      \vfill

      \clearpage

    \item Your company wants to expand the beta testing program. Your
      supervisor would like to be able to obtain a 95\% confidence
      interval for the waiting time whose total width is one
      minute. Assuming that the standard deviation is 1.3 minutes, how
      many samples will be required?

      \vfill

    \item Your company wants to expand the beta testing program. Your
      supervisor would like to be able to obtain a 95\% confidence
      interval for the waiting time whose total width is thirty
      seconds. Assuming that the standard deviation is 1.3 minutes,
      how many samples will be required?

      \vfill


    \end{subproblem}

    \clearpage

  \item The app is used for people to rate the overall experience at a
    restaurant. At the end of the meal the app is used to answer the
    question, ``Would you recommend this restaurant to a friend?''
    Results for one restaurant are given below:

    \begin{tabular}{rrrrrrrr}
      Yes & Yes & No  & No & Yes & Yes & No  & No \\
      No  & Yes & Yes & No & No  & Yes & Yes & Yes
    \end{tabular}

    (For the following questions assume that you can use the normal
    approximation even though it is not appropriate here.)

    \begin{subproblem}
    \item Determine the 95\% confidence interval for the percentage of
      people who would recommend the restaurant.

      \vfill

    \item Determine the 99\% confidence interval for the percentage of
      people who would recommend the restaurant.

      \vfill

      \clearpage

    \item Suppose that the restaurant manager claims that 75\% of its
      customers would recommend the restaurant to a friend. If this is
      true what is the probability that you would have obtained the
      sample proportion above or less?

      \vfill

    \item Suppose that the restaurant manager claims that 85\% of its
      customers would recommend the restaurant to a friend. If this is
      true what is the probability that you would have obtained the
      sample proportion above or less?

      \vfill

      \clearpage

    \item Assume that your company wants to expand its beta testing
      program. Your supervisor would like to obtain a 95\% confidence
      interval for the proportion to be plus or minus 3\%. How many
      samples are required assuming you get the same sample proportion
      above?

      \vfill

    \item Assume that your company wants to expand its beta testing
      program. Your supervisor would like to obtain a 98\% confidence
      interval for the proportion to be plus or minus 3\%. How many
      samples are required assuming you do not know what the true
      proportion is.

      \vfill

    \end{subproblem}

\end{problem}


\actTitle{More Examples of Determining Confidence Intervals} 

\begin{problem}

\item A group of people who spend the majority of their work time at a
  computer are given a questionnaire. The group includes 315 men and
  248 women.

  \begin{subproblem}
  \item One of the questions is whether or not the respondent has felt
    a sharp pain in their neck or their left or right scapular
    area. From the group of men, 197 reported that they have felt a
    sharp pain. Determine the 95\% confidence interval for the
    proportion of male computer workers who have felt a sharp pain.

    \vfill

  \item One of the questions is whether or not the respondent has felt
    a sharp pain in their neck or their left or right scapular
    area. From the group of women, 173 reported that they have felt a
    sharp pain. Determine the 95\% confidence interval for the
    proportion of male computer workers who have felt a sharp pain.

    \vfill

    \clearpage

  \item A paper is published, and the authors claim that 65\% of men
    working at a computer will experience a sharp pain in their neck
    or their left or right scapular area. Assuming that the paper is
    correct what is the probability that you would have found the
    proportion in your study or less?

    \vfill

  \item A paper is published, and the authors claim that 75\% of women
    working at a computer will experience a sharp pain in their neck
    or their left or right scapular area. Assuming that the paper is
    correct what is the probability that you would have found the
    proportion in your study or more?

    \vfill

    \clearpage

  \item Your supervisor wishes to repeat the study. She would like the
    95\% confidence interval for the proportion of people who
    experience a sharp pain in their neck or their left or right
    scapular area to have a total width of 1\%. Use the proportion
    from all people in your study to determine the number of people
    who should be polled.

    \vfill

  \item Your supervisor wishes to repeat the study. She would like the
    95\% confidence interval for the proportion of people who
    experience a sharp pain in their neck or their left or right
    scapular area to have a total width of 1\%. Determine the number
    of people who should be polled assuming you do not have an idea
    what the true proportion is.


    \vfill

  \end{subproblem}

  \clearpage

\item The results from the 2013 Retirement Security
  Survey\footnote{\url{http://www.ebri.org/pdf/briefspdf/ebri_ib_397_mar14.rcs.pdf}}
  were compiled after interviewing 1,501 people. One-thousand of the
  people interviewed were still working, and the rest were retired. 
  \begin{subproblem}
  \item The following quote can be found in the methodology section
    (page 32):
    \begin{quote}
      In theory, the weighted sample of 1,000 workers yields a
      statistical precision of plus or minus 3.5 percentage points
      (with 95 percent certainty) of what the results would be if all
      Americans age 25 and older were surveyed with complete accuracy.
    \end{quote}
    What does this mean?

    \vfill

  \item What value did they use for $p$ to make this calculation?

    \vfill

    \clearpage

  \item On page 6 the authors include the following quote:
    \begin{quote}
      Eighteen percent are now very confident they will have enough
      money to live comfortably throughout their retirement years, 5
      percentage points higher than the low of 13 percent measured in
      2009, 2011, and 2013, but still well below the historic high of
      27 percent observed in 2007.
    \end{quote}

    \begin{subproblem}
    \item Determine the two 95\% confidence intervals for 2013 and the
      year 2009.

      \vfill

    \item Rewrite the paragraph to properly convey the statistical
      uncertainty in the results.

      \vfill

      \clearpage

    \item Suppose that the two proportions, $p_{2009}$ and
      $p_{2013}$ are the same, 13 percent. What is the variance in their
      difference
      \begin{eqnarray*}
        \hat{p}_{2009} - \hat{p}_{2013}?
      \end{eqnarray*}

      \vfill

    \item If you assume that both  proportions are the same, 13
      percent, what is the probability that the difference will be
      more than 5 percent?

      \vfill

    \item If you assume that both  proportions are the same, 13
      percent, what is the probability that the difference will be
      less than -5 percent?

      \vfill

    \end{subproblem}
  \end{subproblem}

\end{problem}
