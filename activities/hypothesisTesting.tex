\preClass{Calculating probabilities associated with sample distributions.}


\begin{problem}
\item Examples of calculating probabilities of sample means.

  \begin{subproblem}
  \item 
    \begin{eqnarray}
      y & = 3x + 4.
    \end{eqnarray}
    \vfill
  \end{subproblem}


\end{problem}


\actTitle{Constructing Hypothesis Tests} 

\begin{problem}

\item For each question below organize and state the important
  information, state the hypothesis test and what distribution you
  would use. (Just state the hypothesis test.)

  \begin{subproblem}
  \item A new approach for a therapy will be tested. The mean number
    of people using the previous therapy required a mean of 30.0 days
    for full recovery. You use the new approach on thirty patients,
    and the sample mean is 28.0 days with a sample standard deviation
    of 12.0 days.

    \vfill

  \item The labels on bottles produced by a bottling plant indicate that
    the drink contains two liters. The county division of weights
    sampled twenty bottles to determine if the label is accurate. They
    found a sample mean of 1.99 liters with a standard deviation of
    0.05 liters.

    \vfill

  \item You are considering making a loan to a company that uses
    injection molding to produce golf tees. Prior to approving the
    loan you want to insure that the company's production process is
    satisfactory. The mean weight of the tees should be .250
    ounces. Forty random tees are sampled, and the sample mean and
    sample standard deviation are calculated. 

    \vfill

  \item The manager of the customer service division said that their
    consumer satisfaction has improved. A random sample of 400 people
    who have called for service is sampled, and 280 reported that they
    were pleased with the service. Last year it was estimated that
    65\% of the people using the service were pleased. 

    \vfill

  \end{subproblem}

\clearpage

\item Employees at your firm get an annual bonus. The goal is to
  provide a mean of \$1,000 for the bonus, but you feel that this is
  not the case. You will poll a number of people to see what their
  bonus was.

\begin{subproblem}
\item State the null and alternative hypothesis.
  \vspace*{2em}

\item The following equation is used to calculate a $p$-value
  \begin{eqnarray*}
    p = P(z \le -|z^*|) + P(z \ge |z^*|)
  \end{eqnarray*}
  Draw a sketch that demonstrates what this means.

  \vfill

\item A $z$-distribution is a symmetric distribution which means that
  \begin{eqnarray*}
    p = 2 \cdot  P(z \ge |z^*|)
  \end{eqnarray*}
  Draw a sketch that demonstrates what this means.

  \vfill



\item For a two-sided hypothesis test the rejection region is
  $|z| > z_{\frac{\alpha}{2},n-1}$ while the acceptance region is
  $|z| < z_{\frac{\alpha}{2},n-1}$.  Find the acceptance region for
  $n=20$ where $\alpha = 0.1$, $0.05$, and $0.01$. What is happening
  to the size of the acceptance region?

  \vfill

\end{subproblem}

\clearpage

\item When the population standard deviation is known we use a
  $z$-statisitic. The $p$-value for a two sided test is 
  \begin{eqnarray*}
    p = 2 \cdot p(|z|\leq Z_{cr} )
  \end{eqnarray*}
  \begin{subproblem}
  \item State the equation for the $z$-statistic
    \vfill

  \item If your sample size is 30, population mean is 7.5, and your
    sample mean is 6.0, with a standard deviation of 3.5, calculate the
    $p$-value

    \vfill

  \item Using a significance level of 0.05, compare your $p$-value from
    part (b) to your significance level. Do you reject or fail to
    reject a null hypothesis?

    \vfill

  \end{subproblem}


\clearpage

\item Perform the following hypothesis tests. Clearly state your
  the $p$-values and whether or not you would reject the null
  hypothesis at the 95\% level.

  \begin{subproblem}
    \item $H_0: ~ \mu=10.2$, $H_a: ~ \mu > 10.2$, $z^*=1.67$.
      \vfill

    \item $H_0: ~ \mu=62.1$, $H_a: ~ \mu \neq 62.1$, $z^*=-1.90$.

      \vfill

    \item $H_0: ~ \mu=321.1$, $H_a: ~ \mu < 321.1$, $z^*=-1.40$.

      \vfill

  \end{subproblem}


\end{problem}

\actTitle{Hypothesis Test Examples} 

\begin{problem}
\item You are considering making a loan to a company that uses
  injection molding to produce golf tees. Prior to approving the loan
  you want to insure that the company's production process is
  satisfactory. The mean weight of the tees should be .250 ounces. A
  random set of tees are sampled, and the weights are given below (in ounces): \\
  \begin{tabular}{rrrrrrrr}
    0.256 & 0.246 & 0.246 & 0.262 & 0.258 & 0.249 & 0.255 & 0.251 \\
    0.246 & 0.254 & 0.253 & 0.244 & 0.243 & 0.248 & 0.247 & 0.240
  \end{tabular}

  \begin{subproblem}
  \item What is the appropriate distribution? 
  \item Organize the relevant information: \\ [30pt]
  \item State the hypothesis test: \\ [20pt]
  \item State you conclusion and justify your results.

    \vfill
  \end{subproblem}


    \vfill
    \clearpage

\item Your firm is considering investing in a sports drink
  company. The company claims that the caffeine levels of its drinks
  are above the industry average of 195 mg/l.\footnote{Caffeinated
    Sports Drink: Ergogenic Effects and Possible Mechanisms,
    \textit{International Journal of Sports Nutrition and Exercise
      Metabolism}, 2007, 17, pp. 35-55.} Before investing in the firm
  you want to confirm that this is indeed the case. Bottles of the
  drink are purchased from random locations and tested. The
  concentration of caffeine for each sample is given below. Are the
  company's claims correct? \\
  \begin{tabular}{rrrrrrrr}
    230.7 & 176.2 & 204.9 & 260.8 & 204.4 & 217.3 & 190.4 & 170.1 \\
    178.4 & 184.2 & 176.2 & 196.2 & 258.2 & 219.6 & 219.0 & 200.4
  \end{tabular}


  \begin{subproblem}
  \item What is the appropriate distribution? 
  \item Organize the relevant information: \\ [30pt]
  \item State the hypothesis test: \\ [20pt]
  \item State you conclusion and justify your results.

    \vfill
  \end{subproblem}

  \clearpage

\end{problem}
