\documentclass[12pt]{article}
\usepackage{multicol}

\oddsidemargin=0.0in
\evensidemargin=0.0in
\textwidth=6in
\topmargin=-1in
\textheight=9.5in


\begin{document}

Bivariate data

The opening values for the DOW are given below as well as the opening
prices for stocks in Eli Lilly.

\begin{multicols}{2}
  \begin{tabular}{l|l|l}
    Date & DOW & Eli Lilly \\ \hline
    2010-01-04 & 28.20 & 35.77 \\
    2011-01-03 & 34.49 & 35.17 \\
    2012-01-03 & 29.28 & 41.95 \\
    2013-01-02 & 32.99 & 49.94 \\
    2014-01-02 & 44.20 & 50.97
  \end{tabular}
  \columnbreak
  \begin{eqnarray*}
    \bar{x} & = & 33.83 \\
    \bar{y} & = & 42.76 \\
    s_{xx}  & = & 161.08 \\
    s_{yy}  & = & 226.08 \\
    s_{xy}  & = & 117.14
  \end{eqnarray*}
\end{multicols}

\begin{enumerate}
\item Your colleague wants to know if the DOW is a good predictor for
  the price of Eli Lilly stocks.

  \begin{enumerate}
  \item Determine the formula for the linear least squares regression
    line. Treat the value of the DOW as the $x$ variable and the price
    of Eli Lilly stocks as the $y$-value.
    \begin{eqnarray*}
      y & = & \hat{m} x + \hat{b} ~~ = ~~ 
    \end{eqnarray*}
  \item Determine the predicted values for $y$  \\
    \begin{tabular}{l|l|l|l|l}
      Date & DOW & Eli Lilly & Predicted Value & Residual\\ \hline
      2010-01-04 & 28.20 & 35.77 & & \\
      2011-01-03 & 34.49 & 35.17 & &  \\
      2012-01-03 & 29.28 & 41.95 & &  \\
      2013-01-02 & 32.99 & 49.94 & &  \\
      2014-01-02 & 44.20 & 50.97 & & 
    \end{tabular}
  \item Determine the value of $s_e$.
    \begin{eqnarray*}
      s_e & = & 
    \end{eqnarray*}
  \item Determine the predicted value when the DOW is 30.0.
    \begin{eqnarray*}
      \mathrm{Predicted Value} & = & 
    \end{eqnarray*}
  \item Determine the 95\% confidence interval for the value of Eli
    Lilly stocks when the DOW is 30.0.
    \vfill
  \end{enumerate}

\clearpage

\item The mean red blood cell counts for patients can decrease after
  radiation treatment. The weeks after treatment and mean blood cell
  count (in millions of cells per micro-litre) are given below.

\begin{multicols}{2}
  \begin{tabular}{l|l}
    Week & Red Blood Cell Count \\ \hline
    1 & 4.10 \\
    2 & 4.08 \\
    3 & 4.18 \\
    4 & 4.38
  \end{tabular}
  \columnbreak
  \begin{eqnarray*}
    \bar{x} & = & 2.5 \\
    \bar{y} & = & 4.185 \\
    s_{xx}  & = & 5 \\
    s_{yy}  & = & 0.0563 \\
    s_{xy}  & = & 0.47
  \end{eqnarray*}
\end{multicols}
  
  \begin{enumerate}
  \item Determine the formula for the linear least squares regression
    line. Treat the number of weeks as the $x$ variable and the red
    blood cell count as the $y$-value.
    \begin{eqnarray*}
      y & = & \hat{m} x + \hat{b} ~~ = ~~ 
    \end{eqnarray*}
  \item Determine the predicted values for $y$  \\
    \begin{tabular}{l|l|l|l|l}
      Week & Red Blood Cell Count & Predicted Value & Residual\\ \hline
      1 & 4.10 & & \\
      2 & 4.08 & & \\
      3 & 4.18 & & \\
      4 & 4.38 & & 
    \end{tabular}
  \item Determine the value of $s_e$.
    \begin{eqnarray*}
      s_e & = & 
    \end{eqnarray*}
  \item Determine the predicted value after 3.5 weeks.
    \begin{eqnarray*}
      \mathrm{Predicted Value} & = & 
    \end{eqnarray*}
  \item Determine the 95\% confidence interval for the red blood cell
    count after 3.5 weeks.
    \vfill
  \end{enumerate}


\end{enumerate}

\end{document}

%%% Local Variables: 
%%% mode: latex
%%% TeX-master: t
%%% End: 
