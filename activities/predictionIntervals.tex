
\preClass{More Basic Calculations on Bivariate Data}


\begin{problem}
\item Examples of calculating sample statistics on bivariate data.

  \begin{subproblem}
  \item 
    \begin{eqnarray}
      y & = 3x + 4.
    \end{eqnarray}
    \vfill
  \end{subproblem}


\end{problem}



\actTitle{Prediction Intervals For Bivariate Data}


The opening values for the DOW are given below as well as the opening
prices for stocks in Eli Lilly.

\begin{multicols}{2}
  \begin{tabular}{l|l|l}
    Date & DOW & Eli Lilly \\ \hline
    2010-01-04 & 28.20 & 35.77 \\
    2011-01-03 & 34.49 & 35.17 \\
    2012-01-03 & 29.28 & 41.95 \\
    2013-01-02 & 32.99 & 49.94 \\
    2014-01-02 & 44.20 & 50.97
  \end{tabular}
  \columnbreak
  \begin{eqnarray*}
    \bar{x} & = & 33.83 \\
    \bar{y} & = & 42.76 \\
    s_{xx}  & = & 161.08 \\
    s_{yy}  & = & 226.08 \\
    s_{xy}  & = & 117.14
  \end{eqnarray*}
\end{multicols}

\begin{problem}
\item Your colleague wants to know if the DOW is a good predictor for
  the price of Eli Lilly stocks.

  \begin{subproblem}
  \item Determine the formula for the linear least squares regression
    line. Treat the value of the DOW as the $x$ variable and the price
    of Eli Lilly stocks as the $y$-value.
    \begin{eqnarray*}
      y & = & \hat{m} x + \hat{b} ~~ = ~~ 
    \end{eqnarray*}
  \item Determine the predicted values for $y$  \\
    \begin{tabular}{l|l|l|l|l}
      Date & DOW & Eli Lilly & Predicted Value & Residual\\ \hline
      2010-01-04 & 28.20 & 35.77 & & \\
      2011-01-03 & 34.49 & 35.17 & &  \\
      2012-01-03 & 29.28 & 41.95 & &  \\
      2013-01-02 & 32.99 & 49.94 & &  \\
      2014-01-02 & 44.20 & 50.97 & & 
    \end{tabular}
  \item Determine the value of $s_e$.
    \begin{eqnarray*}
      s_e & = & 
    \end{eqnarray*}
  \item Determine the predicted value when the DOW is 30.0.
    \begin{eqnarray*}
      \mathrm{Predicted Value} & = & 
    \end{eqnarray*}
  \item Determine the 95\% confidence interval for the value of Eli
    Lilly stocks when the DOW is 30.0.
    \vfill
  \end{subproblem}

\clearpage

\item The mean red blood cell counts for patients can decrease after
  radiation treatment. The weeks after treatment and mean blood cell
  count (in millions of cells per micro-litre) are given below.

\begin{multicols}{2}
  \begin{tabular}{l|l}
    Week & Red Blood Cell Count \\ \hline
    1 & 4.10 \\
    2 & 4.08 \\
    3 & 4.18 \\
    4 & 4.38
  \end{tabular}
  \columnbreak
  \begin{eqnarray*}
    \bar{x} & = & 2.5 \\
    \bar{y} & = & 4.185 \\
    s_{xx}  & = & 5 \\
    s_{yy}  & = & 0.0563 \\
    s_{xy}  & = & 0.47
  \end{eqnarray*}
\end{multicols}
  
  \begin{subproblem}
  \item Determine the formula for the linear least squares regression
    line. Treat the number of weeks as the $x$ variable and the red
    blood cell count as the $y$-value.
    \begin{eqnarray*}
      y & = & \hat{m} x + \hat{b} ~~ = ~~ 
    \end{eqnarray*}
  \item Determine the predicted values for $y$  \\
    \begin{tabular}{l|l|l|l|l}
      Week & Red Blood Cell Count & Predicted Value & Residual\\ \hline
      1 & 4.10 & & \\
      2 & 4.08 & & \\
      3 & 4.18 & & \\
      4 & 4.38 & & 
    \end{tabular}
  \item Determine the value of $s_e$.
    \begin{eqnarray*}
      s_e & = & 
    \end{eqnarray*}
  \item Determine the predicted value after 3.5 weeks.
    \begin{eqnarray*}
      \mathrm{Predicted Value} & = & 
    \end{eqnarray*}
  \item Determine the 95\% confidence interval for the red blood cell
    count after 3.5 weeks.
    \vfill
  \end{subproblem}


\end{problem}

\actTitle{Hypothesis Tests and Bivariate Data}

\begin{problem}
\item The global advertising revenue given in the table
  below\footnote{http://www.statista.com/statistics/237797/total-global-advertising-revenue/}:  \\
  \begin{tabular}{rr}
    Year & Revenue (billions of USD) \\
    2007 & 495.36 \\
    2008 & 491.05 \\
    2009 & 438.25 \\
    2010 & 468.74 \\
    2011 & 485.70 \\
    2012 & 512.92
  \end{tabular}

  \begin{subproblem}
    \item Determine the 95\% confidence interval for the slope.
      \vfill
    \item Perform an hypothesis test to determine if there is a
      relationship between the year and the revenue.
      \vfill
    \item Repeat the previous calculations but ignore the years 2007 and 2008.
      \vfill
  \end{subproblem}

  \clearpage

\item The revenue for total print and online advertising is given in
  the table
  below\footnote{http://www.stateofthemedia.org/2013/newspapers-stabilizing-but-still-threatened/newspapers-by-the-numbers/}: \\
  \begin{tabular}{rr}
    Year & Revenue (millions of USD) \\
    2003 & 46,155 \\
    2004 & 48,244 \\
    2005 & 49,435 \\
    2006 & 49,275 \\
    2007 & 45,375 \\
    2008 & 37,848 \\
    2009 & 24,564 \\
    2010 & 25,838 \\
    2011 & 23,941 \\
    2012 & 22,314
  \end{tabular}

  \begin{subproblem}
    \item Determine the 95\% confidence interval for the slope.
      \vfill
    \item Perform an hypothesis test to determine if there is a
      relationship between the year and the revenue.
      \vfill
    \item Repeat the previous calculations but only use the years 2003 to 2006.
      \vfill
    \item Repeat the previous calculations but only use the years 2007 to 2012.
      \vfill
  \end{subproblem}

  \clearpage

\item The civilian unemmployment rate is given in the following
  table\footnote{\url{http://research.stlouisfed.org/fred2/series/UNRATE/\#}}:

  \begin{tabular}{rrr}
    Year & Month & Unemployment Rate \\
    2008 & 01    & 5.0 \\
    2008 & 02    & 4.9 \\
    2008 & 03    & 5.1 \\
    2008 & 04    & 5.0 \\
    2008 & 05    & 5.4 \\
    2008 & 06    & 5.8 \\
    2008 & 07    & 5.8 \\
    2008 & 08    & 6.1 \\
    2008 & 09    & 6.1 \\
    2008 & 10    & 6.5 \\
    2008 & 11    & 6.8 \\
    2008 & 12    & 6.8
  \end{tabular}

  \begin{subproblem}
    \item Determine the 95\% confidence interval for the slope.
      \vfill
    \item Perform an hypothesis test to determine if there is a
      relationship between the month and the uneployment rate.
      \vfill
    \item Repeat the previous calculations but only use the first 6 months.
      \vfill
    \item Repeat the previous calculations but only use the second 6 months.
      \vfill
  \end{subproblem}


\end{problem}