
\preClass{More Basic Calculations on Bivariate Data}


\begin{problem}
\item Examples of calculating sample statistics on data with two levels.

  \begin{subproblem}
  \item 
    \begin{eqnarray}
      y & = 3x + 4.
    \end{eqnarray}
    \vfill
  \end{subproblem}


\end{problem}



\actTitle{Analysis of Variance - Patient Hand Strength}

The American Society of Hand Therapists has a set of guidelines and
tables associated with grip strength. A group of hand therapists
decide to run their own experiment in response to see how their own
patients compare with each other. They choose a group of patients at
random and measure their grip strength using a dynamometer.

They obtain the following results for strength in their patient's
left hands: \\
\begin{tabular}{lll@{\hspace{4em}}lll}
Age Range & Sex & Force (lb.) & Age Range & Sex & Force (lb.) \\
40-44 & F & 89.8 & 20-24 & F & 63.5 \\
30-34 & M & 57.7 & 20-24 & M & 118.9 \\
40-44 & F & 39.2 & 40-44 & M & 105.3 \\
20-24 & F & 66.0 & 30-34 & F & 66.2 \\
30-34 & M & 87.4 & 20-24 & F & 72.6 \\
40-44 & F & 67.9 & 30-34 & M & 75.5 \\
40-44 & F & 53.0 & 40-44 & M & 106.2 \\
20-24 & M & 52.3 & 40-44 & F & 56.2 \\
30-34 & M & 98.8 & 40-44 & F & 83.6
\end{tabular}

% s <- list(
%    a = c("40-44","30-34","40-44","20-24","30-34","40-44","40-44",
%          "20-24","30-34","20-24","20-24","40-44","30-34","20-24",
%          "30-34","40-44","40-44","40-44"),
%
%    g = c("F","M","F","F","M","F","F","M","M","F","M","M","F","F","M","M","F","F"),
%
%    s = c(89.8,57.7,39.2,66.0,87.4,67.9,53.0,52.3,98.8,63.5 ,118.9,105.3,66.2 ,72.6,75.5,106.2 ,56.2,83.6)
%    )


\begin{problem}
  \item Organize and collect the data into two groups, male and
    female. 
    \vfill

  \item Determine the numbers of degrees of freedom for the male and
    female groups as well as the total number of degrees of freedom.

    \vspace{5em}

  \item The staff at the practice believe that there is a difference
    in the grip strengths for the left hand based on the patient's
    sex. State the hypothesis test that they will use.

    \vfill

    \clearpage
    
  \item Determine the values for the sum of the squares, SSTr, SST,
    and SSE.  Verify that SST = SSTr + SSE.

    \vfill

  \item Fill in the analysis of variance table. 

    \begin{tabular}{|l|l|l|l|l|l|} \hline
      Source & ~df~ & Sum of Squares & Mean Squares & F-statistic & $p$-value \\ \hline
      Treatments & & & & & \\ \hline
      Error & & & & \cellcolor{black!75} &  \cellcolor{black!75} \\ \hline
      Total & & & \cellcolor{black!75} & \cellcolor{black!75} & \cellcolor{black!75}  \\ \hline
    \end{tabular}

  \item State the conclusion for your hypothesis test.

    \vspace{5em}

    \clearpage

  \item Repeat the previous analysis but use the three different age groups instead.

    \clearpage
    

\end{problem}


\actTitle{Analysis of Variance - Mutual Fund Returns}

Your firm wishes to perform a comparison of the annual return rates
for mutual funds. The mutual funds are classified as being small cap,
medium cap, or large cap funds. They are also classified as being
either value or growth funds.

A randomly chosen set of funds are chosen, and their annual return
rates are given below: \\
\begin{tabular}{lll@{\hspace{4em}}lll}
Size   & Type   & Return Rate & Size & Type & Return Rate  \\
large  & value  & 5.7   & large  & value  & 10.1 \\ 
large  & value  & 14.7  & small  & value  & 9.1 \\
large  & growth & 40.6  & large  & growth & 20.8 \\
medium & value  & 18.7  & small  & value  & 17.6 \\
small  & growth & -24.1 & large  & value  & 7.9 \\
medium & value  & 13.3  & medium & value  & 13.2 \\
medium & growth & 13.0  & small  & value  & 12.3 \\
small  & growth & 16.6  & small  & value  & 18.0 \\
large  & growth & 8.9   & medium & value  & 22.4
\end{tabular}

%r <- list(
%    s=c("large","large","large","medium","small","medium","medium","small","large","large","small","large","small","large","medium","small","small","medium"),
%
%    t = c("value","value","growth","value","growth","value","growth","growth","growth","value","value","growth","value","value","value","value","value","value"),
%
%    r = c( 5.7 , 14.7 , 40.6 , 18.7 , -24.1, 13.3 , 13.0 , 16.6 , 8.9 , 10.1, 9.1, 20.8, 17.6, 7.9, 13.2, 12.3, 18.0, 22.4)
%)


\begin{problem}
  \item Organize and collect the data into two groups, growth  and
    value. 
    \vfill

  \item Determine the numbers of degrees of freedom for the growth and
    value groups as well as the total.

    \vspace{5em}

  \item The staff at the firm believe that there is a difference in
    the annual return rates between the growth and value stocks. State
    the hypothesis test that they will use.

    \vfill

    \clearpage
    
  \item Determine the values for the sum of the squares, SSTr, SST,
    and SSE.  Verify that SST = SSTr + SSE.

    \vfill

  \item Fill in the analysis of variance table. 

    \begin{tabular}{|l|l|l|l|l|l|} \hline
      Source & ~df~ & Sum of Squares & Mean Squares & F-statistic & $p$-value \\ \hline
      Treatments & & & & & \\ \hline
      Error & & & & \cellcolor{black!75} &  \cellcolor{black!75} \\ \hline
      Total & & & \cellcolor{black!75} & \cellcolor{black!75} & \cellcolor{black!75}  \\ \hline
    \end{tabular}

  \item State the conclusion for your hypothesis test.

    \vspace{5em}

    \clearpage

  \item Repeat the previous analysis but use the three different sizes
    (small, medium, and large cap) instead.

    \clearpage
    

\end{problem}
