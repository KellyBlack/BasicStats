
\preClass{Counting}


\begin{problem}
\item Connecting Coin tossing with Discrete Random Variables

  \begin{subproblem}
  \item 
    \begin{eqnarray}
      y & = 3x + 4.
    \end{eqnarray}
    \vfill
  \end{subproblem}


\end{problem}


\actTitle{Discrete Random Variables} 

\textit{You are not expected to finish this during the class
  period. Anything you don't finish is good to practice at home.}

\begin{problem}
  \item Please define the following terms in your own words: (1 or 2 sentences)

    \begin{subproblem}
      \item Random Variable:

        \vspace{4em}

      \item Discrete Random Variable:

        \vspace{4em}

    \end{subproblem}

    \begin{definition}
      The probability mass function (PMF) for a discrete random
      variable is the set of probabilities for all possible outcomes.
    \end{definition}

  \item How would you graph a probability mass function?

    \vspace{4em}

  \item List three examples of discrete random variables.

    \vfill

    \clearpage

  \item Toss a coin and let Y equal the number of tails. Determine and
    express the PMF  of Y.

    \vfill


  \item Toss a coin twice and let W equal the number of
    tails. Determine and express the PMF of W.

    \vfill


  \item A cross country coach tosses a fair six sided fair dice twice
    and determines the sum of the two rolls. The members of the team
    must then run the distance from the sum measured in miles.
    Determine and then graph the PMF of the number of miles run by the
    team.

    \vfill


\end{problem}


\actTitle{Probability Mass Functions} 


\begin{problem}


\item According to the
  IRS\footnote{\url{http://www.irs.gov/PUP/newsroom/FY\%202013\%20Enforcement\%20and\%20Service\%20Results\%20--\%20WEB.pdf}}
  there were 140,537,584 tax returns filed for people whose annual
  income was less than \$200,000 in 2012 and there were 1,232,972
  audits. Suppose that a return for families who made less than
  \$200,000 is picked at random. Determine and state the PMF
  associated with whether or not there is audit associated with the
  return.

  \vfill

\item According to the IRS\footnotemark[1] there were 5,281,804 tax returns filed for
  people whose annual income was more than \$200,000 in 2012 and there
  were 171,959 audits. Suppose that a return for families who made
  more than \$200,000 is picked at random. Determine and state the PMF
  associated with whether or not there is audit associated with the
  return.

  \vfill

\item According to the IRS\footnotemark[1] there were 363,386 tax returns filed for
  people whose annual income was more than \$1,000,000 in 2012 and
  there were 39,421 audits. Suppose that a return for families who
  made more than \$1,000,000 is picked at random. Determine and state
  the PMF associated with whether or not there is audit associated
  with the return.

  \vfill

\item Determine the PMF associated with tax audits for people making
  between \$200,000 and \$1,000,000.

  \vfill

\clearpage

\item It is estimated that roughly 91\% of travel agencies charge a
  service fee when arranging
  travel\footnote{\url{http://www.asta.org/News/content.cfm?ItemNumber=1985}}. 
  \begin{subproblem}
  \item One hundred travel agencies are sampled at random, and the
    total number that charge a service fee is determined. What is the
    probability that exactly 91 charge a fee?

    \vfill

  \item One hundred travel agencies are sampled at random, and the
    total number that charge a service fee is determined. What is the
    probability that more than 91 charge a fee?

    \vfill

  \item One hundred travel agencies are sampled at random, and the
    total number that charge a service fee is determined. What is the
    standard deviation for the total?

    \vfill

  \item You are asked to contact travel agencies chosen at random and
    ask if they charge a service fee. You want the standard deviation
    to be 0.1. How many travel agencies will you need to contact?

    \vfill

  \end{subproblem}



\end{problem}
