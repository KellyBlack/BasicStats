\documentclass[12pt]{article}

\oddsidemargin=0.0in
\evensidemargin=0.0in
\textwidth=6in
\topmargin=-1in
\textheight=9.5in

\setlength{\parindent}{0pt}

\begin{document}

 \textbf{Probability: Discrete Random Variables} \\
\\
\textit{You are not expected to finish this during the class period. Anything you don't finish is good to practice at home.}  \\ [5pt]

\textbf{Please define the following terms in your own words:} (1 or 2 sentences) \\ [5pt]

 \textit{Random Variable:} \\
\\
\textit{Discrete Random Variable:}\\


\textbf{Definittions:}  \\ [5pt]

\textit{Probability Mass Function: } (PMF) The probability that a discrete random variable is exactly equal to some value.  A PMF is equal to values between 0 and 1 with the sum of those values adding up to 1. \\  [12pt]

How would you graph a probability mass function? \\ [12pt]


 \textbf{Examples: } \\

\begin{enumerate}

\item List three examples of discrete random variables.\\[12pt]


\item Toss a coin and let Y equal the number of tails. Find the PMF of Y. \\[12pt]


\item Toss a coin twice and let Y equal the number of tails. Find the PMF of Y.\\[12pt]


\item You toss two six sided fair dice. The number of miles you have to run is equal to the sun on the dice and Y is equal to the number of miles you have to run after tossing the dice once. Find and graph the PMF of Y. \\[12pt]



\end{enumerate}

\end{document}




