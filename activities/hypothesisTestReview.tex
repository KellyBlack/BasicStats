
\preClass{Calculating probabilities associated with sample distributions.}


\begin{problem}
\item Examples of calculating probabilities of sample means.

  \begin{subproblem}
  \item 
    \begin{eqnarray}
      y & = 3x + 4.
    \end{eqnarray}
    \vfill
  \end{subproblem}


\end{problem}

\actTitle{Hypothesis Testing for Proportions}

\begin{problem}
\item 200 students are surveyed about their fondness for sports at
  Clarkson University. Of those 80 say they like sports. Is this
  proportion low for college students? Let $P_{0}$ be the proportion
  of college students who like sports.
  \begin{subproblem}
  \item Write the null and alternative hypothesis below
    \vspace{3em}

  \item The test statisitic can be calculated as:
    \begin{eqnarray*}
      z=\frac{\hat{p}-p}{\sqrt{\frac{p(1-p)}{n}}}
    \end{eqnarray*}
    Another survey finds $\hat{p} = .265$, calculate $z^*$.

    \vfill

  \item The $p$-value can be calculated as $P(z \ge |z^*|).$ Calculate
    the $p$-value and compare it to $\alpha = 0.05$. Do you accept or
    reject $H_{0}$?

    \vfill

  \end{subproblem}

  \clearpage

  \item The manager of the customer service division said that their
    consumer satisfaction has improved. A random sample of 400 people
    who have called for service is sampled, and 280 reported that they
    were pleased with the service. Last year it was estimated that
    65\% of the people using the service were pleased with their
    experience.

    State the hypothesis test, organize and calculate any necessary
    values, sketch an appropriate plot, determine whether or not you
    can reject the null hypothesis, and then make a formal statement
    of your conclusions.

    \vfill
    \clearpage




\item For the comparison of two proportions the test statistic is
  defined differently, but the concept is still the same as a single
  proportion. Let $d_{0}$ be the difference between the two
  proportions (if you claim that the proportions are the same then
  $d_{0}$ =0). Before we defined $\hat{p}$ as
  $\hat{p}=\frac{x}{n}$. This time we'll say that
  $\hat{p_{1}} = \frac{x_{1}}{n_{1}}$ and similarly
  $\hat{p_{2}} = \frac{x_{2}}{n_{2}}$. The combined proportion, which
  we'll call $\hat{P}$ will be equal to
  $\hat{P} = \frac{x_{1}+x_{2}}{n_{1}+n_{2}}$. This means that the
  test statisitic is:
  \begin{eqnarray*}
    z=\frac{(\hat{p_{1}}-\hat{p_{2}})-d_{0}}{
    \sqrt{\hat{P}(1-\hat{P})\left(\frac{1}{n_{1}}+\frac{1}{n_{2}}\right)}}
  \end{eqnarray*}

  \begin{subproblem}
  \item Given the values below, calculate the $z$-statistic
    \begin{itemize}
    \item $n_{1} = 140 $
    \item $n_{2} = 160 $
    \item $x_{1} = 90 $
    \item $x_{2} = 140$
    \end{itemize}

  \item Test the hypothesis that $d_{0} = 0$ by calculating the
    $p$-value (this is done the same was as a single proportion
    problem).

    \vfill

  \item Give the $p$-value, do you reject or fail to reject $H_{0}$?

    \vfill

  \end{subproblem}

  \clearpage


\end{problem}


\actTitle{Survey of Hypothesis Testing}

\begin{problem}
\item A new procedure is tested for patients recovering from knee
  replacement surgery. Currently patients have a mean range of motion
  of ninety degrees after four weeks. A randomly chosen group of
  twenty-three patients take part in a new therapy schedule. Four
  weeks after their surgery the sample mean for the range of motion is
  ninety-three degrees with a sample standard deviation of nine
  degrees. Did the new schedule increase the range of motion?

  \begin{subproblem}
  \item What is the appropriate distribution? 
  \item Organize the relevant information: \\ [30pt]
  \item State the hypothesis test: \\ [20pt]
  \item Determine whether or not you can reject the null hypotheis:
    \vfill
  \item Provide a formal statement of your conclusion: 
  \end{subproblem}

\clearpage

\item A new procedure is tested for patients recovering from knee
  replacement surgery. The standard of care is to determine whether or
  not a patient's range of motion exceeds ninety degrees four weeks
  after their surgery, and currently sixty-five percent of all
  patients meet that standard of care. A randomly chosen group of
  fifty-eight patients take part in a new therapy schedule. Four weeks
  after their surgery forty-four of the patients had a range of motion
  greater than ninety degrees. Did the new schedule increase the range
  of motion?

  \begin{subproblem}
  \item What is the appropriate distribution? 
  \item Organize the relevant information: \\ [30pt]
  \item State the hypothesis test: \\ [20pt]
  \item Determine whether or not you can reject the null hypotheis:
    \vfill
  \item Provide a formal statement of your conclusion: 
  \end{subproblem}

\clearpage

\item A new procedure will be tested for patients recovering from knee
  replacement surgery. Currently patients have a mean range of motion
  of ninety degrees after four weeks with a standard deviation of nine
  degrees. A randomly chosen group of patients will take part in a new
  therapy schedule. How many patients will be required to take part in
  the study if we wish to reject the null hypothesis if the error in
  the means is more than three degrees?

  \begin{subproblem}
  \item What is the appropriate distribution? 
  \item Organize the relevant information: \\ [30pt]
  \item State the hypothesis test: \\ [20pt]
  \item Determine the required number of patients.
    \vfill
  \end{subproblem}




\end{problem}

