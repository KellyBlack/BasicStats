\preClass{Counting Number of Approaches}


\begin{problem}
\item Examples of uniform probability distributions.

  \begin{subproblem}
  \item 
    \begin{eqnarray}
      y & = 3x + 4.
    \end{eqnarray}
    \vfill
  \end{subproblem}


\end{problem}


\actTitle{Data Collection} 


\begin{problem}

\item An investment bank will conduct a stress test to determine how
  it can react to swings in the markets. As part of the test the
  managers within different offices are asked the following questions:
  \begin{itemize}
  \item What is the total balance in your reserve accounts?
  \item What is the mean error in your return projections over the
    past year?
  \item On a scale from one to five how would you rate your current
    risk? (One means low risk and five is high risk.)
  \item On a scale from one to five how would you rate your current
    operating process? (One means inefficient and five is highly
    efficient.)
  \item When evaluating a fund which aspects of the fund are most
    important in terms of deciding your investment levels? (Circle all
    that apply: \\
    Price; Derivatives; Credit Scoring; Financial Activity; Open
    Options; Dynamic Financial Analysis Projections.
  \end{itemize}

  For each question above determine the nature of the data that will
  result from the survey. Determine if it qualitative, quantitative,
  and if it is quantitative determine if it is ordinal or continuous
  data.

  \vfill

  \clearpage

\item The human resources department examines applications for
  employment. Each applicant answers a number of questions:

  \begin{itemize}
  \item What is your gender? 
  \item What is your age?
  \item What is your racial background?
  \item What is your GPA?
  \item What is your total number of credit hours completed?
  \item What is your expected salary?
  \end{itemize}

  For each question above determine the nature of the data that will
  result from the survey. Determine if it qualitative, quantitative,
  and if it is quantitative determine if it is ordinal or continuous
  data.

  \vfill

  \clearpage

\item Firms that are in bankruptcy are randomly sampled. For each firm
  the reorganization plan in place and the time they have been in
  bankruptcy (in months) is recorded. The data is given below:

  \begin{tabular}{ll|ll|ll} % @{\hspace{3em}
    Plan & Time  (months) & 
    Plan & Time  (months) & 
    Plan & Time  (months)\\ \hline
    None    & 10.1 & Prepack & 4.7  & Prepack & 7.3 \\
    Prepack & 6.7  & Prepack & 7    & Joint   & 7.4 \\
    None    & 10.5 & Joint   & 3.3  & Prepack & 5.5 \\
    Joint   & 5.8  & None    & 11.2 & Joint   & 7.4 \\
    Joint   & 7.5  & Prepack & 8.6  & Joint   & 7.2 
  \end{tabular}

  \begin{subproblem}
  \item Determine the frequency of firms using each type of bankruptcy
    plan.
    \vfill
  \item Determine the frequency of firms whose time in bankruptcy is less
    than 6 months. 
    \vfill
  \item For each type of plan determine the frequency of firms whose
    time in bankruptcy is less than 6 months, between 6 and 8 months,
    between 8 and 10 months, and between 10 and 12 months. Express the
    results as a table with each row being one of the plan types, and
    the columns are the time in bankruptcy.
    \vfill
  \end{subproblem}


\clearpage

\item The human resources division conducts a  survey for people who
  have applied for a position in the past year. Each person is asked a
  number of questions. Two questions in particular are examined:
  \begin{enumerate}
  \item On a scale from one to four how would you rate the feedback
    you received from our corporation? (One is poor and four is
    excellent.)
  \item On a scale from one to four how would rate your overall
    experience?  (One is poor and four is excellent.)
  \end{enumerate}

  \begin{tabular}{ll|ll|ll|ll|ll} % @{\hspace{3em}
    Q1 & Q2 & Q1 & Q2 & Q1 & Q2 & Q1 & Q2 & Q1 & Q2 \\ \hline
    2 & 3 & 4 & 4 & 4 & 2 & 3 & 4 & 2 & 3 \\
    3 & 3 & 1 & 3 & 4 & 4 & 2 & 4 & 4 & 3 \\
    4 & 1 & 1 & 1 & 4 & 3 & 2 & 2 & 3 & 3 \\
    4 & 1 & 3 & 2 & 4 & 4 & 3 & 3 & 4 & 2 \\
    4 & 3 & 3 & 3 & 4 & 3 & 3 & 1 & 1 & 2
  \end{tabular}

  \begin{subproblem}
  \item Determine the frequency of people answering 4 on question one
    and 4 on question two.
    \vspace{2em}
  \item Make a table in which each row corresponds to the possible
    answers on question one, and each column corresponds to the possible
    answers on question two. For each entry in the table determine the
    frequency of occurrences for the number of people with the given
    pair of responses.

    \vfill

  \end{subproblem}

  
\end{problem}


\actTitle{Experimental Design} 

\begin{problem}
\item The mean sales for facilities will be examined. Your firm's
  facilities can be categorized as being stand alone, in a strip mall,
  or in an enclosed mall. The stores can also be categorized as being
  boutique, small, or large. Finally the stores can be categorizes as
  being for appointment, open, office, or call center.
    \begin{subproblem}
      \item How many different overall categories are there?
        \vfill
      \item If the study requires that there will be 20 stores for
        each category how many stores must be chosen?
        \vfill
      \item Describe the type of data that will be calculated for each
        category of store.
        \vspace{2em}
    \end{subproblem}

\clearpage

\item The effectiveness of a rehabilitation treatment after surgery
  will be tested. The team wishes to test the effectiveness for one,
  two, and three sessions per day. They also wish to test the
  effectiveness for sessions lasting 30 minutes, 60 minutes, 75
  minutes, and 90 minutes. Finally, the treatment can be given over a
  one week period or a two week period. At the end of the treatment
  each patient's improvement is rated as being poor, adequate, or excellent.
    \begin{subproblem}
      \item How many different overall categories are there?
        \vfill
      \item If the study requires that there will be 15 patients for
        each category how many patients must be chosen?
        \vfill
      \item Describe the type of data that will be calculated for each
        category.
        \vspace{2em}
    \end{subproblem}

\clearpage

\item A study will be conducted to determine the cost of vandalism
  over a year. Your firm will randomly select stores and ask how much
  money was spent in repairs due to vandalism. It is estimated that
  the standard deviation of the sample mean for the costs is
  \begin{eqnarray*}
    \hat{\sigma} & = & \frac{120.5}{\sqrt{n}},
  \end{eqnarray*}
  where $n$ is the number of stores to sample.

  \begin{subproblem}
    \item Determine how many samples are required if the standard
      deviation of the sample mean should be 10.0.

      \vfill

    \item Determine how many samples are required if the standard
      deviation of the sample mean should be 5.0.

      \vfill

    \item Repeat the previous calculations if the standard deviation
      for the sample mean is
      \begin{eqnarray*}
        \hat{\sigma} & = & \frac{130.5}{\sqrt{n}}.
      \end{eqnarray*}

      \vfill


  \end{subproblem}

\end{problem}


%  LocalWords:  Prepack
