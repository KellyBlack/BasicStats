\preClass{Counting Number of Approaches}


\begin{problem}
\item Examples of uniform probability distributions.

  \begin{subproblem}
  \item 
    \begin{eqnarray}
      y & = 3x + 4.
    \end{eqnarray}
    \vfill
  \end{subproblem}


\end{problem}


\actTitle{Plotting Discrete Data} 


\begin{problem}

\item A new product is tested in the Midwest. People are chosen at
  random and asked to try the product. They are asked to provide their
  initial reaction on a scale from 1 to 5, where 1 means they would
  not purchase it and 5 means they are strongly wish to purchase the
  product.

  Their results are given below:\\
  \begin{tabular}{rrrrrrrrrrrrrrrr}
    3 & 3 & 4 & 3 & 2 & 2 & 4 & 3 &
    1 & 4 & 3 & 3 & 3 & 3 & 4 & 4 \\
    4 & 3 & 3 & 3 & 3 & 3 & 4 & 3 &
    4 & 3 & 3 & 4 & 3 & 3 & 4 & 2 \\
  \end{tabular}

  \begin{subproblem}
  \item Determine the frequency of the occurences for each possible
    response.
    \vfill
  \item Make a rough sketch of a bar plot with the responses given in
    order (1, 2, 3, 4, and 5).
    \vfill
    \clearpage
  \item Make a rough sketch of the Pareto plot for the responses.
    \vfill
  \item Provide a description of the data. What are the important features?
    \vspace{10em}
  \end{subproblem}

\clearpage

\item A clinic wishes to test a therapy for children that they have
  not used in the past. They choose a number of patients at random and
  use the Face, Legs, Activity, Cry, Consolability scale (FLACC) to
  assess the patients' pain levels. (The FLACC scale provides a number
  between 0 and 10.) After the study they have the following
  observations.

  Their results are given below:\\
  \begin{tabular}{rrrrrrrrrrrrrrrr}
    5 & 7 & 6 & 6 & 8 & 5 & 6 & 5 &
    4 & 7 & 8 & 5 & 6 & 5 & 8 & 3 \\
    6 & 6 & 5 & 6 & 4 & 6 & 5 & 4 &
    5 & 5 & 3 & 6 & 8 & 5 & 6 & 4 \\
    8 & 6 & 8 & 5 & 6 & 8 & 3 & 6 
  \end{tabular}

  \begin{subproblem}
  \item Determine the frequency of the occurences for each possible
    response.
    \vfill
  \item Make a rough sketch of a bar plot with the responses given in
    order (0, 1, 2, 3, through 10).
    \vfill
    \clearpage
  \item Make a rough sketch of the Pareto plot for the responses.
    \vfill
  \item Provide a description of the data. What are the important features?
    \vspace{5em}
  \end{subproblem}

\clearpage

\item Concerns about costs associated with insurance processing fees
  has been raised by the Human Resources Department. They chose a
  random group of people and rated the associated costs as being L
  (low), M (medium), or H (high). Each person was then determined to
  be either a S (smoker) or NS (non-smoker).

  Their results are given below:\\
  \begin{tabular}{rr|rr|rr|rr|rr|rr|rr|rr}
    H & S  & M & S  & M & NS & L & NS & L & S  & M & NS & M & S  & M & NS  \\
    M & S  & M & NS & M & NS & H & S  & L & NS & L & S  & M & NS & L & S  \\
    L & NS & M & S  & H & S  & L & NS & M & S  & M & NS & L & NS & H & NS \\
    M & NS & M & NS & M & S  & M & S  & M & NS & M & NS & M & NS & L & NS \\
    M & S  & L & NS & L & NS & L & S  & M & NS & M & S  & H & S  & L & S 
  \end{tabular}

  \begin{subproblem}
    \item Determine the frequency of occurrences for  each possible
      combination.
      \vfill
    \item Arrange the frequencies into a table with the costs arranged
      in rows and the smoking status in the columns.
      \vfill
  \end{subproblem}
  
\end{problem}

\actTitle{Plotting Continuous Data} 

\begin{problem}
\item The calls made by people in a call center are monitored. The
  total amount of time for each call is observed, and a number of
  calls are chosen at random.

    The call times are given in the table below:\\
    \begin{tabular}{rrrrrrrrrrrr}
      37.9 & 31.9 & 21.4 & 24.7 & 34.3 & 47.5 & 27.6 & 16.4 &
      28.7 & 31.0 & 32.6 & 37.4 \\
      31.0 & 25.5 & 35.9 & 35.1 & 27.3 & 37.9 & 31.6 & 29.0 &
      18.2 & 15.0 & 33.4 & 42.8  \\
      36.1 & 28.2 & 19.6 & 35.2 & 31.1 & 30.5 & 34.9 & 22.8 &
      29.2 & 32.0 & 28.5 & 32.9 \\
      33.9 & 25.6 & 28.6 & 26.3
    \end{tabular}
    
    \begin{subproblem}
      \item Make a stem-leaf plot of the data by rounding the numbers.
        \vfill
      \item Make a stem-leaf plot of the data by truncating the
        numbers.
        \vfill
        \clearpage
      \item Sketch a histogram using an interval width of 5.
        \vfill
      \item Sketch a histogram using an interval width of 10.
        \vfill
      \item Provide a description of the data. What are the important features?
        \vspace{5em}
    \end{subproblem}

\clearpage

  \item The heart rates in beats per minute are measured for a group
    of adults. 

    The results are given below:\\
    \begin{tabular}{rrrrrrrrrrrrr}
      73.3 & 70.8 & 75.8 & 67.5 & 67.6 & 69.0 & 73.0 & 69.8 & 70.8 & 70.5 & 69.2 & 70.1 & 69.7 \\
      69.7 & 73.8 & 73.7 & 68.9 & 69.7 & 67.9 & 66.5 & 68.7 & 72.1 & 70.1 & 67.9 & 72.1 & 66.9 \\
      66.9 & 70.2 & 73.0 & 73.0 & 71.6 & 65.7 & 72.0 & 70.9 & 73.0 & 69.8 & 66.0 & 65.9 & 71.3 \\
      71.3 & 70.2 & 68.0 & 72.3 
    \end{tabular}
    
    \begin{subproblem}
      \item Make a stem-leaf plot of the data by rounding the numbers.
        \vfill
      \item Make a stem-leaf plot of the data by truncating the
        numbers.
        \vfill
        \clearpage
      \item Sketch a histogram using an interval width of 5.
        \vfill
      \item Sketch a histogram using an interval width of 10.
        \vfill
      \item Provide a description of the data. What are the important features?
        \vspace{5em}
    \end{subproblem}

\clearpage

\item The heart rates in beats per minute are measured for a different
  group of adults.

    The results are given below:\\
    \begin{tabular}{rrrrrrrrrrrrr}
      75.4 & 76.8 & 73.9 & 72.4 & 70.9 & 70.4 & 75.4 & 79.3 & 77.0 & 77.6 & 75.1 & 74.4 & 75.1 \\
      75.1 & 73.7 & 78.9 & 73.8 & 74.7 & 82.3 & 68.8 & 76.4 & 75.7 & 70.5 & 78.0 & 75.4 & 81.1 \\
      81.1 & 71.0 & 71.1 & 76.0 & 76.1 & 78.1 & 81.3 & 67.2 & 74.6 & 74.2 & 76.2 & 72.4 & 72.4 \\
      72.4 & 71.4 & 71.7 & 69.4
    \end{tabular}
    
    \begin{subproblem}
      \item Make a stem-leaf plot of the data by rounding the numbers.
        \vfill
      \item Sketch a histogram using an interval width of 5.
        \vfill
        \clearpage
      \item Sketch a histogram using an interval width of 10.
        \vfill
      \item Is this group of people's heart rate different from the
        previous group? Justify your conclusions.

        \vspace{4em}
    \end{subproblem}


\end{problem}

