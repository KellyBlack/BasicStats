
\lecture{Continuous Random Variables}{continuous-random-variables}
\section{Continuous Random Variables}

\title{Continuous Random Variables}
\subtitle{Quantifying Outcomes}

%\author{Kelly Black}
%\institute{Clarkson University}
\date{5 September 2014}

\begin{frame}
  \titlepage
\end{frame}

\begin{frame}
  \frametitle{Outline}
  \tableofcontents[hideothersubsections,sectionstyle=show/hide]
\end{frame}


\subsection{Clicker Quiz}


\iftoggle{clicker}{%
  \begin{frame}
    \frametitle{Clicker Quiz}

    A fair, six sided die has the numbers one through six marked on
    it. I roll the die and read the number, $x$. What is the
    probability that I get a number less than 5?

    \vfill
    
    \begin{tabular}{l@{\hspace{3em}}l@{\hspace{3em}}l@{\hspace{3em}}l}
      A: 2/6 & B: 3/6 & C: 4/6  & D: 5/6
    \end{tabular}

    \vfill
    \vfill
    \vfill

  \end{frame}
}


\subsection{Continuous Random Variables}

\begin{frame}
  \frametitle{Example}

  A wheel has a circumference of two meters, and it has a marker at
  the top. I spin the wheel and read the number representing the
  length around the side from the top of the wheel where it stops and
  the marker.

\end{frame}

\subsection{Examples}

\begin{frame}{Examples}

  \begin{itemize}
  \item Pick a concrete sample at random. Record the force at which
    the sample breaks.
  \item A cow is picked at random and milked for a set period of
    time. Record the volume of milk produced.
  \end{itemize}
  
\end{frame}

\subsection{Probability Density Function}

\begin{frame}{Probability Density Function}

  
  \begin{columns}
    \column{.60\textwidth}

    \begin{definition}[Probability Density Function]
      The area under the curve, $f(x)$, between $a$ and $b$ is the
      probability we get a result between $a$ and $b$:
      \begin{eqnarray*}
        p( a \leq X \leq b ) & = & \int^b_a f(x) ~ dx.
      \end{eqnarray*}
    \end{definition}


    \column{.40\textwidth}
    \uncover<2->{%
      Properties:
      \begin{eqnarray*}
        f(x) & \geq & 0 \\
        \int_{\mathrm{domain}} f(x) ~ dx & = & 1.
      \end{eqnarray*}
    }
  \end{columns}
  
\end{frame}

\subsection{Examples}

\begin{frame}{Example}
  Is the function
  \begin{eqnarray*}
    f(x) & = & \left\{
    \begin{array}{r@{\hspace{2em}}l}
      \frac{x}{8} & 0\leq x \leq 4, \\
      0 & \mathrm{otherwise},
    \end{array}
    \right.
  \end{eqnarray*}
  a probability density function?

  \vfill

  \uncover<2->{%
    Determine the probability that the random variable returns a
    number between one and three.
  }

  \vfill

\end{frame}

\iftoggle{clicker}{%
  \begin{frame}
    \frametitle{Clicker Quiz}

    Determine the value of a constant $c$ so that
    \begin{eqnarray*}
      f(x) & = & \left\{
        \begin{array}{r@{\hspace{2em}}l}
          c & 3 \leq x \leq 7, \\
          0 & \mathrm{otherwise},
        \end{array}
      \right.
    \end{eqnarray*}
    is a probability density function.

    \vfill
    
    \begin{tabular}{l@{\hspace{3em}}l@{\hspace{3em}}l@{\hspace{3em}}l}
      A: 4 & B: 1 & C: $\frac{1}{4}$ & D: $\frac{1}{7}$
    \end{tabular}

    \vfill
    \vfill
    \vfill

  \end{frame}
}


\begin{frame}{Example}
  Is it possible to find a $c$ such that
  \begin{eqnarray*}
    f(x) & = & \frac{c}{x^2}
  \end{eqnarray*}
  is a probability density function for $x\geq 1$.
\end{frame}

\subsection{Cumulative Distribution Function}

\begin{frame}{The Cumulative Distribution Function}

  \begin{definition}[The cumulative distribution]
    \begin{eqnarray*}
      F(a) & = & p(X \leq a), \\
      & = & \int^a_{-\infty} f(s) ~ ds.
    \end{eqnarray*}
  \end{definition}
\end{frame}


\begin{frame}{Example}
  Find the cumulative distribution function for the probability
  distribution function
  \begin{eqnarray*}
    f(x) & = & \left\{
    \begin{array}{r@{\hspace{2em}}l}
      \frac{x}{8} & 0\leq x \leq 4, \\
      0 & \mathrm{otherwise}.
    \end{array}
  \right.
  \end{eqnarray*}

\end{frame}

\begin{frame}{Example}
  Is it possible to find a $c$ such that
  \begin{eqnarray*}
    f(x) & = & \left\{
    \begin{array}{r@{\hspace{2em}}l}
      c e^{-x/2} &  x \geq 0, \\
      0 & \mathrm{otherwise},
    \end{array}
    \right.
  \end{eqnarray*}
  is a probability density function?
\end{frame}

\begin{frame}{Example}
  Determine the cumulative distribution function for 
  \begin{eqnarray*}
    f(x) & = & \left\{
    \begin{array}{r@{\hspace{2em}}l}
      \half e^{-x/2} &  x \geq 0, \\
      0 & \mathrm{otherwise}.
    \end{array}
    \right.
  \end{eqnarray*}
\end{frame}



% LocalWords:  Clarkson pausesection hideallsubsections sectionstyle
%  LocalWords:  hideothersubsections
