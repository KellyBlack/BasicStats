
\lecture{Introduction to Probability}{intro-to-probability}
\section{Introduction to Probability}

\title{Probability}
\subtitle{What are the chances?}

%\author{Kelly Black}
%\institute{Clarkson University}
\date{27 January 2012}

\begin{frame}
  \titlepage
\end{frame}

\begin{frame}
  \frametitle{Outline}
  \tableofcontents[pausesection,hideothersubsections,sectionstyle=show/hide]
\end{frame}


\subsection{Clicker Quiz}


\iftoggle{clicker}{%

  \begin{frame}
    \frametitle{Clicker Quiz}

    If I flip a fair coin ten times how many tails will I get?

    \begin{tabular}{l@{\hspace{3em}}l@{\hspace{3em}}l@{\hspace{3em}}l}
      A: 4 & B: 5 & C: 6 & D: I do not know
    \end{tabular}


  \end{frame}

}




\subsection{Example}

\begin{frame}
  \frametitle{Clicker Exercise}

  Flip a coin 3 times. How many tails did you get?

  \begin{tabular}{l@{\hspace{3em}}l@{\hspace{3em}}l@{\hspace{3em}}l}
  A: 0 & B: 1 & C: 2 & D: 3
  \end{tabular}

  \only<2->%
  {
    Question: What is the probability that we get the same result if
    we do it again?
  }

\end{frame}

\begin{frame}{Recall}

  Recall from last time:

  \begin{definition}[Probability]
    ``\textbf{Probability} is the measure of the likelihood of a random
    phenomena or chance behavior. Probability describes the long-term
    proportion with which a certain \textbf{outcome} will occur in
    situations with short-term uncertainty.'' (Page 223)
  \end{definition}

  
\end{frame}


\begin{frame}{Also Recall}

  \begin{description}
  \item[Event:] Something that \textit{can} happen.
  \item[Outcome:] Something that did happen.
  \item[Experiment:] A structured activity that includes the
    measurement of the outcomes of the activity after predefined and
    intentional changes to some aspect of the events.
  \item[Observational Study:] A structured activity that includes the
    measurement of the outcomes of the activity without intentional
    changes to the events.
  \item[Expected Outcome:] The ``average'' of the possible outcomes.
  \item[Variation:] Some measure of the spread of possible outcomes.
  \end{description}
  
\end{frame}

\begin{frame}{Also Also Recall}

  ``Probability'' is an idealized notion. If we repeat the experiment
  an infinite number of times we ask what \textit{will} happen.

  \vfill

  ``Statistics'' is the study of how to interpret data. We ask what
  ``did'' happen and what does it imply about the underlying
  probabilities?

  \vfill

\end{frame}


\subsection{Events}

\begin{frame}
  \frametitle{Events}

  \begin{definition}
    An event is a possible outcome from an experiment.
  \end{definition}

\end{frame}


\begin{frame}
  \frametitle{Sample Space}

  \begin{definition}
    The sample space is the collection of all possible events.
  \end{definition}

\end{frame}


\begin{frame}
  \frametitle{Sample Space}

  There are two ways to visualize the sample space:
  \begin{itemize}
  \item Venn Diagram
  \item Tree Diagram
  \end{itemize}

\end{frame}

\subsection{Examples}

\begin{frame}{Example}

  We flip a coin two times. What are the events?

  \only<2->%
  {

    \begin{eqnarray*}
      p(\mathrm{first~T}) & = & ? \\
      p(TT) & = & ? \\
      p(TH) & = & ? \\
      p(\mathrm{second~H}) & = & ? \\
      p(\mathrm{one~T}) & = & ? \\
      p(\mathrm{two~H}) & = & ?
    \end{eqnarray*}

  }
  
\end{frame}

\begin{frame}{Clicker Quiz}
  A couple has a child. Two years later they have another child. What
  is the probability that they have one girl and one boy?

  \begin{tabular}{l@{\hspace{3em}}l@{\hspace{3em}}l@{\hspace{3em}}l}
    A: 0 & B: 1/4 & C: 1/2 & D: 1
  \end{tabular}


\end{frame}

\begin{frame}{Example}

  I roll a die twice. What is $p(\mathrm{sum}=5)$?
  
\end{frame}


\subsection{Properties of Probabilities}

\begin{frame}{Properties of Probabilities}

  Suppose that A is an event in the sample space, then
  \begin{eqnarray*}
    \begin{array}{rcccl}
      0 & \leq & P(A) & \leq & 1
    \end{array}
  \end{eqnarray*}
  
\end{frame}


% LocalWords:  Clarkson pausesection hideallsubsections
