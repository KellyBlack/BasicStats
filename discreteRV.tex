
\lecture{Random Variables}{random-variables}
\section{Random Variables}

\title{Random Variables}
\subtitle{Quantifying Outcomes}

%\author{Kelly Black}
%\institute{Clarkson University}
\date{4 September 2013}

\begin{frame}
  \titlepage
\end{frame}

\begin{frame}
  \frametitle{Outline}
  \tableofcontents[hideothersubsections,sectionstyle=show/hide]
\end{frame}


\subsection{Clicker Quiz}


\iftoggle{clicker}{%
  \begin{frame}
    \frametitle{Clicker Quiz}

    I flip a coin two times. What is the probability that I get exactly
    one tail?

    \vfill
    
    \begin{tabular}{l@{\hspace{3em}}l@{\hspace{3em}}l@{\hspace{3em}}l}
      A: 0 & B: 1/4 & C: 1/2 & D: 3/4
    \end{tabular}

    \vfill
    \vfill
    \vfill

  \end{frame}
}


\subsection{Random Variables}

\begin{frame}
  \frametitle{Random Variables}

  \begin{definition}
    A \textbf{random variable} is a variable whose outcome is a number
    and has a random component.
  \end{definition}

  \uncover<2->
  {
    \begin{definition}
      A \textbf{discrete random variable} is a random variable whose
      outcome is one of a countable number of outcomes.
    \end{definition}
  }

  \uncover<3->
  {
    \begin{definition}
      A \textbf{continuous random variable} is a random variable whose
      outcome come from a range of values.
    \end{definition}
  }

  \uncover<4->
  {
    \begin{definition}
      A \textbf{probability distribution} is a list of probabilities of
      \textit{all possible outcomes} of a random variable.
    \end{definition}
  }

\end{frame}

\subsection{Examples}

\begin{frame}{Examples of Random Variables}

  \begin{itemize}
  \item Flip a coin 3 times. Report the number of heads.
  \item Measure the voltage across a capacitor.
  \item Pick 20 people at random and ask if they are satisfied with
    Clarkson. Report the number who say ``yes.''
  \item A box contains 100 light bulbs and 3 are defective. Pull one
    at random and repeat without replacement until you find a
    defective bulb. Report the number of trials.
  \end{itemize}
  
\end{frame}



\begin{frame}{Discrete Random Variables}

  Probability Mass Function
  \begin{columns}
    \column{.40\textwidth}

    \begin{eqnarray*}
      \begin{array}{l|l}
        \mathrm{X} & p \\ \hline
        x_1 & p_1 \\
        x_2 & p_2 \\
        x_3 & p_3 \\
        \vdots & \vdots \\
        x_n & p_n
      \end{array}
    \end{eqnarray*}

    \column{.60\textwidth}
    Properties:
    \begin{eqnarray*}
      \begin{array}{rcccl}
        0 & \leq & p_i & \leq 1
      \end{array}
      \\
      p_1 + p_2 + p_3 + \cdots + p_n & = & 1.
    \end{eqnarray*}

  \end{columns}
  
\end{frame}


\begin{frame}{Example}
  Flip a coin three times. Report the number of tails. Determine the
  probability mass function.
\end{frame}

\iftoggle{clicker}{%
  \begin{frame}
    \frametitle{Clicker Quiz}

    Flip a coin three times. Report the number of tails. What is the
    probability that I get two or fewer tails?

    \vfill
    
    \begin{tabular}{l@{\hspace{3em}}l@{\hspace{3em}}l@{\hspace{3em}}l}
      A: 1/8 & B: 4/8 & C: 7/8 & D: 1
    \end{tabular}

    \vfill
    \vfill
    \vfill

  \end{frame}
}


\begin{frame}{Example}
  Suppose that the probability that $x$ packages are delivered on a
  given day is 
  \begin{eqnarray*}
    p(X=x) & = & \frac{2}{3^{x+1}},
  \end{eqnarray*}
  where $x=0,1,2,3,\ldots$.
\end{frame}

\begin{frame}{Example}

  A box contains 100 light bulbs and 3 are defective. Pull one at
  random and repeat without replacement until you find a defective
  bulb. Report the number of trials. What is the probability mass
  function?
  
\end{frame}

\subsection{Cumulative Distribution Function}

\begin{frame}{Example}
  Flip a coin three times. Report the number of tails. Determine the
  cumulative distribution function.
\end{frame}


\begin{frame}{Example}

  Roll a six sided die twice and add the results. Find the cumulative
  distribution function.

  \only<2->{
    \begin{tabular}{l|l|l|l|l|l|l|}
      \multicolumn{1}{l}{~} & \multicolumn{1}{l}{\redText{1}} & \multicolumn{1}{l}{\redText{2}} & \multicolumn{1}{l}{\redText{3}} & \multicolumn{1}{l}{\redText{4}} & \multicolumn{1}{l}{\redText{5}} & \multicolumn{1}{l}{\redText{6}}  \\ \hline
       \redText{1}  & 2 & 3 & 4 & 5 & 6 & 7  \\ \hline 
       \redText{2}  & 3 & 4 & 5 & 6 & 7 & 8  \\ \hline 
       \redText{3}  & 4 & 5 & 6 & 7 & 8 & 9  \\ \hline 
       \redText{4}  & 5 & 6 & 7 & 8 & 9 & 10 \\ \hline 
       \redText{5}  & 6 & 7 & 8 & 9 & 10& 11 \\ \hline 
       \redText{6}  & 7 & 8 & 9 & 10& 11& 12\\ \hline 
    \end{tabular}
  }
  
\end{frame}



\subsection{Infinite Series}

~

% \begin{frame}{Discrete Random Variables}

%   \begin{columns}
%     \column{.10\textwidth}
%     \begin{eqnarray*}
%       \begin{array}{l|l}
%         \mathrm{X} & p \\ \hline
%         x_1 & p_1 \\
%         x_2 & p_2 \\
%         x_3 & p_3 \\
%         \vdots & \vdots \\
%         x_n & p_n
%       \end{array}
%     \end{eqnarray*}
%     \vfill

%     \column{.90\textwidth}
%     \uncover<2->
%     {
%       \begin{definition}
%         The \textbf{mean} of a random variable is
%         \begin{eqnarray*}
%           \mu_X & = & x_1 p_1 + x_2 p_2 + \cdots + x_n p_n.
%         \end{eqnarray*}
%       \end{definition}
%     }

%     \uncover<3->
%     {
%       \begin{definition}
%         The \textbf{variation} of a random variable is
%         \begin{eqnarray*}
%           \sigma^2_X & = & (x_1-\mu_X)^2 p_1 + (x_2-\mu_X)^2 p_2 + \cdots + (x_n-\mu_X)^2 p_n.
%         \end{eqnarray*}
%       \end{definition}
%     }

%     \uncover<4->
%     {
%       \begin{definition}
%         The \textbf{standard deviation} of a random variable is the
%         square root of the variance.
%       \end{definition}
%     }

    
%   \end{columns}
  
% \end{frame}


% \subsection{Examples}

% \begin{frame}{Example}
%   \begin{columns}
%     \column{.15\textwidth}
%     \begin{eqnarray*}
%       \begin{array}{r|l}
%         \mathrm{X} & p \\ \hline
%         -2 & \frac{1}{6} \\ [5pt]
%          0 & \frac{3}{6} \\ [5pt]
%          2 & \frac{2}{6}
%       \end{array}
%     \end{eqnarray*}

%     \column{.90\textwidth}
%     \uncover<2->
%     {
%       \begin{eqnarray*}
%         \mu_X & = & -2 \cdot \frac{1}{6} + 0 \cdot \frac{3}{6} + 2 \cdot \frac{2}{6}, \\
%         & = & \frac{1}{3}.
%       \end{eqnarray*}
%     }

%   \end{columns}

%     \uncover<3->
%     {
%         \begin{eqnarray*}
%           \sigma^2_X & = & \lp -2-\frac{1}{3}\rp^2 \frac{1}{6} + 
%           \lp 0-\frac{1}{3}\rp^2 \frac{3}{6} + \lp 2-\frac{1}{3}\rp^2 \frac{2}{6}, \\
%           & = & \frac{17}{9}.
%         \end{eqnarray*}
%     }

%     \uncover<4->
%     {
%       \begin{eqnarray*}
%         \sigma_X & = & \sqrt{\frac{17}{9}}.
%       \end{eqnarray*}
%       \vfill
%     }

    

% \end{frame}



% \begin{frame}{Clicker Quiz}

%   What is the mean for the following probability distribution?
%     \begin{eqnarray*}
%       \begin{array}{r|l}
%         \mathrm{X} & p \\ \hline
%          0 & \frac{1}{8} \\ [5pt]
%          1 & \frac{2}{8} \\ [5pt]
%          2 & \frac{4}{8} \\ [5pt]
%          3 & \frac{1}{8}
%       \end{array}
%     \end{eqnarray*}

%     \vfill

%   \begin{tabular}{l@{\hspace{3em}}l@{\hspace{3em}}l}
%     A: 6/8  & B: 6/4 & C: 13/8
%   \end{tabular}

%   \vfill
%   \vfill
%   \vfill

% \end{frame}

% \begin{frame}{Example}
%   \begin{columns}
%     \column{.15\textwidth}
%     \begin{eqnarray*}
%       \begin{array}{r|l}
%         \mathrm{X} & p \\ \hline
%          0 & \frac{1}{8} \\ [5pt]
%          1 & \frac{1}{8} \\ [5pt]
%          2 & \frac{4}{8} \\ [5pt]
%          3 & \frac{2}{8}
%       \end{array}
%     \end{eqnarray*}

%     \column{.90\textwidth}
%     \uncover<2->
%     {
%       \begin{eqnarray*}
%         \mu_X & = & 0 \cdot \frac{1}{8} + 1 \cdot \frac{1}{8} + 2 \cdot \frac{4}{8} + 3 \frac{2}{8}, \\
%         & = & \frac{15}{8}.
%       \end{eqnarray*}
%     }

%   \end{columns}

%     \uncover<3->
%     {
%         \begin{eqnarray*}
%           \sigma^2_X & = & \lp 0-\frac{15}{8}\rp^2 \frac{1}{8} + 
%           \lp 1-\frac{15}{8}\rp^2 \frac{1}{8} + \lp 2-\frac{15}{8}\rp^2 \frac{4}{8} + 
%           \lp 3-\frac{15}{8}\rp^2 \frac{2}{8}, \\
%           & = & \frac{55}{64}.
%         \end{eqnarray*}
%     }

%     \uncover<4->
%     {
%       \begin{eqnarray*}
%         \sigma_X & = & \sqrt{\frac{55}{64}}.
%       \end{eqnarray*}
%       \vfill
%     }

    

% \end{frame}



% LocalWords:  Clarkson pausesection hideallsubsections sectionstyle
%  LocalWords:  hideothersubsections
