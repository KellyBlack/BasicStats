
\lecture{Variance}{variance}
\section{Variance of Random Variables}

\title{Variance of Random Variables}
\subtitle{Defining the Variance of a Random Variable}

%\author{Kelly Black}
%\institute{Clarkson University}
\date{10 September 2014}

\begin{frame}
  \titlepage
\end{frame}

\begin{frame}
  \frametitle{Outline}
  \tableofcontents[hideothersubsections,sectionstyle=show/hide]
\end{frame}


\subsection{Clicker Quiz}


\iftoggle{clicker}{%
  \begin{frame}
    \frametitle{Clicker Quiz}

    A random variable, $X$, is uniformly distributed between one and
    four. Determine the probability distribution function for the
    random variable.

    \vfill
    
    \begin{tabular}{l@{\hspace{3em}}l@{\hspace{3em}}l@{\hspace{3em}}l}
      A: $\frac{1}{4}$ & B: $\frac{1}{3}$ & C: $1$  & D: $\frac{5}{2}$
    \end{tabular}

    \vfill
    \vfill
    \vfill

  \end{frame}
}

\subsection{Discrete Examples}

\begin{frame}
  \frametitle{Discrete Example}


   \begin{columns}
     \column{.15\textwidth}
     \begin{eqnarray*}
       \begin{array}{r|l}
         \mathrm{X} & p \\ \hline
          0 & \frac{1}{16}  \\ [5pt]
          1 & \frac{2}{16}  \\ [5pt]
          2 & \frac{10}{16} \\ [5pt]
          3 & \frac{2}{16} \\ [5pt]
          4 & \frac{1}{16}
       \end{array}
     \end{eqnarray*}

     \column{.90\textwidth}
     \uncover<2->
     {
       \begin{eqnarray*}
         \mu_X & = & 0 \cdot \frac{1}{16} + 1 \cdot \frac{2}{16} + 2 \cdot \frac{10}{16} + 3 \cdot \frac{2}{16} + 4 \cdot \frac{1}{16}, \\
         & = & 2.
       \end{eqnarray*}
     }

   \end{columns}

  
\end{frame}


\begin{frame}
  \frametitle{Discrete Example}


   \begin{columns}
     \column{.15\textwidth}
     \begin{eqnarray*}
       \begin{array}{r|l}
         \mathrm{X} & p \\ \hline
          0 & \frac{1}{5} \\ [5pt]
          1 & \frac{1}{5} \\ [5pt]
          2 & \frac{1}{5} \\ [5pt]
          3 & \frac{1}{5} \\ [5pt]
          4 & \frac{1}{5}
       \end{array}
     \end{eqnarray*}

     \column{.90\textwidth}
     \uncover<2->
     {
       \begin{eqnarray*}
         \mu_X & = & 0 \cdot \frac{1}{5} + 1 \cdot \frac{1}{5} + 2 \cdot \frac{1}{5} + 3 \cdot \frac{1}{5} + 4 \cdot \frac{1}{5}, \\
         & = & 2.
       \end{eqnarray*}
     }

   \end{columns}

  
\end{frame}

\subsection{Variance}


\begin{frame}
  \frametitle{Variance}
  Spread away from what?
  \uncover<2->{
    \begin{definition}
      The \redText{variance} of a random variable is defined to be
      \begin{eqnarray*}
        \sigma^2 & = & E\left[\lp X - E[X] \rp^2\right].
      \end{eqnarray*}
    \end{definition}
  }
  \uncover<3->{
    \begin{definition}
      The \redText{standard deviation} of a random variable is defined to be
      \begin{eqnarray*}
        \sigma   & = & \sqrt{\sigma^2}, \\
                 & = & \sqrt{E\left[\lp X - E[X] \rp^2\right]}.
      \end{eqnarray*}
    \end{definition}
  }

  \uncover<4->{See page 103 for another way to calculate the variance.}

\end{frame}

\subsection{Discrete Examples}

\begin{frame}
  \frametitle{Discrete Example}


   \begin{columns}
     \column{.15\textwidth}
     \begin{eqnarray*}
       \begin{array}{r|l|l}
         \mathrm{X} & p & \uncover<2->{\mathrm{X}^2} \\ \hline
          0 & \frac{1}{16} & \uncover<2->{0}  \\ [5pt]
          1 & \frac{2}{16} & \uncover<2->{1} \\ [5pt]
          2 & \frac{10}{16}& \uncover<2->{4} \\ [5pt]
          3 & \frac{2}{16} & \uncover<2->{9} \\ [5pt]
          4 & \frac{1}{16} & \uncover<2->{16}
       \end{array}
     \end{eqnarray*}

     \column{.90\textwidth}
       \begin{eqnarray*}
         \mu_X & = & 2.
       \end{eqnarray*}

     \uncover<3->
     {
       \begin{eqnarray*}
         E\left[\mathrm{X}^2\right] & = & 0 \cdot \frac{1}{16} + 1 \cdot \frac{2}{16} + 4 \cdot \frac{10}{16} + 9 \cdot \frac{2}{16} + 16 \cdot \frac{1}{16}, \\
         & = & \frac{19}{4}, \\
         \sigma^2_X & = & \frac{3}{4}.
       \end{eqnarray*}
     }

   \end{columns}

  
\end{frame}


\begin{frame}
  \frametitle{Discrete Example}


   \begin{columns}
     \column{.15\textwidth}
     \begin{eqnarray*}
       \begin{array}{r|l|l}
         \mathrm{X} & p & \mathrm{X}^2\\ \hline
          0 & \frac{1}{5} & 0 \\ [5pt]
          1 & \frac{1}{5} & 1 \\ [5pt]
          2 & \frac{1}{5} & 4 \\ [5pt]
          3 & \frac{1}{5} & 9 \\ [5pt]
          4 & \frac{1}{5} & 16
       \end{array}
     \end{eqnarray*}

     \column{.90\textwidth}
       \begin{eqnarray*}
         \mu_X & = & 2.
      \end{eqnarray*}

     \uncover<2->
     {
       \begin{eqnarray*}
         E\left[\mathrm{X}^2\right] & = & 0 \cdot \frac{1}{5} + 1 \cdot \frac{1}{5} + 4 \cdot \frac{1}{5} + 9 \cdot \frac{1}{5} + 16 \cdot \frac{1}{5}, \\
         & = & 6, \\
         \sigma^2 & = & 2.
       \end{eqnarray*}
     }

   \end{columns}

  
\end{frame}



\subsection{Continuous Examples}

\begin{frame}
  \frametitle{Example}

  A random variable is uniformly distributed between one and
  four. Calculate its standard deviation.
  
\end{frame}


\subsection{Quartiles}

\begin{frame}
  \frametitle{Quartiles Example}

  A random variable is uniformly distributed between one and
  four. Calculate the quartiles and the median.
  
\end{frame}

\begin{frame}
  \frametitle{Quartiles Example}

  A random variable has the following probability density function:
      \begin{eqnarray*}
      f(x) & = & \left\{
        \begin{array}{l@{\hspace{2em}}l}
          \frac{x}{8}, & 0\leq x \leq 4, \\
          0 & \mathrm{otherwise.}
        \end{array}
      \right.
    \end{eqnarray*}
  Calculate the quartiles and the median.
  
\end{frame}


\begin{frame}
  \frametitle{Chebychev Inequality Example}

  A manufacturing process yields springs whose spring constants have a
  mean of 1.5 N/m and a standard deviation of 0.2 N/m. Provide an
  estimate of the percentage of springs with constants between 1.1 N/m
  and 1.9 N/m.
  
\end{frame}


%%% Local Variables: 
%%% mode: latex
%%% TeX-master: "IntroStats"
%%% End: 

%  LocalWords:  Quartiles quartiles Chebychev
