
\lecture{Point Estimates II}{point-estimates-two}
\section{Point Estimates II}


\title{Point Estimates II}
\subtitle{Comparing Different Estimators}

%\author{Kelly Black}
%\institute{Clarkson University}
\date{23 October 2013}

\begin{frame}
  \titlepage
\end{frame}

\begin{frame}
  \frametitle{Outline}
  \tableofcontents[hideothersubsections,sectionstyle=show/hide]
\end{frame}


\iftoggle{clicker}{%
  \subsection{Clicker Quiz}


  \begin{frame}
    \frametitle{Clicker Quiz}

    \vfill
    Find the median of the following data set:\\
    \begin{tabular}{llllll}
      28, & 17, & 12, & 20, & 21, & 18
    \end{tabular}

    \vfill
    
    \begin{tabular}{l@{\hspace{3em}}l@{\hspace{3em}}ll@{\hspace{3em}}l}
      A: 17 & B: 18 & C: 19 & D: 20
    \end{tabular}

    \vfill

  \end{frame}

}



\subsection{Sample Mean}

\begin{frame}{Sample Mean}

  We run an experiment and get data: \\
    \begin{tabular}{l|l}
      Number  & Value \\ \hline
      1 & $x_1$ \\
      2 & $x_2$ \\
      3 & $x_3$ \\
      $\vdots$ & $\vdots$ \\
      $n-1$ & $x_{n-1}$ \\
      $n$ & $x_n$
    \end{tabular}

  \only<2->%
  {
    We calculate a sample mean:
    \begin{eqnarray*}
      \bar{x} & = & \frac{x_1+x_2+\cdots+x_n}{n}.
    \end{eqnarray*}
  }

  \only<3->%
  {
    Problem: If $x_1$ is a random variable then so is $\bar{x}$!
  }

  
\end{frame}


\begin{frame}{Bias}

  $X$ is a random variable with some \blueText{parameter}, $\theta$,
  associated with it. We have a \blueText{method} to estimate the
  value, $\hat{\theta}$. 

  \begin{definition}[Unbiased Estimate]
    If $E[\hat{\theta}]=\theta$ we say that $\hat{\theta}$ is an
    \blueText{unbiased} estimator.
  \end{definition}

  \begin{definition}[Biased Estimate]
    If $E[\hat{\theta}]\neq\theta$ we say that $\hat{\theta}$ is a
    \blueText{biased} estimator.
  \end{definition}
  
\end{frame}




\begin{frame}{Bias}

  \begin{definition}[bias]
    The \blueText{bias} for an estimator is defined to be
    \begin{eqnarray*}
      \mathrm{bias} & = & E[\hat{\theta}] - \theta.
    \end{eqnarray*}

  \end{definition}
  
\end{frame}


\begin{frame}{Example}

  The following estimate for the variance is biased:
  \begin{eqnarray*}
    \frac{(x_1-\bar{x})^2 + (x_2-\bar{x})^2 + \cdots + (x_n - \bar{x})^2}{n}.
  \end{eqnarray*}

  See p. 304 of the book:
  \begin{eqnarray*}
    E\left[\frac{(x_1-\bar{x})^2 + (x_2-\bar{x})^2 + \cdots + (x_n -\bar{x})^2}{n}\right] & = & 
    \sigma^2 \frac{n-1}{n}.
  \end{eqnarray*}
  
\end{frame}


\iftoggle{clicker}{%
  \subsection{Clicker Quiz}


  \begin{frame}
    \frametitle{Clicker Quiz}

    \vfill

    $X$ is a Bernoulli random variable with parameter $p$. We take $N$
    trials and count the number of successes. The count follows what
    distribution?

    \vfill
    
    \begin{tabular}{l@{\hspace{3em}}l@{\hspace{3em}}ll@{\hspace{3em}}l}
      A: Bernoulli & B: Binomial & C: Poisson
    \end{tabular}

    \vfill

  \end{frame}

}


\begin{frame}{Example}

 $X$ is a Bernoulli random variable with parameter $p$. We take $N$
 trials and count the number of successes. We then define
 \begin{eqnarray*}
   \hat{p} & = & \frac{\mathrm{count}}{N}.
 \end{eqnarray*}
 
\end{frame}





\begin{frame}{Relative Efficiency}

  \begin{definition}
    The \blueText{relative efficiency} between two estimates,
    $\hat{\theta}_1$ and $\hat{\theta}_2$ is
    \begin{eqnarray*}
      \frac{\mathrm{var}\left( \hat{\theta}_2\right) }{\mathrm{var}\left( \hat{\theta}_1\right) }
    \end{eqnarray*}
  \end{definition}
  
\end{frame}


\begin{frame}{Example}

  $X$ is normally distributed with mean $\mu$ and standard deviation
  $\sigma$. We run two experiments. In the first experiment we take 20
  observations and find a sample mean. In the second experiment we
  take 40 observations and find a sample mean. What is the relative
  efficiency between the two estimates of the mean?
  
\end{frame}


\begin{frame}{Example}

  $X$ is a Bernoulli random variable with parameter $p$. We perform an
  experiment with $N$ trials and let
  \begin{eqnarray*}
    \hat{p} & = & \frac{\mathrm{count}}{N}.
  \end{eqnarray*}

  What is the MSE?
  
\end{frame}

