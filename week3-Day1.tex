%\part{}
\lecture{Introduction to Probability}{intro-to-probability}


\title{Probability}
\subtitle{What are the chances?}

\author{Kelly Black}
\institute{Clarkson University}
\date{27 January 2012}

\begin{frame}
  \titlepage
\end{frame}

\begin{frame}
  \frametitle{Outline}
  \tableofcontents[pausesection,hideallsubsections]
\end{frame}


\section{Clicker Quiz}


\begin{frame}
  \frametitle{Clicker Quiz}

  If I flip a fair coin ten times how many tails will I get?

  \begin{tabular}{l@{\hspace{3em}}l@{\hspace{3em}}l@{\hspace{3em}}l}
    A: 4 & B: 5 & C: 6 & D: I do not know
  \end{tabular}


\end{frame}




\section{Clicker Exercise}

\begin{frame}
  \frametitle{Clicker Exercise}

  Flip a coin 3 times. How many tails did you get?

  \begin{tabular}{l@{\hspace{3em}}l@{\hspace{3em}}l@{\hspace{3em}}l}
  A: 0 & B: 1 & C: 2 & D: 3
  \end{tabular}
  

\end{frame}

\section{Events}

\begin{frame}
  \frametitle{Events}

  \begin{definition}
    An event is a possible outcome from an experiment.
  \end{definition}

\end{frame}


\begin{frame}
  \frametitle{Sample Space}

  \begin{definition}
    The sample space is the collection of all possible events.
  \end{definition}

\end{frame}


\begin{frame}
  \frametitle{Sample Space}

  There are two ways to visualize the sample space:
  \begin{itemize}
  \item Venn Diagram
  \item Tree Diagram
  \end{itemize}

\end{frame}


\begin{frame}{Clicker Quiz}
  A couple has a child. Two years later they have another child. What
  is the probability that they have one girl and one boy?

  \begin{tabular}{l@{\hspace{3em}}l@{\hspace{3em}}l@{\hspace{3em}}l}
    A: 0 & B: 1/4 & C: 1/2 & D: 1
  \end{tabular}


\end{frame}

\section{Properties of Probabilities}

\begin{frame}{Properties of Probabilities}

  Suppose that A is an event in the sample space, then
  \begin{eqnarray*}
    \begin{array}{rcccl}
      0 & \leq & P(A) & \leq & 1
    \end{array}
  \end{eqnarray*}
  
\end{frame}

% LocalWords:  Clarkson pausesection hideallsubsections
