
\lecture{Sample Proportions}{sample-proportions}
\section{Sample Proportions}

\title{Sample Proportions}
\subtitle{Sampling a Binomial Distribution}

%\author{Kelly Black}
%\institute{Clarkson University}
\date{1 March 2013}

\begin{frame}
  \titlepage
\end{frame}

\begin{frame}
  \frametitle{Outline}
  \tableofcontents[hideothersubsections,sectionstyle=show/hide]
\end{frame}



\subsection{Clicker Quiz}


\iftoggle{clicker}{%
  \begin{frame}{Clicker Quiz}

    \iftoggle{clicker}{%

      It is estimated that 51\% of the people in a congressional
      district support the incumbent.  A poll is conducted in which 1000
      people are called and asked if they support the incumbent. The
      total number of people who say ``yes'' is counted. What is the
      distribution that the total will follow?

      \vfill 


      \begin{tabular}{l@{\hspace{3em}}l@{\hspace{3em}}l@{\hspace{3em}}l}
        A: Binomial  & B: Normal & C: Poisson
      \end{tabular}

      \vfill
      \vfill
      \vfill

    }

  \end{frame}
}





\subsection{Sample Proportions}

\begin{frame}
  \frametitle{Clicker Example}

  Will Kanye West's love child be a boy or a girl?

  \begin{tabular}{l@{\hspace{3em}}l}
    A: Boy    & B: Girl
  \end{tabular}
  
  

  \vfill

  \only<2->{ 

    Left section: Will Kanye West's love child be a boy or a
    girl?

    \begin{tabular}{l@{\hspace{3em}}l@{\hspace{3em}}l}
      A: Boy   & B: Girl & C: Not in left section
    \end{tabular}

  }

  \vfill

  \only<3->{ 

    Center section: Will Kanye West's love child be a boy or a
    girl?

    \begin{tabular}{l@{\hspace{3em}}l@{\hspace{3em}}l}
      A: Boy   & B: Girl& C: Not in center section
    \end{tabular}

  }

  \vfill

  \only<4->{ 

    Right section: Will Kanye West's love child be a boy or a
    girl?

    \begin{tabular}{l@{\hspace{3em}}l@{\hspace{3em}}l}
      A: Boy    & B: Girl & C: Not in right section
    \end{tabular}

  }

  \vfill


\end{frame}


\begin{frame}
  \frametitle{Clicker Example}

  \begin{itemize}
  \item We just made samples. (This is a not a random sample, but you
    get the idea.)

  \item Each time you take a sample you \textit{could} get something
    different.

  \item We make a calculation based on the sample. The calculation is
    called the ``sample proportion.''

  \end{itemize}

  \vfill


\end{frame}

\begin{frame}{Sampling a Binomial Distribution}

  Recall that if $X$ be a binomial distribution with a probability $p$
  of success and $n$ trials.
  \begin{itemize}
  \item Check if $n\cdot p \geq 5$ 
  \item Check if $n\cdot (1-p) \geq 5$ 
  \item If \textbf{both} conditions are true then $X$ is approximately
    normally distributed with mean $n\cdot p$ and standard deviation
    $\sqrt{p(1-p)n}$.
  \end{itemize}

  
\end{frame}


\begin{frame}
  \frametitle{Not What We Have!}

  We are asked about the \textit{proportion} not the total number who
  say ``yes.''

  \vfill

  \begin{definition}[The sample proportion]

    We take a sample from $n$ trials. Each trial has probability $p$
    of ``success.'' We count the total number of successes from the
    $n$ trials.  The sample proportion is defined to be
    \begin{eqnarray*}
      \hat{p} & = & \frac{\mathrm{number~of~successes}}{n}.
    \end{eqnarray*}
    
  \end{definition}



\end{frame}

\begin{frame}
  \frametitle{The sample proportion}

  The sample proportion has a mean of $p$, and a standard deviation of
  $\sqrt{\frac{p(1-p)}{n}}$.

  \vfill

  \only<2->
  {

    If $p\cdot n \geq 5$ \textbf{and} $(1-p)\cdot n \geq 5$
    \textbf{and} $n\geq 20$ then $\hat{p}$ is approximately normally
    distributed with mean $p$ and standard deviation
    $\sqrt{\frac{p(1-p)}{n}}$.

  }

  \vfill

  \only<3->
  {
    If this is the case we have
    \begin{eqnarray*}
      z & = & \frac{\hat{p}-p}{\sqrt{\frac{p(1-p)}{n}}}.
    \end{eqnarray*}
  }

  \vfill

  \only<4->
  {

    Problem: We usually do not know $p$. That is okay just use
    $\hat{p}$ or assume $p=\frac{1}{2}$ for the ``worst case
    scenario.''

  }


\end{frame}


\subsection{Examples}

\begin{frame}
  \frametitle{Example}

  Roughly 51\% of the people in a district support the
  incumbent. One-thousand people are polled. What is the probability
  that less than half of the people will say ``yes?''

  \vfill


\end{frame}

\begin{frame}{Example}

  It is estimated that 64\% of the people who get a flu shot this year
  will not get the flu. Four hundred people who have had the shot are
  polled. What is the probability that more than 35\% of them will get
  the flu?

    \vfill

\end{frame}



\iftoggle{clicker}{%
  \begin{frame}{Clicker Quiz}


    A previous study indicates that 34\% of people who use a cell
    phone regularly will get parotid gland cancer. Two hundred people
    are tracked who are regular cell phone users. What is the
    probability that fewer than 30\% of the people will get parotid
    gland cancer?

    \vfill 


    \begin{tabular}{l@{\hspace{3em}}l@{\hspace{3em}}l@{\hspace{3em}}l}
      A: 0.10  & B: 0.11 & C: 0.12 & D: 0.13
    \end{tabular}

    \vfill
    \vfill
    \vfill



\end{frame}

}



\begin{frame}{In a nutshell}


  You conduct $n$ trials. Each trial has a probability $p$ for
  ``success'' and a probability of $1-p$ for a different
  result. Define the sample proportion to be 
  \begin{eqnarray*}
    \hat{p} & = & \frac{\mathrm{number~of~successes}}{\mathrm{number~of~trials}}.
  \end{eqnarray*}

  \begin{itemize}
  \item The sample proportion has a mean of $p$, and a standard
    deviation of $\sqrt{\frac{p(1-p)}{n}}$.
  \item Check to see if $p\cdot n \geq 5$, $(1-p)\cdot n \geq 5$ and
    $n\geq 20$. If \textit{all} of these things are true
    \begin{itemize}
    \item $\hat{p}$ is approximately normally distributed
    \item The mean of $\hat{p}$ is mean $p$.
    \item The standard deviation of $\hat{p}$ is
      $\sqrt{\frac{p(1-p)}{n}}$
    \item A $z$-statistic can be approximated using
      \begin{eqnarray*}
        z & = & \frac{\hat{p}-p}{\sqrt{\frac{p(1-p)}{n}}}.
      \end{eqnarray*}
    \end{itemize}

  \end{itemize}



  
\end{frame}

\begin{frame}{Example}

  I want to conduct a poll. It is estimated that each time I ask
  someone the question there is a probability of 0.35 that the person
  will say ``yes.'' How many people should I poll so that there is a
  probability of 0.05 that the sample proportion will be less than
  0.30?

  \vfill


\end{frame}

\begin{frame}{Example}

  I want to conduct a poll, and each question will be answered with
  either a ``yes'' or a ``no.''  How many people should I poll so that
  there is a probability of 0.05 that the sample proportion will be
  less than 0.1 lower than the true value?

  \vfill


\end{frame}



% LocalWords:  Clarkson pausesection hideallsubsections hideothersubsections
% LocalWords:  sectionstyle Kanye parotid
