
\lecture{Independent Populations}{independent-populations}
\section{Independent Populations}

\title{Independent Populations}
\subtitle{Comparing Two Populations}

%\author{Kelly Black}
%\institute{Clarkson University}
\date{11 April 2012}

\begin{frame}
  \titlepage
\end{frame}

\begin{frame}
  \frametitle{Outline}
  \tableofcontents[pausesection,hideothersubsections,sectionstyle=show/hide]
\end{frame}


\subsection{Clicker Quiz}

\begin{frame}{Clicker Quiz}

  \iftoggle{clicker}{%

    A small university on a remote northern border requires that
    incoming students who enroll in the calculus course take a
    placement test before the start of classes. The data is to be
    examined to see if the placement practices are working as
    intended. The student's scores on the pretest will be compared to
    the student's average in the class at the end of the semester. Are
    these independent or dependent populations?

    \vfill

    \begin{tabular}{l@{\hspace{3em}}l@{\hspace{3em}}l@{\hspace{3em}}l}
      A: Independent & 
      B: Dependent
    \end{tabular}

    \vfill
    \vfill
    \vfill

  }

\end{frame}

\begin{frame}{Clicker Quiz}

  \iftoggle{clicker}{%

    A small university on a remote northern border requires that
    incoming students who enroll in the calculus course take a
    placement test before the start of classes. The data is to be
    examined to see if the placement practices are working as
    intended. The student's who come from urban areas will be compared
    to students who come from rural areas. Are these independent or
    dependent populations?

    \vfill

    \begin{tabular}{l@{\hspace{3em}}l@{\hspace{3em}}l@{\hspace{3em}}l}
      A: Independent & 
      B: Dependent
    \end{tabular}

    \vfill
    \vfill
    \vfill

  }

\end{frame}


\subsection{Independent Populations}

\begin{frame}
 \frametitle{Two Independent Samples}

 We have two random variables:

 \begin{columns}
    \column{.5\textwidth}
    \begin{eqnarray*}
      \bar{x}
    \end{eqnarray*}
    \only<2->
    {
      This random variable has the following properties:
      \begin{tabular}{ll}
        Mean & $\mu_1$, \\
        Samples & $N_1$, \\
        Std. Dev. & $\frac{\sigma_1}{\sqrt{N_1}}$, \\
      \end{tabular}
    }
    \vfill

    \column{.5\textwidth}
    \begin{eqnarray*}
      \bar{y}
    \end{eqnarray*}
    \only<3->
    {
      This random variable has the following properties:
      \begin{tabular}{ll}
        Mean & $\mu_2$, \\
        Samples & $N_2$., \\
        Std. Dev. & $\frac{\sigma_2}{\sqrt{N_2}}$, \\
      \end{tabular}
    }

    \vfill

 \end{columns}

\end{frame}

\begin{frame}{Take The Difference}

  Define a new random variable:
  \begin{eqnarray*}
    W & = & \bar{x} - \bar{y}.
  \end{eqnarray*}

  The mean of this random variable is 
  \begin{eqnarray*}
    \mathrm{Mean~of~}W & = & \mu_1 - \mu_2.
  \end{eqnarray*}

  \textbf{IF} the populations are independent then the variation of
  this random variable is
  \begin{eqnarray*}
    \mathrm{Variance~of~}W & = & \frac{\sigma_1^2}{N_1} + \frac{\sigma_2^2}{N_2}.
  \end{eqnarray*}

  So the standard deviation is
  \begin{eqnarray*}
    \mathrm{standard~deviation~of~}W & = & \sqrt{\frac{\sigma_1^2}{N_1} + \frac{\sigma_2^2}{N_2}}.
  \end{eqnarray*}

  
\end{frame}


\begin{frame}
 \frametitle{If We Have Sample Standard Deviation}

 We have two random variables:

 \begin{columns}
    \column{.5\textwidth}
    \begin{eqnarray*}
      \bar{x}
    \end{eqnarray*}
    {
      \begin{tabular}{ll}
        Mean & $\mu_1$, \\
        Samples & $N_1$, \\
        Sample Std. Dev. & $\frac{s_1}{\sqrt{N_1}}$, \\
      \end{tabular}
    }
    \vfill

    \column{.5\textwidth}
    \begin{eqnarray*}
      \bar{y}
    \end{eqnarray*}
    {
      \begin{tabular}{ll}
        Mean & $\mu_2$, \\
        Samples & $N_2$., \\
        Sample Std. Dev. & $\frac{s_2}{\sqrt{N_2}}$, \\
      \end{tabular}
    }

 \end{columns}

\vfill

This defines a $t$-distribution
\begin{eqnarray*}
  t & = & \frac{(\bar{x}-\bar{y})-(\mu_1-\mu_2)}{\sqrt{\frac{s_1^2}{N_1} + \frac{s_2^2}{N_2}}}.
\end{eqnarray*}
The numbers of degrees of freedom is the smaller value of $N_1-1$ and $N_2-1$.
 

\end{frame}

\begin{frame}{Example}

  I run an experiment, get a mean, standard deviation, and some number
  of trials. I then do a hypothesis test:
  \begin{eqnarray*}
    \begin{array}{l@{\hspace{2em}}rcl}
      H_0 & \mu & = & 0, \\
      H_a & \mu & > & 0.
    \end{array}
  \end{eqnarray*}

  \only<2->
  {
    Suppose I get a value for $t$ and look up the critical $t$ value
    ($t^*$):
    \begin{eqnarray*}
      t   & \approx & 2.25, \\
      t^* & \approx & 2.28.
    \end{eqnarray*}

    Can I reject $H_0$?

  }
  
\end{frame}

\begin{frame}{Example}

  I compare two factories. The mean number of items manufactured per
  day is compared. A sample of 24 days from the first factory produced
  a sample mean of 145 items per day with a sample standard deviation
  of 15 items per day. A sample of 28 days from the second factory
  produced a sample mean of 152 items per day with a sample standard
  deviation of 17 items per day. Find the 95\% confidence interval for
  the difference.
  
\end{frame}

\begin{frame}{Example}

  I compare two factories. The mean number of items manufactured per
  day is compared, and we suspect that the first factory is
  under-performing. A sample of 24 days from the first factory produced
  a sample mean of 145 items per day with a sample standard deviation
  of 15 items per day. A sample of 28 days from the second factory
  produced a sample mean of 152 items per day with a sample standard
  deviation of 17 items per day. Can we confirm our suspicions?
  
\end{frame}


\begin{frame}{Clicker Quiz}

  \iftoggle{clicker}{%

    The yield per acre of two types of seeds are examined. The first
    seed type has fifty-one plots that yield a sample mean of 42
    bushels per acre with a sample standard deviation of 2.6 bushels
    per acre. The second seed type has fifty-five plots that yield a
    sample mean of 43 bushels per acre with a sample standard
    deviation of 3.1 bushels per acre. Find the 95\% confidence
    interval for the difference.

    \vfill 


    \begin{tabular}{l@{\hspace{3em}}l@{\hspace{3em}}l@{\hspace{3em}}l}
      A: -2.57 to .57, & 
      B: -2.11 to .11 \\
      C: -1.93 to -.07 &
      D: -1.92 to -0.08
    \end{tabular}

    \vfill
    \vfill
    \vfill

  }

\end{frame}

\begin{frame}{Clicker Quiz}

  \iftoggle{clicker}{%

    The yield per acre of two types of seeds are examined. The first
    seed type has fifty-one plots that yield a sample mean of 42
    bushels per acre with a sample standard deviation of 2.6 bushels
    per acre. The second seed type has fifty-five plots that yield a
    sample mean of 43 bushels per acre with a sample standard
    deviation of 3.1 bushels per acre. Is there a difference? (Use a
    95\% confidence level.)

    \vfill 


    \begin{tabular}{l@{\hspace{3em}}l@{\hspace{3em}}l@{\hspace{3em}}l}
      A: Reject $H_o$ & Do not reject $H_o$
    \end{tabular}

    \vfill
    \vfill
    \vfill

  }

\end{frame}






% LocalWords:  Clarkson pausesection hideallsubsections
