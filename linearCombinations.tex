
\lecture{Linear Combinations of Random Variables}{linear-combinations}
\section{Linear Combinations of Random Variables}

\title{Linear Combinations of Random Variables}
\subtitle{What is the Difference?}

%\author{Kelly Black}
%\institute{Clarkson University}
\date{12 September 2014}

\begin{frame}
  \titlepage
\end{frame}

\begin{frame}
  \frametitle{Outline}
  \tableofcontents[hideothersubsections,sectionstyle=show/hide]
\end{frame}


\subsection{Clicker Quiz}


\iftoggle{clicker}{%
  \begin{frame}
    \frametitle{Clicker Quiz}

    A random variable, $X$, is uniformly distributed between one and
    six. Determine the expected value of $X$.

    \vfill
    
    \begin{tabular}{l@{\hspace{3em}}l@{\hspace{3em}}l@{\hspace{3em}}l}
      A: $\frac{1}{6}$ & B: $2$ & C: $\frac{7}{2}$  & D: $3$
    \end{tabular}

    \vfill
    \vfill
    \vfill

  \end{frame}
}

\subsection{Linear Transformation}

\begin{frame}
  \frametitle{Example}
  Suppose that we perform an experiment. We measure the temperature of
  a cooling tube in Fahrenheit. 

  \vfill
  
  Later we realize that it should have been in Celsius. 

  \vfill

  What do we do?

\end{frame}

\begin{frame}{Linear Transformation}

  In general we have a Random Variable, $X$.

  \vfill

  We define a new random variable,
  \begin{eqnarray*}
    Y & = & a X + b.
  \end{eqnarray*}

  \vfill

  Question, what is the expected value and variance of this new random
  variable?

  
\end{frame}


\begin{frame}
  \frametitle{Expectation and Variance of a Linear Transformation}

  \begin{block}{Expectation of a Linear Transformation}
    If $X$ is a random variable, and $a$ and $b$ are constants then
    the expected value of $Y=aX+b$ is 
    \begin{eqnarray*}
      E[Y] & = & a E[X] + b.
    \end{eqnarray*}
  \end{block}

  \begin{block}{Variation of a Linear Transformation}
    If $X$ is a random variable, and $a$ and $b$ are constants then
    the variation of $Y=aX+b$ is 
    \begin{eqnarray*}
      Var[Y] & = & a^2 Var[X].
    \end{eqnarray*}
  \end{block}

  
\end{frame}

\subsection{Linear Combinations}

\begin{frame}
  \frametitle{Linear Combinations of Random Variables}

  A linear combination of the random variables $X_1$, $X_2$, $\ldots$,
  $X_n$ is given by
  \begin{eqnarray*}
    Y & = & a_1 X_1 + a_2 X_2 + \cdots + a_n X_n,
  \end{eqnarray*}
  where $a_1$, $a_2$, $\ldots$, and $a_n$ are constants.
  
\end{frame}

\begin{frame}
  \frametitle{Independence of Two Random Variables}

    If $X_1$ and $X_2$ are random variables then if they are independent
    then their covariance is zero,
    \begin{eqnarray*}
      \mathrm{Cov}(X_1,X_2) & = & 0.
    \end{eqnarray*}
  
\end{frame}

\begin{frame}
  \frametitle{Expectation and Variance of a Linear Combination}

  A linear combination of the random variables $X_1$, $X_2$, $\ldots$,
  $X_n$ is given by
  \begin{eqnarray*}
    Y & = & a_1 X_1 + a_2 X_2 + \cdots + a_n X_n,
  \end{eqnarray*}
  where $a_1$, $a_2$, $\ldots$, and $a_n$ are constants.


  \begin{block}{Expectation of a Linear Combination}
    The expected value of $Y$ is 
    \begin{eqnarray*}
      E[Y] & = & a_1 E[X_1] + a_2 E[X_2] + \cdots + a_n E[X_n].
    \end{eqnarray*}
  \end{block}

  \begin{block}{Variation of a Linear Combination}
    \redText{If the random variables are independent} then the
    variation of $Y$ is
    \begin{eqnarray*}
      \mathrm{Var}[Y] & = & a_1^2 \mathrm{Var}[X_1] + a_2^2 \mathrm{Var}[X_2] + \cdots + a_n^2 \mathrm{Var}[X_n].
    \end{eqnarray*}

  \end{block}

  
\end{frame}


\subsection{Examples}

\begin{frame}
  \frametitle{Example}
  A plant assembles motors. We pick an engine at random and measure
  the outside of a piston and the inside of the piston chamber. The
  outside radius of the piston has a mean of 5.50cm and a standard
  deviation of 0.02cm. The inside radius of the piston chamber has a
  mean radius of 5.54cm and a standard deviation of 0.01cm. What is
  the mean and standard deviation of the gap between the piston and
  the chamber?
\end{frame}

\iftoggle{clicker}{%
  \begin{frame}
    \frametitle{Clicker Quiz}

    A random variable, $X$, has a mean of 3 and a standard deviation
    of 5. A random variable $Y$ has a mean of -2 and a standard
    deviation of 4. What is the standard deviation of $2x-y$?

    \vfill
    
    \begin{tabular}{l@{\hspace{3em}}l@{\hspace{3em}}l@{\hspace{3em}}l}
      A: $6$ & B: $10.7$ & C: $14$  & D: $116$
    \end{tabular}

    \vfill
    \vfill
    \vfill

  \end{frame}
}



\begin{frame}
  \frametitle{Example}
  There are two books used in a statistics course. The mean test score
  for those using the first book is 93.0 with a standard deviation of
  12.0. The mean test score for those using the second book is 89.0
  with a standard deviation of 14.0. How do we compare the two groups?
\end{frame}

\begin{frame}
  \frametitle{Example}
  A random variable, $X$, has a probability density function given by
  \begin{eqnarray*}
    f(x) & = & \left\{
      \begin{array}{l@{\hspace{2em}}l}
        \frac{x}{8}, & 0\leq x \leq 4, \\
        0 & \mathrm{otherwise.}
      \end{array}
    \right.
  \end{eqnarray*}

  Suppose that $Y=3X+8$. What is the cumulative distribution function
  for $Y$?

\end{frame}


%%% Local Variables: 
%%% mode: latex
%%% TeX-master: "IntroStats"
%%% End: 
