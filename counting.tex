
\lecture{Counting}{counting}
\section{Counting}

\title{Counting}
\subtitle{Stuff We Thought We Knew}

%\author{Kelly Black}
%\institute{Clarkson University}
\date{20 January 2014}

\begin{frame}
  \titlepage
\end{frame}

\begin{frame}
  \frametitle{Outline}
  \tableofcontents[hideothersubsections,sectionstyle=show/hide]
\end{frame}


\iftoggle{clicker}{%
  \subsection{Clicker Quiz}


  \begin{frame}
    \frametitle{Clicker Quiz}

    If $p(A)=0.8$, $p(B|A)=0.4$ what is $p(A\mathrm{~and~}B)$?

    \vfill

    \begin{tabular}{l@{\hspace{3em}}l@{\hspace{3em}}l}
      A: .32 & B: 0.50 & C: 2.0
    \end{tabular}

    \vfill
    \vfill
    \vfill

  \end{frame}

}

\subsection{Multiplication Rule}

\begin{frame}
  \frametitle{How Many Ways?}
  Suppose that I flip a fair coin three times. How many possible
  outcomes are there?

  \vfill

  \only<2->{

    \begin{picture}(180,180)
      \put(0,90){\circle*{5}}
      \put(0,90){\line(5,3){50}}
      \put(0,90){\line(5,-3){50}}
      \put(52,120){H}
      \put(52,60){T}

      \only<3->{
        \put(60,130){\line(5,3){40}}
        \put(60,130){\line(5,-2){40}}
        \put(105,157.5){H}
        \put(105,112.5){T}
        \put(60,60){\line(5,2){40}}
        \put(60,60){\line(5,-3){40}}
        \put(105, 67.5){H}
        \put(105, 22.5){T}
      }

      \only<4->{
        \put(115,162.5){\line(5,1){40}}
        \put(115,162.5){\line(5,-1){40}}
        \put(157.5,168.75){H}
        \put(157.5,146.25){T}
        \put(115,117){\line(5,1){40}}
        \put(115,117){\line(5,-1){40}}
        \put(157.5,123.75){H}
        \put(157.5,101.25){T}
        \put(115,74){\line(5,1){40}}
        \put(115,74){\line(5,-1){40}}
        \put(157.5, 78.75){H}
        \put(157.5, 56.25){T}
        \put(115,26){\line(5,1){40}}
        \put(115,26){\line(5,-1){40}}
        \put(157.5, 33.75){H}
        \put(157.5, 11.25){T}
      }

    \end{picture}

  }

  \vfill

\end{frame}

\begin{frame}{Example}

  I have three pairs of pants and two shirts. How many different
  outfits do I have?

  \vfill

  \only<2->{

    \begin{picture}(180,180)
      \put(0,90){\circle*{5}}
      \put(0,90){\line(4,5){50}}
      \put(0,90){\line(4,0){50}}
      \put(0,90){\line(4,-5){50}}
      \put(52,150){$P_1$}
      \put(52, 90){$P_1$}
      \put(52, 30){$P_1$}

      \only<3->{
        \put( 65,155){\line(5,2){35}}
        \put( 65,155){\line(5,-2){35}}
        \put(105,165){$S_1$}
        \put(105,135){$S_2$}
        \put(65,95){\line(5,2){35}}
        \put(65,95){\line(5,-2){35}}
        \put(105,105){$S_1$}
        \put(105, 75){$S_2$}
        \put(65,35){\line(5,2){35}}
        \put(65,35){\line(5,-2){35}}
        \put(105, 45){$S_1$}
        \put(105, 15){$S_2$}
      }

    \end{picture}

  }

  \vfill
  
\end{frame}

\begin{frame}
  \frametitle{Multiplication Rule}

  \begin{definition}[The Multiplication Rule]
    If I have choices where I have
    \begin{itemize}
    \item $p_1$ choices for the first item
    \item $p_2$ choices for the second item
    \item $p_3$ choices for the third item \\
      $\vdots$
    \item $p_k$ choices for the $k$\textsuperscript{th} item \\
    \end{itemize}

    The total number of possible selections is 
    \begin{eqnarray*}
      p_1 \cdot p_2 \cdot p_3 \cdots ~ \cdot p_k.
    \end{eqnarray*}

  \end{definition}

\end{frame}

\subsection{Permutations}

\begin{frame}
  \frametitle{Example}
  A committee has eight people. There is a chair and a secretary for
  the committee. How many ways are there to choose a chair and a
  secretary?

  \vfill

  \textit{{\color{red}{order matters!}}}

\end{frame}

\begin{frame}
  \frametitle{Example}
  Twelve people are in a race. The first three people get an
  award. How many different ways can we hand out the awards?

  \vfill

  \textit{{\color{red}{order matters!}}}
\end{frame}


\begin{frame}
  \frametitle{Permutations}
  \begin{definition}[Permutations]
    The number of permutations is the number of ways to arrange $n$
    objects when $r$ are chosen. \textit{{\color{red}{(The order matters.)}}}
  \end{definition}

  \only<2->%
  {
    \begin{definition}[Notation]
      The number of permutations for $n$ objects when you choose $r$
      is 
      \begin{eqnarray*}
        \prescript{~}{n}{P}_r.
      \end{eqnarray*}
    \end{definition}
  }

\end{frame}

\begin{frame}
  \frametitle{Previous Examples}

  \begin{eqnarray*}
    \prescript{~}{8}{P}_2 & = & 8\cdot 7, \\
    \prescript{~}{12}{P}_3 & = & 12\cdot 11 \cdot 10. \\
  \end{eqnarray*}


  \only<2->%
  {

    \begin{definition}[General Formula for Permutations]

      In general
      \begin{eqnarray*}
        \prescript{~}{n}{P}_r & = & n(n-1)(n-2)\cdots(n-r+1).
      \end{eqnarray*}

      
    \end{definition}

  }
  
\end{frame}

\begin{frame}
  \frametitle{Notation}

  \begin{definition}[Factorials]

    If $n$ is an integer then
    \begin{eqnarray}
      n! & = & n(n-1)(n-2)\cdot 3 \cdot 2 \cdot 1.
    \end{eqnarray}
    
  \end{definition}

  Examples:
  \begin{eqnarray}
    4! & = & 4 \cdot 3  \cdot 2 \cdot 1, \\
    5! & = & 5 \cdot 4 \cdot 3  \cdot 2 \cdot 1.
  \end{eqnarray}
  
\end{frame}



\begin{frame}
  \frametitle{By the Way}

      \begin{eqnarray*}
        \prescript{~}{n}{P}_r & = & n(n-1)(n-2)\cdots(n-r+1), \\
        & = & \frac{n!}{(n-r)!}.
      \end{eqnarray*}

  
\end{frame}

\iftoggle{clicker}{%


  \begin{frame}
    \frametitle{Clicker Quiz}

    There are eight distinct marbles in a bag. I pull three out one
    after the other and place them side by side from left to
    right. How many different ways can this be done?

    \vfill

    \begin{tabular}{l@{\hspace{3em}}l@{\hspace{3em}}l@{\hspace{3em}}l}
      A: $8!$ & B: $\frac{8!}{3!}$  & C: $\frac{8!}{5!}$ & C: $3!$
    \end{tabular}

    \vfill
    \vfill
    \vfill

  \end{frame}

}



\subsection{Combinations}

\begin{frame}
  \frametitle{Example}

  What if I have three people on a team but only two can play at a
  time? How many different groups can play?

  \vfill

  \only<2->%
  {

    Notation:
    \begin{eqnarray*}
      \prescript{~}{3}C_2 & = & 3, \\
      3 \choose 2  & = & 3.
    \end{eqnarray*}

  }
  
\end{frame}


\begin{frame}
  \frametitle{Combinations}

  In general we have 
  \begin{eqnarray*}
    \underbrace{
      \rule{.3cm}{.2mm} \hspace{.05cm} 
      \rule{.3cm}{.2mm} \hspace{.05cm} 
      \rule{.3cm}{.2mm} \hspace{.05cm} 
      \rule{.3cm}{.2mm} \hspace{.05cm} 
      \rule{.3cm}{.2mm} \hspace{.05cm} \ldots
      \rule{.3cm}{.2mm} \hspace{.05cm} 
      \rule{.3cm}{.2mm} \hspace{.05cm}}_{\mathrm{n~possible~items}}
    & \rightarrow & 
    \underbrace{
      \rule{.3cm}{.2mm} \hspace{.05cm} 
      \rule{.3cm}{.2mm} \hspace{.05cm} 
      \rule{.3cm}{.2mm} \hspace{.05cm} \ldots
      \rule{.3cm}{.2mm} \hspace{.05cm} 
      \rule{.3cm}{.2mm} \hspace{.05cm}}_{\mathrm{k~slots}}
  \end{eqnarray*}

  \only<2->%
  {

    There are $\prescript{~}{n}{P}_r$ permutations. Each permutation
    can be expressed in $k!$ different ways,
    \begin{eqnarray*}
      \prescript{~}{n}{P}_r & = & k! \prescript{~}{n}{C}_r, \\
      \prescript{~}{n}{C}_r & = & \frac{\prescript{~}{n}{P}_r}{k!}, \\
      \prescript{~}{n}{C}_r & = & \frac{n!}{(n-k)!k!}.
    \end{eqnarray*}
  }
  
\end{frame}

\begin{frame}
  \frametitle{Example}
  Thirty people are on a committee. The executive committee is made up
  of four people. How many ways are there to form the executive
  committee?
  
\end{frame}

\iftoggle{clicker}{%


  \begin{frame}
    \frametitle{Clicker Quiz}

    There are eight distinct marbles in a bag. I pull three out one
    after the other and give them to a friend. How many different
    combinations can I give away?

    \vfill

    \begin{tabular}{l@{\hspace{3em}}l@{\hspace{3em}}l@{\hspace{3em}}l}
      A: $\frac{8!}{3!}$ & B: $\frac{8!}{5!3!}$  & C: $\frac{5!}{3!}$ & C: $3!$
    \end{tabular}

    \vfill
    \vfill
    \vfill

  \end{frame}

}


\subsection{Generalized Permutations}

\begin{frame}
  \frametitle{Generalized Permutations}

  I have
  \begin{itemize}
  \item A set of $n_1$ identical objects
  \item Another set of $n_2$ identical objects 
  \item Another set of $n_3$ identical objects \\
    \vdots
  \item Another set of $n_k$ identical objects, \\
  \end{itemize}

  \vfill

  \uncover<2->{
    Define the total number of object to be $N=n_1+n_2+n_3+\cdots+n_k$.
  }

  \vfill

  \uncover<3->{
    The number of ways to arrange these objects is 
    \begin{eqnarray*}
      \frac{N!}{n_1! n_2! n_3! \cdots n_k!}.
    \end{eqnarray*}
  }
  
\end{frame}

\begin{frame}
  \frametitle{Example}

  I have five blue flags and eight red flags. How many ways can I
  arrange them on a vertical flag pole?

  
  
\end{frame}

%%% Local Variables: 
%%% mode: latex
%%% TeX-master: "IntroStats"
%%% End: 

% LocalWords:  pausesection hideothersubsections sectionstyle
