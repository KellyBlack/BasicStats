
\lecture{Counting}{counting}
\section{Counting}

\title{Counting}
\subtitle{Stuff We Thought We Knew}

%\author{Kelly Black}
%\institute{Clarkson University}
\date{18 September 2013}

\begin{frame}
  \titlepage
\end{frame}

\begin{frame}
  \frametitle{Outline}
  \tableofcontents[hideothersubsections,sectionstyle=show/hide]
\end{frame}


\iftoggle{clicker}{%
  \subsection{Clicker Quiz}


  \begin{frame}
    \frametitle{Clicker Quiz}

    I have eight markers, and they are all different colors. I pull
    out the red one to set it aside because it is only for special
    occasions, and I decide not to use it today. If I do not use the
    red one what is the greatest number of colors that I can use
    today?

    \vfill

    \begin{tabular}{l@{\hspace{3em}}l@{\hspace{3em}}l@{\hspace{3em}}l}
      A: 7 & B: 8 & C: 42 & D: 56
    \end{tabular}

    \vfill
    \vfill
    \vfill

  \end{frame}

}

\subsection{Multiplication Rule}

\begin{frame}
  \frametitle{How Many Ways?}

  Suppose that I will run an experiment. I will make two
  repetitions. Each repetition can have three possible outcomes. How
  many possible events are possible?

  \vfill

\end{frame}

\begin{frame}
  \frametitle{Multiplication Rule}

  \begin{definition}[The Multiplication Rule]
    If I have choices where I have
    \begin{itemize}
    \item $n_1$ choices for the first item
    \item $n_2$ choices for the second item
    \item $n_3$ choices for the third item \\
      $\vdots$
    \item $n_k$ choices for the $k$\textsuperscript{th} item \\
    \end{itemize}

    The total number of possible selections is 
    \begin{eqnarray*}
      n_1 \cdot n_2 \cdot n_3 \cdots ~ \cdot n_k.
    \end{eqnarray*}

  \end{definition}

\end{frame}


\begin{frame}
  \frametitle{Example}
  There are three steps in a manufacturing process:
  \begin{itemize}
  \item Step one is done in one of four different stations.
  \item Step two is done in one of three different stations.
  \item Step three is done in one of five different stations.
  \end{itemize}

  How many different ways can the object be assembled?
\end{frame}

\begin{frame}
  \frametitle{Example}
  The operation of a solar panel involves five variables: the amount
  of silicon, the temperature, the pressure, electrical potential, and
  the amperage. We want to test the influence of the different
  variables at different levels:
  \begin{itemize}
  \item Six levels of silicon.
  \item Eight different temperatures.
  \item Four different pressures.
  \item Seven different electrical potentials.
  \item Three different amperage levels.
  \end{itemize}

  How many different experimental setups are required?
\end{frame}


\subsection{Permutations}

\begin{frame}
  \frametitle{Example}
  Suppose that we have forty glass jars. We will randomly pull them
  out of a bin without replacement. How many ways can we do this?
  (Assume that the order matters.)
  \vfill


\end{frame}

\begin{frame}
  \frametitle{Notation}

  \begin{definition}[Factorials]

    If $n$ is a positive integer then
    \begin{eqnarray}
      n! & = & n(n-1)(n-2)\cdots (3) \cdot (2) \cdot 1.
    \end{eqnarray}

    We define $0!$ to be equal to one.    
  \end{definition}

  Examples:
  \begin{eqnarray}
    4! & = & 4 \cdot 3  \cdot 2 \cdot 1, \\
    5! & = & 5 \cdot 4 \cdot 3  \cdot 2 \cdot 1.
  \end{eqnarray}
  
\end{frame}



\begin{frame}
  \frametitle{Permutations}

  \begin{definition}[Permutations]
    The number of permutations is the number of ways to arrange $n$
    objects when $k$ is chosen. \textit{(The order matters.)}
  \end{definition}

  \begin{definition}[Notation]
    The number of permutations for $n$ objects when you choose $k$
    is 
    \begin{eqnarray*}
      \prescript{~}{n}{P}_k & = & n(n-1)(n-2)\cdots(n-r+1), \\
                            & = & \frac{n!}{(n-k)!}.
    \end{eqnarray*}
  \end{definition}

\end{frame}

\begin{frame}
  \frametitle{Previous Examples}

  \begin{eqnarray*}
    \prescript{~}{40}{P}_{40} & = & \frac{40!}{0!}, \\
    \prescript{~}{40}{P}_2 & = & 40\cdot 39 ~ = ~ \frac{40!}{38!}. \\
  \end{eqnarray*}


\end{frame}


\iftoggle{clicker}{%


  \begin{frame}
    \frametitle{Clicker Quiz}

    There are eight distinct marbles in a bag. I pull three out one
    after the other and place them side by side from left to
    right. How many different ways can this be done?

    \vfill

    \begin{tabular}{l@{\hspace{3em}}l@{\hspace{3em}}l@{\hspace{3em}}l}
      A: $8!$ & B: $\frac{8!}{3!}$  & C: $\frac{8!}{5!}$ & C: $3!$
    \end{tabular}

    \vfill
    \vfill
    \vfill

  \end{frame}

}



\subsection{Combinations}

\begin{frame}
  \frametitle{Example}

  A box of paper clips contains two-hundred clips. I pull out five
  paper clips without replacement. How many ways can I do this?

  What if I do not care about the order?

\end{frame}


\begin{frame}
  \frametitle{Combinations}

  In general we have 
  \begin{eqnarray*}
    \underbrace{
      \rule{.3cm}{.2mm} \hspace{.05cm} 
      \rule{.3cm}{.2mm} \hspace{.05cm} 
      \rule{.3cm}{.2mm} \hspace{.05cm} 
      \rule{.3cm}{.2mm} \hspace{.05cm} 
      \rule{.3cm}{.2mm} \hspace{.05cm} \ldots
      \rule{.3cm}{.2mm} \hspace{.05cm} 
      \rule{.3cm}{.2mm} \hspace{.05cm}}_{\mathrm{n~possible~items}}
    & \rightarrow & 
    \underbrace{
      \rule{.3cm}{.2mm} \hspace{.05cm} 
      \rule{.3cm}{.2mm} \hspace{.05cm} 
      \rule{.3cm}{.2mm} \hspace{.05cm} \ldots
      \rule{.3cm}{.2mm} \hspace{.05cm} 
      \rule{.3cm}{.2mm} \hspace{.05cm}}_{\mathrm{k~slots}}
  \end{eqnarray*}

  \only<2->%
  {

    There are $\prescript{~}{n}{P}_r$ permutations. Each permutation
    can be expressed in $k!$ different ways,
    \begin{eqnarray*}
      \prescript{~}{n}{P}_r & = & k! \prescript{~}{n}{C}_r, \\
      \prescript{~}{n}{C}_r & = & \frac{\prescript{~}{n}{P}_r}{k!}, \\
      \prescript{~}{n}{C}_r & = & \frac{n!}{(n-k)!k!}.
    \end{eqnarray*}
  }
  
\end{frame}

%\begin{frame}
%  \frametitle{Example}
%
%  \begin{eqnarray*}
%    \prescript{~}{8}{C}_5 & = & \frac{8!}{3!5!}, \\
%    & = & 56.
%  \end{eqnarray*}
%
%\end{frame}

\iftoggle{clicker}{%


  \begin{frame}
    \frametitle{Clicker Quiz}
    What is $\prescript{~}{8}{C}_5$?
 
    \vfill

    \begin{tabular}{l@{\hspace{3em}}l@{\hspace{3em}}l@{\hspace{3em}}l}
      A: $56$ & B: $336$  & C: $6720$ & C: $40320$
    \end{tabular}

    \vfill
    \vfill
    \vfill

  \end{frame}

}


\begin{frame}
  \frametitle{Example}

  I have a box of forty glass jars. I am going to draw two without
  replacement. How many combinations are possible?

\end{frame}


\begin{frame}
  \frametitle{Example}

  I have seven identical cards and three envelopes. How many ways are
  there to put the cards into the envelopes?

\end{frame}






\subsection{Generalized Permutations}

\begin{frame}
  \frametitle{Generalized Permutations}

  I have
  \begin{itemize}
  \item A set of $n_1$ identical objects
  \item Another set of $n_2$ identical objects 
  \item Another set of $n_3$ identical objects \\
    \vdots
  \item Another set of $n_k$ identical objects, \\
  \item $n=n_1+n_2+n_3+\cdots+n_k$.
  \end{itemize}

  The number of ways to arrange these objects is 
  \begin{eqnarray*}
    \frac{n!}{n_1! n_2! n_3! \cdots n_k!}.
  \end{eqnarray*}
  
\end{frame}


%%% Local Variables: 
%%% mode: latex
%%% TeX-master: "IntroStats"
%%% End: 

% LocalWords:  pausesection hideothersubsections sectionstyle
