
\lecture{Poisson Distribution}{poisson}
\section{Poisson Distribution}

\title{Poisson Distribution}
\subtitle{How Many Times?}

%\author{Kelly Black}
%\institute{Clarkson University}
\date{25 September 2013}

\begin{frame}
  \titlepage
\end{frame}

\begin{frame}
  \frametitle{Outline}
  \tableofcontents[hideothersubsections,sectionstyle=show/hide]
\end{frame}


\subsection{Clicker Quiz}


\iftoggle{clicker}{%
  \begin{frame}
    \frametitle{Clicker Quiz}

    A fair die is rolled until a one is rolled. What is the
    probability that the first one appears after five rolls?

    \vfill

    \begin{tabular}{l@{\hspace{3em}}l@{\hspace{3em}}l@{\hspace{3em}}l}
      A: $\lp\frac{1}{6}\rp^4 \frac{5}{6}$ & 
      B: $\lp\frac{1}{6}\rp^5 \frac{5}{6}$ &
      C: $\lp\frac{5}{6}\rp^4 \frac{1}{6}$ & 
      D: $\lp\frac{5}{6}\rp^5 \frac{1}{6}$
    \end{tabular}

    \vfill
    \vfill
    \vfill


  \end{frame}
}

\subsection{Poisson Distribution}

\begin{frame}
  \frametitle{Poisson Distribution}

  \begin{definition}
    A random variable follows a \redText{Poisson distribution} if it
    consists of
    \begin{itemize}
    \item A count of the total number of times some event occurs,
    \item The count occurs under some constraint (ex: within a given
      time).
    \item There is no limit on the total number of times that the
      event can occur.
    \end{itemize}
  \end{definition}


\end{frame}

\begin{frame}{Properties}

  The probability mass function for a Poisson distribution with
  parameter $\lambda$ for the number of times an event can occur
  within the constraint is the following:
  \begin{eqnarray*}
    p(X=k) & = & \frac{e^{-\lambda}\lambda^k}{k!}.
  \end{eqnarray*}

  The expectation and variance of a Poisson distribution are
  \begin{eqnarray*}
    E[X] & = & \lambda, \\
    \mathrm{Var}[X] & = & \lambda.
  \end{eqnarray*}

\end{frame}


\subsection{Examples}

\begin{frame}{Examples}

  \begin{itemize}
  \item Pick a tire at random from a production line and test the
    number of places where the tread's depth is too shallow. Report
    the number of places where the tread depth is too shallow.\\
  \item Observe a polling station and ask the people waiting if they
    support candidate A. Report the number of people in the line who
    support candidate A.
  \item People go surfing in shark infested waters. Count the number
    of people eaten by sharks over a one year period.
  \end{itemize}
  
\end{frame}

\begin{frame}{Example}

  \vfill

  Ladislaus Josephovich Bortkiewicz studied deaths by horse kicks in
  the Prussian Army. He found that the number of deaths per year per
  corp followed a Poisson distribution with a mean of approximately
  four.

  \vfill

  What is the probability of three deaths in a year for a corp?

  \vfill

\end{frame}


\begin{frame}{Example}

  The number of scratches in the paint of the fenders coming off of a
  production line is counted. The mean number of defects is observed
  to be 7.1 scratches per fender. A fender is chosen at random. What
  is the probability it has at least four scratches?

\end{frame}


\begin{frame}{Example}

  The number of people crossing a bridge per day has a mean of twenty
  people. What is the probability that at most twelve people use the
  bridge on a given day?

\end{frame}


\iftoggle{clicker}{%
  \begin{frame}{Clicker Quiz}

    \vfill

    An experiment is run. The potential difference in a solar energy
    system is checked. There are twenty trials, and the system is
    assessed as to whether or not the potential difference falls below
    a 220V threshold. It is estimated that the probability that the
    system will meet or exceed the threshold is 97\% at any time. What
    is the probability that more than eighteen of the checks will meet
    or exceed the threshold?
  
    \vfill

    What kind of distribution is this?

    \vfill

    \begin{tabular}{l@{\hspace{3em}}l@{\hspace{3em}}l@{\hspace{3em}}l}
      A: Binomial  & B: Geometric \\ [12pt] C: Hypergeometric & D: Poisson
    \end{tabular}

    \vfill
  
  \end{frame}
}



\begin{frame}
  \frametitle{Example}

  \vfill

  An experiment is run. The potential difference in a solar energy
  system is checked. There are twenty trials, and the system is
  assessed as to whether or not the potential difference falls below a
  220V threshold. It is estimated that the probability that the system
  will meet or exceed the threshold is 97\% at any time. What is the
  probability that more than eighteen of the checks will meet or
  exceed the threshold?

  \vfill

  (.883)

\end{frame}



\iftoggle{clicker}{%
  \begin{frame}{Clicker Quiz}

    \vfill

    An experiment is run. The potential difference in a solar energy
    system is checked. The system will be assessed for one hour, and
    the number of times the potential difference falls below a 220V
    threshold is determined. It is estimated that the mean number of
    times the system will fall below the threshold in one hour is 3.2
    times. What is the probability that the system will fall below the
    threshold at most three times?
  
    \vfill

    What kind of distribution is this?

    \vfill

    \begin{tabular}{l@{\hspace{3em}}l@{\hspace{3em}}l@{\hspace{3em}}l}
      A: Binomial  & B: Geometric \\ [12pt] C: Hypergeometric & D: Poisson
    \end{tabular}

    \vfill
  
  \end{frame}
}



\begin{frame}
  \frametitle{Example}

  \vfill

  An experiment is run. The potential difference in a solar energy
  system is checked. The system will be assessed for one hour, and the
  number of times the potential difference falls below a 220V
  threshold is determined. It is estimated that the mean number of
  times the system will fall below the threshold in one hour is 3.2
  times. What is the probability that the system will fall below the
  threshold at most three times?

  \vfill

  (.603)

\end{frame}


\iftoggle{clicker}{%
  \begin{frame}{Clicker Quiz}

    \vfill

    An experiment is run. The potential difference in a solar energy
    system is checked. The system will be assessed at sixty minute
    intervals until the potential difference falls below a 220V
    threshold. It is estimated the probability that the system will
    meet or exceed the threshold is 97\% at any time. What is the
    probability that the system will require more than four checks?
  
    \vfill

    What kind of distribution is this?

    \vfill

    \begin{tabular}{l@{\hspace{3em}}l@{\hspace{3em}}l@{\hspace{3em}}l}
      A: Binomial  & B: Geometric \\ [12pt] C: Hypergeometric & D: Poisson
    \end{tabular}

    \vfill
  
  \end{frame}
}



\begin{frame}
  \frametitle{Example}

  \vfill

  An experiment is run. The potential difference in a solar energy
  system is checked. The system will be assessed at sixty minute
  intervals until the potential difference falls below a 220V
  threshold. It is estimated the probability that the system will meet
  or exceed the threshold is 97\% at any time. What is the probability
  that the system will require more than four checks?

  \vfill

  (.913)

\end{frame}


\iftoggle{clicker}{%
  \begin{frame}{Clicker Quiz}

    \vfill

    An experiment is run. The potential differences in some solar
    energy systems are checked. Twenty systems are available to
    determine whether or not the potential difference falls below a
    220V threshold and six will be tested. It is estimated that
    eighteen of the units will meet or exceed the threshold. What is
    the probability that at most one unit will not meet the threshold?
  
    \vfill

    What kind of distribution is this?

    \vfill

    \begin{tabular}{l@{\hspace{3em}}l@{\hspace{3em}}l@{\hspace{3em}}l}
      A: Binomial  & B: Geometric \\ [12pt] C: Hypergeometric & D: Poisson
    \end{tabular}

    \vfill
  
  \end{frame}
}



\begin{frame}
  \frametitle{Example}

  \vfill 

  An experiment is run. The potential differences in some solar energy
  systems are checked. Twenty systems are available to determine
  whether or not the potential difference falls below a 220V threshold
  and six will be tested. It is estimated that eighteen of the units
  will meet or exceed the threshold. What is the probability that at
  most one unit will not meet the threshold?

  \vfill

  (.921)

\end{frame}



% LocalWords:  Clarkson pausesection hideallsubsections Ladislaus
%  LocalWords:  Hypergeometric hypergeometric Josephovich Bortkiewicz

%%% Local Variables: 
%%% mode: latex
%%% TeX-master: t
%%% End: 
