
\lecture{Sample Distributions}{sample-distributions}
\part{Sample Distributions}

\title{Sample Distributions}
\subtitle{The Sample Mean Varies}

\author{Kelly Black}
\institute{Clarkson University}
\date{24 Feb 2012}

\begin{frame}
  \titlepage
\end{frame}

\begin{frame}
  \frametitle{Outline}
  \tableofcontents[pausesection,hideallsubsections,part=1]
\end{frame}


\section{Clicker Quiz}


\begin{frame}{Clicker Quiz}

  Clarkson University claims that over eighty percent of all students
  take part in intramural sports. You choose three hundred students at
  random. What is the probability that more than two-hundred and fifty will say
  that they take part in intramural sports?

  \vfill

  \begin{tabular}{l@{\hspace{3em}}l@{\hspace{3em}}l}
    A: 0.0643  & B: 0.4679 & C: 0.9375
  \end{tabular}

  \vfill
  \vfill
  \vfill

\end{frame}




\section{Coin Flip}

\begin{frame}
  \frametitle{Coin Flip}

  Flip a Coin

  \only<2->{How many tails?}

  \only<3->{Divide by the number of people.}

  \only<4->{Do it again.}

  \only<5->{Do it again.}

\end{frame}

\section{The Sample Mean}

\begin{frame}
  \frametitle{Sample Mean}

  We have a bunch of measurements:
  \begin{eqnarray*}
    \bar{x} & = & x_1,~x_2,~x_3,\cdots,~x_n.
  \end{eqnarray*}
  Each measurement has a mean, $\mu$, and a standard deviation of
  $\sigma$. We assume that they are all independent of one another.
  
  \only<1-2>
  {
    \begin{definition}
      Given measurements the sample mean is 
      \begin{eqnarray*}
        \bar{x} & = & \frac{x_1+x_2+x_3+\cdots+x_n}{n}.
      \end{eqnarray*}
    \end{definition}
  }

  \only<3->
  {
    \begin{definition}
      Given measurements the sample mean is 
      \begin{eqnarray*}
        \bar{x} & = & \frac{x_1+x_2+x_3+\cdots+x_n}{n}.
      \end{eqnarray*}
      The sample mean, $\bar{x}$,  has a mean of $\mu$, and it has a
      standard deviation of $\frac{\sigma}{\sqrt{n}}$.
    \end{definition}
  }


  \only<2->
  {
    The sample mean is a random variable!
  }

\end{frame}


\begin{frame}
  (central limit theorem example)
\end{frame}

\section{Examples}

\begin{frame}
  \frametitle{}

  A random variable has a mean of 3.0 and a standard deviation of
  4.5. If I take one sample what is the probability that it is less
  than zero?

  \vfill

  \only<2->
  {
    If I take four samples what is the probability that it is less
    than zero?
  }

  \vfill

\end{frame}


\begin{frame}
  \frametitle{Clicker Quiz}

  The total change in a stock's price on one particular day has a mean
  of -.35\$ with a standard deviation of .86\$. A sample of 16 stocks
  is taken. What is the probability that the mean will be more than
  zero?

  \vfill

  \begin{tabular}{l@{\hspace{3em}}l@{\hspace{3em}}l@{\hspace{3em}}l}
    A: 0.0516  & B: 0.3420 & C: 0.6580 & D: 0.9484
  \end{tabular}

  \vfill
  \vfill
  \vfill

\end{frame}


\begin{frame}
  \frametitle{Example}

  I have a random variable. It is normally distributed with a mean of
  0.5 and a standard deviation of 2.0. How many samples should I take
  so that
  \begin{eqnarray*}
    p(\bar{x} < 0.25) & = & 0.05?
  \end{eqnarray*}

\end{frame}




% LocalWords:  Clarkson pausesection hideallsubsections
