
\lecture{Introduction to Continuous Data}{introduction-to-continuous-data}
\section{Introduction to Continuous Data}

\title{Quantitative Data}
\subtitle{Methods to Assess Numerical Data}

%\author{Kelly Black}
%\institute{Clarkson University}
\date{15 October 2014}

\begin{frame}
  \titlepage
\end{frame}

\begin{frame}
  \frametitle{Outline}
  \tableofcontents[hideothersubsections,sectionstyle=show/hide]
\end{frame}


\subsection{Clicker Quiz}


\iftoggle{clicker}{%
  \begin{frame}
    \frametitle{Clicker Quiz}

    A random variable is normally distributed with a mean of 1.2 and a
    standard deviation of 2.5. What is the probability that it is
    larger than 3.0?
    
    \vfill

    \begin{tabular}{l@{\hspace{3em}}l@{\hspace{3em}}l@{\hspace{3em}}l}
      A: 0.236 & B: 0.720  & C: 0.764 & D: 0.999
    \end{tabular}

    \vfill

  \end{frame}
}

\begin{frame}{Things You Need to Know}

  Read Sections 6.1, 6.2, and 6.4:
  \begin{itemize}
  \item Pareto Chart \redText{Discrete Data}
  \item Bar Charts \redText{Discrete Data}
  \item Histograms \redText{Continuous Data}
  \item Skew (left vs. right)
  \item Sample Mode
  \item Sample Mean
  \item Sample Variance and Sample Standard Deviation
  \item Trimmed Data
  \item Coefficient of Variation
  \item Sample Quartiles
  \item Boxplots 
  \end{itemize}
  
\end{frame}



\subsection{Quantitative Data}

\begin{frame}{Quantitative Data}

  \begin{definition}[Qualitative Data]

    A discrete random variable produces a countable number of
    values. The values are one of a fixed set of values.
    
  \end{definition}
  
  \begin{definition}[Discrete Quantitative Data]

    A discrete random variable produces a countable number of
    numbers. The values are one of a fixed set of values. 

  \end{definition}

  \begin{definition}[Continuous Data]

    The numbers produced come from an infinite number of possible
    outcomes.
    
  \end{definition}

\end{frame}

\begin{frame}{Examples}

  \begin{itemize}
  \item Count the number of people who arrive in a line during a given
    time period. \redText{Discrete, quantitative data}

  \item Measure the amperage across a circuit
    element. \redText{Continuous, quantitative data}

  \item Ask someone how much they like a ride at a park on a scale
    from one to five. \redText{Qualitative data}

  \end{itemize}
  
\end{frame}


\begin{frame}{Quantitative Data}
  
  \begin{definition}[Univariate Data]
    The data consists of a set of numbers, and there are no explicit
    relationships between them.
  \end{definition}

  \begin{definition}[Bivariate Data]
    The data consists of pairs of numbers, and the pairs have some
    assumed relationship to one another.
  \end{definition}

  \begin{definition}[Multivariate Data]
    The data consists of groups of numbers, and the groups have some
    assumed relationship to one another. There is more than two
    numbers per group.
  \end{definition}


\end{frame}



\subsection{Bivariate Data}

\begin{frame}
  \frametitle{Bivariate Data}

  \begin{eqnarray*}
    \begin{array}{l|l}
      x      & y \\ \hline 
      x_1    & y_1 \\
      x_2    & y_2 \\
      \vdots & \vdots \\
      x_n    & y_n
    \end{array}
  \end{eqnarray*}
\end{frame}



\iftoggle{clicker}{%
  \begin{frame}
    \frametitle{Clicker Quiz}

    An experiment is run in which an intersection is observed over
    twenty different two hour periods. The time period and the number
    of people who do not stop at a stop sign is observed and is
    recorded. .

    What kind of data is this?

    \vfill

    \begin{tabular}{l@{\hspace{3em}}l@{\hspace{3em}}l@{\hspace{3em}}l}
      A: univariate, discrete & B: univariate, continuous  \\
      C: bivariate, discrete  & D: bivariate, continuous
    \end{tabular}

    \vfill



  \end{frame}
}

\subsection{Discrete Quantitative Data}

\begin{frame}{Discrete Data}

  Examples:

  \begin{itemize}
  \item a, b, b, a, b, b, b, a, b
  \item 2, 2, 3, 2, 4, 1, 3, 2
  \end{itemize}
  
\end{frame}

\iftoggle{clicker}{%
  \begin{frame}
    \frametitle{Clicker Quiz}

    Determine the number of times the letter a occurs in the following
    data set: \\
      a, b, b, c, a, b, b, c, b, c

    \vfill

    \begin{tabular}{l@{\hspace{3em}}l@{\hspace{3em}}l@{\hspace{3em}}l}
      A: 0 & B: 1  & C: 2 & D: 3
    \end{tabular}

    \vfill

  \end{frame}
}


\begin{frame}
  \frametitle{Discrete Data}

  \vfill

  Data: \\
  a, b, b, c, a, b, b, c, b, c

  \vfill

  \begin{tabular}{l|l|l}
    Observation  & Frequency & \uncover<2>{Relative Frequency} \\ \hline
    a & 2 & \uncover<2>{2/10} \\
    b & 5 & \uncover<2>{5/10} \\
    c & 3 & \uncover<2>{3/10}
  \end{tabular}

  \vfill

  \uncover<2>{%
    \textit{To calculate the relative frequency divide by the total
      number of data points.}
  }

  \vfill

\end{frame}

\begin{frame}{Sample Mode}

  \begin{definition}[Sample Mode]
    The sample mode is/are the observations with the highest frequency.
  \end{definition}

  \uncover<2>{%
    Data: \\
    a, b, b, c, a, b, b, c, b, c

    \vfill

    \begin{tabular}{l|l}
      Number  & Frequency \\ \hline
      a & 2 \\
      \redText{b} & \redText{5} \\
      c & 3 
    \end{tabular}

    The sample mode is 5. If more than one of the values has the maximum
    frequency then the set of modes includes all of the values that
    occur the maximum number of times.

  }

\end{frame}

\subsection{Continuous Data}



\begin{frame}{Continuous Data}

  \begin{tabular}{l}
    Data \\ \hline
    $x_1$ \\
    $x_2$ \\
    $x_3$ \\
    $x_4$ \\
    $\vdots$ \\
    $x_{n-1}$ \\
    $x_{n}$ \\
  \end{tabular}

  \vfill

  It makes sense to perform arithmetic operations on the numbers.
  
\end{frame}

\begin{frame}{Sample Mean and Sample Variance}

  \begin{eqnarray*}
    x_1,~x_2,~x_3,~x_4,\ldots,x_{n-1},~x_n.
  \end{eqnarray*}

  \begin{definition}{Sample Mean}
    The sample mean of a set of numbers is 
    \begin{eqnarray*}
      \begin{array}{lclcl}
        \bar{x} & = & \frac{x_1+x_2+x_3+x_4+\cdots+x_{n-1}+x_n}{n}
        & = & \frac{1}{n}\sum^n_{i=1} x_i.
    \end{array}
    \end{eqnarray*}
  \end{definition}

  \begin{definition}{Sample Variation}
    The sample variation of a set of numbers is 
    \begin{eqnarray*}
      s^2_x & = & \frac{\lp x_1-\bar{x}\rp^2 + \lp x_2-\bar{x}\rp^2 +
                  \cdots+\lp x_{n-1} - \bar{x} \rp^2 + \lp x_n - \bar{x}\rp^2}{n-1}, \\
      & = & \frac{\sum^n_{i=1} \lp x_i - \bar{x} \rp^2 }{n-1}.
    \end{eqnarray*}
  \end{definition}

  
\end{frame}

\begin{frame}{Example}

  \begin{tabular}{l}
    1.2 \\ 1.5 \\ 2.1 \\ 4.3 \\ 1.7
  \end{tabular}

\end{frame}

\begin{frame}{Sample Quartiles}

  \begin{enumerate}
  \item Sort the data
  \item Divide into sets of four
  \end{enumerate}

  (Graphical view: boxplot)
  
\end{frame}

\begin{frame}{Sample Quantiles}

  \newcount\xnum
  \newcount\xnumpos

    \begin{picture}(310,100)(0,0)
      \multiput(10,90)(20, 0){15}{\line(1,0){15}}
      \xnum=1
      \xnumpos=12
      \loop
      \put(\xnumpos,95){{\color{red}$x_{{\the\xnum}}$}}
      \advance\xnumpos by 20
      \ifnum\xnum < 3 \advance\xnum by 1
      \repeat

      \xnum=5
      \xnumpos=92
      \loop
      \put(\xnumpos,95){{\color{blue}$x_{{\the\xnum}}$}}
      \advance\xnumpos by 20
      \ifnum\xnum < 7 \advance\xnum by 1
      \repeat

      \xnum=9
      \xnumpos=172
      \loop
      \put(\xnumpos,95){{\color{Violet}$x_{{\the\xnum}}$}}
      \advance\xnumpos by 20
      \ifnum\xnum < 12 \advance\xnum by 1
      \repeat

      \xnum=13
      \xnumpos=252
      \loop
      \put(\xnumpos,95){{\color{Brown}$x_{{\the\xnum}}$}}
      \advance\xnumpos by 20
      \ifnum\xnum < 15 \advance\xnum by 1
      \repeat


      \put(72,95){$x_{4}$}
      \put(152,95){$x_{8}$}
      \put(232,95){$x_{12}$}

      \put(10,0){\line(0,1){80}}
      \put(310,0){\line(0,1){80}}
      \put(110,10){\vector(-1,0){100}}
      \put(210,10){\vector( 1,0){100}}
      \put(120,5){Range}

      \put(156,25){\line(0,1){55}}
      \put(60,30){\vector(-1,0){50}}
      \put(105,30){\vector( 1,0){50}}
      \put(63,25){Median}

      \put(78,45){\line(0,1){25}}
      \put(30,50){\vector(-1,0){20}}
      \put(58,50){\vector(1,0){20}}
      \put(37,45){Q1}


      \put(240,45){\line(0,1){25}}
      \put(260,50){\vector(-1,0){20}}
      \put(290,50){\vector(1,0){20}}
      \put(265,45){Q3}


    \end{picture}
  
\end{frame}

\begin{frame}{Sample Quantiles}

    \begin{picture}(310,100)(0,0)
      \multiput(20,90)(20, 0){14}{\line(1,0){15}}
      \xnum=1
      \xnumpos=22
      \loop
      \put(\xnumpos,95){{\color{red}$x_{{\the\xnum}}$}}
      \advance\xnumpos by 20
      \ifnum\xnum < 3 \advance\xnum by 1
      \repeat

      \xnum=5
      \xnumpos=102
      \loop
      \put(\xnumpos,95){{\color{blue}$x_{{\the\xnum}}$}}
      \advance\xnumpos by 20
      \ifnum\xnum < 7 \advance\xnum by 1
      \repeat

      \xnum=8
      \xnumpos=162
      \loop
      \put(\xnumpos,95){{\color{Violet}$x_{{\the\xnum}}$}}
      \advance\xnumpos by 20
      \ifnum\xnum < 10 \advance\xnum by 1
      \repeat

      \xnum=12
      \xnumpos=242
      \loop
      \put(\xnumpos,95){{\color{Brown}$x_{{\the\xnum}}$}}
      \advance\xnumpos by 20
      \ifnum\xnum < 14 \advance\xnum by 1
      \repeat


      \put(82,95){$x_{4}$}
      \put(222,95){$x_{11}$}

      \put(20,0){\line(0,1){90}}
      \put(300,0){\line(0,1){90}}
      \put(110,10){\vector(-1,0){90}}
      \put(210,10){\vector( 1,0){90}}
      \put(120,5){Range}

      \put(156,25){\line(0,1){55}}
      \put(60,30){\vector(-1,0){40}}
      \put(105,30){\vector( 1,0){50}}
      \put(63,25){Median}

      \put(85,45){\line(0,1){25}}
      \put(40,50){\vector(-1,0){20}}
      \put(65,50){\vector(1,0){20}}
      \put(45,45){Q1}


      \put(230,45){\line(0,1){25}}
      \put(250,50){\vector(-1,0){20}}
      \put(280,50){\vector(1,0){20}}
      \put(255,45){Q3}


    \end{picture}
  
\end{frame}


\begin{frame}{Sample Quantiles}

    \begin{picture}(310,100)(0,0)
      \multiput(30,90)(20, 0){13}{\line(1,0){15}}
      \xnum=1
      \xnumpos=32
      \loop
      \put(\xnumpos,95){{\color{red}$x_{{\the\xnum}}$}}
      \advance\xnumpos by 20
      \ifnum\xnum < 3 \advance\xnum by 1
      \repeat

      \xnum=4
      \xnumpos=92
      \loop
      \put(\xnumpos,95){{\color{blue}$x_{{\the\xnum}}$}}
      \advance\xnumpos by 20
      \ifnum\xnum < 6 \advance\xnum by 1
      \repeat

      \xnum=8
      \xnumpos=172
      \loop
      \put(\xnumpos,95){{\color{Violet}$x_{{\the\xnum}}$}}
      \advance\xnumpos by 20
      \ifnum\xnum < 10 \advance\xnum by 1
      \repeat

      \xnum=11
      \xnumpos=232
      \loop
      \put(\xnumpos,95){{\color{Brown}$x_{{\the\xnum}}$}}
      \advance\xnumpos by 20
      \ifnum\xnum < 13 \advance\xnum by 1
      \repeat


      \put(152,95){$x_{7}$}

      \put(30,0){\line(0,1){90}}
      \put(290,0){\line(0,1){90}}
      \put(120,10){\vector(-1,0){90}}
      \put(200,10){\vector( 1,0){90}}
      \put(145,5){Range}

      \put(156,25){\line(0,1){55}}
      \put(70,30){\vector(-1,0){40}}
      \put(105,30){\vector( 1,0){50}}
      \put(70,25){Median}

      \put(88,45){\line(0,1){25}}
      \put(50,50){\vector(-1,0){20}}
      \put(68,50){\vector(1,0){20}}
      \put(50,45){Q1}


      \put(228,45){\line(0,1){25}}
      \put(248,50){\vector(-1,0){20}}
      \put(270,50){\vector(1,0){20}}
      \put(255,45){Q3}


    \end{picture}
  
\end{frame}


\begin{frame}{Example}

  \begin{tabular}{l}
    1.2 \\ 1.5 \\ 2.1 \\ 4.3 \\ 1.7 \\ 1.5
  \end{tabular}

  

\end{frame}

\begin{frame}{Trimmed Data}

  Delete some percentage of the largest and smallest numbers.

  Example: 
  We have one hundred data points.

  \vfill

  For 100 points, a 10\% trimmed mean: delete the 10 smallest and the 10 largest
  numbers.

  \vfill

  For 100 points, a 5\% trimmed mean: delete the 5 smallest and the 5
  largest numbers.

  \vfill

\end{frame}



% LocalWords:  Clarkson pausesection hideallsubsections Bivariate
