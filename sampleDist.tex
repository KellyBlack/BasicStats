
\lecture{More Sample Distributions}{more-sample-distributions}
\section{More Sample Distributions}

\title{More Sample Distributions}
\subtitle{More Examples}

%\author{Kelly Black}
%\institute{Clarkson University}
\date{Feb 27, 2013}

\begin{frame}
  \titlepage
\end{frame}

\begin{frame}
  \frametitle{Outline}
  \tableofcontents[hideothersubsections,sectionstyle=show/hide]
\end{frame}


\subsection{Clicker Quiz}


\iftoggle{clicker}{%
  \begin{frame}
    \frametitle{Clicker Quiz}

    The mass of the contents of a product are normally distributed
    with a mean of 0.35 kg and a standard deviation of 0.04 kg. What
    is the probability that one sample will be less than 0.33 kg?

    \vfill

    \begin{tabular}{l@{\hspace{3em}}l@{\hspace{3em}}l@{\hspace{3em}}l}
      A: .31  & B: 0.34  & C: 0.50 & D: 0.69
    \end{tabular}

    \vfill
    \vfill
    \vfill

  \end{frame}
}


\subsection{The Sample Mean}

\iftoggle{clicker}{%
  \begin{frame}{Flip a Coin}

    Everybody: Flip a coin three times. Count the number of
    Heads. Press the number associated with the count on your clicker.

    \only<2->%
    {
      Do it again.
    }
  
  \end{frame}
}

\begin{frame}{Coin Flip}

What did we just do? 

  \only<2->%
  {
    We just calculated a sample mean. Twice!
  }

  \only<3->%
  {
    Each time we run these experiments we can get different results.

    The sample mean is a random variable.
  }
  
\end{frame}


\begin{frame}
  \frametitle{Sample Mean}

  We have a bunch of measurements:
  \begin{eqnarray*}
    x_1,~x_2,~x_3,\cdots,~x_n.
  \end{eqnarray*}
  Each measurement is a Bernoulli distribution with parameter $p$. We
  assume that they are all independent of one another.
  
  \only<1-2>
  {
    \begin{definition}
      Given measurements the sample proportion is 
      \begin{eqnarray*}
        \hat{p} & = & \frac{x_1+x_2+x_3+\cdots+x_n}{n}.
      \end{eqnarray*}
    \end{definition}
  }

  \only<3->
  {
    \begin{definition}
      Given measurements the sample proportion is 
      \begin{eqnarray*}
        \hat{p} & = & \frac{x_1+x_2+x_3+\cdots+x_n}{n}.
      \end{eqnarray*}
      The sample proportion, $\bar{p}$,  has a mean of $p$, and it has a
      standard deviation of $\sqrt{\frac{p(1-p)}{n}}$.
    \end{definition}
  }


  \only<2->{The sample proportion is a random variable!}

\end{frame}


\begin{frame}
  \frametitle{Sample Mean}

  We have a bunch of measurements:
  \begin{eqnarray*}
    \bar{x} & = & x_1,~x_2,~x_3,\cdots,~x_n.
  \end{eqnarray*}
  Each measurement has a mean, $\mu$, and a standard deviation of
  $\sigma$. We assume that they are all independent of one another.
  
  \only<1-2>
  {
    \begin{definition}
      Given measurements the sample mean is 
      \begin{eqnarray*}
        \bar{x} & = & \frac{x_1+x_2+x_3+\cdots+x_n}{n}.
      \end{eqnarray*}
    \end{definition}
  }

  \only<3->
  {
    \begin{definition}
      Given measurements the sample mean is 
      \begin{eqnarray*}
        \bar{x} & = & \frac{x_1+x_2+x_3+\cdots+x_n}{n}.
      \end{eqnarray*}
      The sample mean, $\bar{x}$,  has a mean of $\mu$, and it has a
      standard deviation of $\frac{\sigma}{\sqrt{n}}$.
    \end{definition}
  }


  \only<2->
  {
    The sample mean is a random variable!
  }

\end{frame}





\begin{frame}
  \frametitle{Example}

  The resistance across a resistor is measured. The measurements have
  mean 3,000 Ohms and a standard deviation of 800 Ohms. What is the
  probability that a sample of fifty measurements will be between 2200
  Ohms and 3800 Ohms.

\end{frame}



\subsection{Central Limit Theorem}


\begin{frame}
  (central limit theorem example)
\end{frame}




\subsection{t-distribution}

\begin{frame}
  \frametitle{Example}

  An experiment is conducted. Twenty measurements are taken from a
  random variable. The \redText{sample} standard deviation is 0.50,
  and the sample mean is 3.15. Your colleague says that the mean
  should be 3.0, and your experiment is wrong. Is there at least a ten
  percent chance that you could have found this result
  \redText{\textit{or worse}} if the true mean is 3.0?

\end{frame}

