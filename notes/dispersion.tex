
\lecture{Central Tendencies}{central-tendencies}
\section{Central Tendencies}

\title{Central Tendencies}
\subtitle{Center and Spread of Data}

%\author{Kelly Black}
%\institute{Clarkson University}
\date{13 February 2013}

\begin{frame}
  \titlepage
\end{frame}

\begin{frame}
  \frametitle{Outline}
  \tableofcontents[hideothersubsections,sectionstyle=show/hide]
\end{frame}


\iftoggle{clicker}{%
  \subsection{Clicker Question}
  \begin{frame}
    \frametitle{Find The Median}
    (Use channel 42)

    \vfill 
    
    \begin{tabular}{lll}
      3, & 5, & -2
    \end{tabular}

    \vfill

    \begin{tabular}{l@{\hspace{3em}}l@{\hspace{3em}}l}
      a: -2 & b: 2 & c: 4
    \end{tabular}

    \vfill


  \end{frame}
}



\subsection{Spread of Data}


\begin{frame}
  \frametitle{Spread of Data}

  Given data:
  \begin{eqnarray*}
    x_1, ~ x_2, ~ x_3, ~ \ldots ~,x_n.
  \end{eqnarray*}

  \begin{definition}
    The range is the biggest number minus the smallest number.
  \end{definition}


  \begin{definition}
    The sample variation of the data is 
    \begin{eqnarray*}
      s^2 & = & \frac{(x_1-\bar{x})^2+(x_2-\bar{x})^2+\cdots+(x_n-\bar{x})^2}{n-1}.
    \end{eqnarray*}
  \end{definition}

  \begin{definition}
    The sample standard deviation of the data is the square root of
    the sample variation.
  \end{definition}


\end{frame}


\begin{frame}
  \frametitle{Clicker Quiz}
  (Use channel 42)

  Find the standard deviation of the following data set:
  \vfill 

  \begin{tabular}{llll}
    3, & 5, & 2, & 4
  \end{tabular}

  \vfill

  \begin{tabular}{l@{\hspace{3em}}l@{\hspace{3em}}l}
    a: 1.667 & b: 1.180 & c: 1.291
  \end{tabular}

  \vfill

  

\end{frame}





% LocalWords:  Clarkson pausesection hideallsubsections
