
\lecture{Sample Proportions}{sample-proportions}
\section{Sample Proportions}

\title{Sample Proportions}
\subtitle{Sampling a Binomial Distribution}

%\author{Kelly Black}
%\institute{Clarkson University}
\date{20 February 2015}

\begin{frame}
  \titlepage
\end{frame}

\begin{frame}
  \frametitle{Outline}
  \tableofcontents[hideothersubsections,sectionstyle=show/hide]
\end{frame}



\subsection{Clicker Quiz}


\iftoggle{clicker}{%
  \begin{frame}{Clicker Quiz}

    \iftoggle{clicker}{%

      It is estimated that 51\% of the people in a congressional
      district support the incumbent.  A poll is conducted in which 1000
      people are called and asked if they support the incumbent. The
      total number of people who say ``yes'' is counted. What is the
      distribution that the total will follow?

      \vfill 


      \begin{tabular}{l@{\hspace{3em}}l@{\hspace{3em}}l@{\hspace{3em}}l}
        A: Binomial  & B: Normal & C: Poisson
      \end{tabular}

      \vfill
      \vfill
      \vfill

    }

  \end{frame}
}





\subsection{Sample Proportions}

\begin{frame}
  \frametitle{Clicker Example}

  Would you vote for Hillary Clinton?

  \begin{tabular}{l@{\hspace{3em}}l}
    A: Yes    & B: No
  \end{tabular}
  
  

  \vfill

  \uncover<2->{ 

    \textcolor{red}{Left section}: 
    Would you vote for Hillary Clinton?

    \begin{tabular}{l@{\hspace{3em}}l@{\hspace{3em}}l}
      A: Yes   & B: No & C: Not in left section
    \end{tabular}

  }

  \vfill

  \uncover<3->{ 

    \textcolor{red}{Center section}: 
    Would you vote for Hillary Clinton?

    \begin{tabular}{l@{\hspace{3em}}l@{\hspace{3em}}l}
      A: Yes   & B: No& C: Not in center section
    \end{tabular}

  }

  \vfill

  \uncover<4->{ 

    \textcolor{red}{Right section}: 
    Would you vote for Hillary Clinton?

    \begin{tabular}{l@{\hspace{3em}}l@{\hspace{3em}}l}
      A: Yes    & B: No & C: Not in right section
    \end{tabular}

  }

  \vfill


\end{frame}


\begin{frame}
  \frametitle{What Just Happened?}

  \begin{itemize}
  \item We just made samples. (This is a not a random sample, but you
    get the idea.)

  \item Each time you take a sample you \textit{could} get something
    different.

  \item We make a calculation based on the sample. The calculation is
    called the ``sample proportion.''

    \uncover<2->{%
      \begin{definition}[The Sample Proportion]
        The sample proportion is defined to be
        \begin{eqnarray*}
          \hat{p} & = & \frac{\mathrm{Number~of~Yeses}}{\mathrm{Total~Number}}.
        \end{eqnarray*}
      \end{definition}

    }

  \end{itemize}

  \vfill


\end{frame}

\begin{frame}{Sampling a Binomial Distribution}

  Recall that if $X$ follows a binomial distribution. 
  \begin{itemize}
  \item There are $n$ independent trials.
  \item Each trial has a probability of a ``yes.''
  \item Check if $n\cdot p \geq 5$ 
  \item Check if $n\cdot (1-p) \geq 5$ 
  \item If \textcolor{red}{\textbf{both}} conditions above are true
    then $X$ is \textcolor{red}{approximately normally distributed}
    with mean $n\cdot p$ and standard deviation $\sqrt{p(1-p)n}$.
  \end{itemize}

  
\end{frame}


\begin{frame}
  \frametitle{Not What We Have!}

  We are not seeking the number ``yeses.''  Instead we are asked about
  the \textcolor{red}{\textit{proportion}} of ``yeses.''

  \vfill

  \begin{definition}[\textcolor{red}{The sample proportion}]

    We take a sample from $n$ trials. Each trial has probability $p$
    of ``success.'' We count the total number of successes from the
    $n$ trials.  The sample proportion is defined to be
    \begin{eqnarray*}
      \hat{p} & = & \frac{\mathrm{number~of~successes}}{n}.
    \end{eqnarray*}
    
  \end{definition}



\end{frame}

\begin{frame}
  \frametitle{The Sample Proportion}

  The sample proportion has a mean of $p$, and a standard deviation of
  $\sqrt{\frac{p(1-p)}{n}}$.

  \vfill

  \uncover<2->
  {

    If $p\cdot n \geq 5$ \textbf{and} $(1-p)\cdot n \geq 5$
    \textbf{and} \textcolor{red}{$n\geq 20$} then $\hat{p}$ is
    approximately normally distributed with mean $p$ and standard
    deviation $\sqrt{\frac{p(1-p)}{n}}$.

  }

  \vfill

  \uncover<3->
  {
    If this is the case we have
    \begin{eqnarray*}
      z & = & \frac{\hat{p}-p}{\sqrt{\frac{p(1-p)}{n}}}.
    \end{eqnarray*}
  }

  \vfill

\end{frame}


\begin{frame}{The Sample Proportion}

  If we can use the normal approximation we have
  \begin{eqnarray*}
    z & = & \frac{\hat{p}-p}{\sqrt{\frac{\textcolor{red}{p(1-p)}}{n}}}.
  \end{eqnarray*}

  \vfill

  Problem: We usually do not know $p$. {\color{blue}If necessary use
    the largest possible value of the standard deviation as a
    ``worst case scenario.''} That occurs when $p=\frac{1}{2}$, and
    we sometimes use that if we are not sure.

    \vfill

\end{frame}

\subsection{Examples}

\begin{frame}
  \frametitle{Example}

  Roughly 51\% of the people in a district support the
  incumbent. One-thousand people are polled. What is the probability
  that less than half of the people will say ``yes?''

  \vfill

  \note{

    \begin{itemize}
    \item Explicitly check the conditions to make sure that you can
      use the normal approximation and show your work!
    \item Answer: .2643
    \end{itemize}

  }

\end{frame}

\begin{frame}{Example}

  It is estimated that 64\% of the people who get a flu shot this year
  will not get the flu. Four hundred people who have had the shot are
  polled. What is the probability that less than 34\% of them will get
  the flu?

    \vfill


  \note{%
    \begin{itemize}
    \item Explicitly check the conditions to make sure that you can
      use the normal approximation and show your work!
    \item Be careful when reading the problem. This can be confusing!
    \item Answer: 0.2033
    \end{itemize}

  }


\end{frame}



\begin{frame}{In a nutshell}


  You conduct $n$ trials. Each trial has a probability $p$ for a
  ``yes'' and a probability of $1-p$ otherwise. The sample
  proportion is defined to be
  \begin{eqnarray*}
    \hat{p} & = & \frac{\mathrm{number~of~yeses}}{\mathrm{number~of~trials}}.
  \end{eqnarray*}

  \begin{itemize}
  \item The sample proportion has a mean of $p$, and a standard
    deviation of $\sqrt{\frac{p(1-p)}{n}}$.
  \item Check to see if $p\cdot n \geq 5$, $(1-p)\cdot n \geq 5$ and
    $n\geq 20$. If \textit{all} of these things are true then a
    $z$-statistic can be approximated using
    \begin{eqnarray*}
      z & = & \frac{\hat{p}-p}{\sqrt{\frac{p(1-p)}{n}}}.
    \end{eqnarray*}


  \end{itemize}



  
\end{frame}


\iftoggle{clicker}{%

  \begin{frame}{Clicker Quiz}

    Find the value of $z^*$ so that the probability of obtaining a
    value less than $z^*$ from a standard normal is 0.05.

    \vfill 

    \begin{tabular}{l@{\hspace{3em}}l@{\hspace{3em}}l@{\hspace{3em}}l}
      A: -1.96  & B: -1.65 & C: 1.65 & D: 1.96
    \end{tabular}

    \vfill
    \vfill
    \vfill



\end{frame}

}



\begin{frame}{Example}

  I want to conduct a poll. It is estimated that each time I ask
  someone the question there is a probability of 0.35 that the person
  will say ``yes.'' How many people should I poll so that there is a
  probability of 0.05 that the sample proportion will be less than
  0.30?

  \vfill

  \note{%

    \begin{itemize}
    \item We assume that we can use the normal approximation.
    \item Ans: $n=248$ 
    \item Here we always round up!
    \end{itemize}

  }

\end{frame}

\begin{frame}{Example}

  I want to conduct a poll, and each question will be answered with
  either a ``yes'' or a ``no.''  How many people should I poll so that
  there is a probability of 0.05 that the sample proportion will be
  less than 0.1 lower than the true value?

  \vfill

  \note{%

    \begin{itemize}
    \item We assume that we can use the normal approximation.
    \item Ans: $n=69$ 
    \item Here we always round up!
    \end{itemize}

  }


\end{frame}



% LocalWords:  Clarkson pausesection hideallsubsections hideothersubsections
% LocalWords:  sectionstyle Kanye parotid
