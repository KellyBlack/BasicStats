
\lecture{More Sample Distributions}{more-sample-distributions}
\section{The Sample Distribution}

\title{The Sample Distribution}
\subtitle{The Central Limit Theorem}

%\author{Kelly Black}
%\institute{Clarkson University}
\date{Feb 18, 20145}

\begin{frame}
  \titlepage
\end{frame}

\begin{frame}
  \frametitle{Outline}
  \tableofcontents[hideothersubsections,sectionstyle=show/hide]
\end{frame}


\subsection{Clicker Quiz}


\iftoggle{clicker}{%
  \begin{frame}
    \frametitle{Clicker Quiz}

    The mass of the contents of a product are normally distributed
    with a mean of 0.35 kg and a standard deviation of 0.04 kg. What
    is the probability that one sample will be less than 0.33 kg?

    \vfill

    \begin{tabular}{l@{\hspace{3em}}l@{\hspace{3em}}l@{\hspace{3em}}l}
      A: .31  & B: 0.34  & C: 0.50 & D: 0.69
    \end{tabular}

    \vfill
    \vfill
    \vfill

  \end{frame}
}


\subsection{The Sample Mean}

\iftoggle{clicker}{%
  \begin{frame}{Flip a Coin}

    Everybody: Flip a coin three times. Count the number of
    Heads. Press the number associated with the count on your clicker.

    \only<2->%
    {
      Do it again.
    }
  
  \end{frame}
}

\begin{frame}{Coin Flip}

What did we just do? 

  \only<2->%
  {
    We just calculated a sample mean. Twice!
  }

  \only<3->%
  { 

    Each time we run these experiments we \textcolor{red}{can} get
    different results.

    The sample mean is a random variable.
  }
  
\end{frame}

\subsection{The Central Limit Theorem}

\begin{frame}
  \frametitle{Sample Mean}

  We have a set of measurements:
  \begin{eqnarray*}
    x_1,~x_2,~x_3,\cdots,~x_n.
  \end{eqnarray*}
  Each measurement has a mean, $\mu$, and a standard deviation of
  $\sigma$. We assume that they are all independent of one another.
  
  \only<2->
  {
    \begin{definition}
      Given measurements the sample mean is 
      \begin{eqnarray*}
        \bar{x} & = & \frac{x_1+x_2+x_3+\cdots+x_n}{n}.
      \end{eqnarray*}
    \end{definition}
  }

  \only<3->
  {
    The sample mean is a random variable!
  }

\end{frame}


\begin{frame}
  \frametitle{Sample Mean}

  We have a set of measurements:
  \begin{eqnarray*}
    x_1,~x_2,~x_3,\cdots,~x_n.
  \end{eqnarray*}
  
  \begin{definition}
    Given measurements the sample mean is 
    \begin{eqnarray*}
      \bar{x} & = & \frac{x_1+x_2+x_3+\cdots+x_n}{n}.
    \end{eqnarray*}
    The sample mean, $\bar{x}$,  has a mean of $\mu$, and it has a
    standard deviation of $\frac{\sigma}{\sqrt{n}}$.
  \end{definition}

\end{frame}


\begin{frame}
  (central limit theorem example)
\end{frame}


\begin{frame}
  \frametitle{Distribution of the Sample Mean}

  We have a random variable, $X$, which has mean $\mu$ and standard
  deviation $\sigma.$

  \vfill

  We have a collection of measurements:
  \begin{eqnarray*}
    x_1,~x_2,~x_3,\cdots,~x_n.
  \end{eqnarray*}
  Each measurement has a mean, $\mu$, and a standard deviation of
  $\sigma$. We assume that they are all independent of one another.

  \vfill

\end{frame}

\begin{frame}
  \frametitle{Distribution of the Sample Mean}

  We have a collection of measurements:
  \begin{eqnarray*}
    x_1,~x_2,~x_3,\cdots,~x_n.
  \end{eqnarray*}

  \vfill
  
  Given the measurements the sample mean is
  \begin{eqnarray*}
    \bar{x} & = & \frac{x_1+x_2+x_3+\cdots+x_n}{n}.
  \end{eqnarray*}
  The sample mean, $\bar{x}$,  has a mean of $\mu$, and it has a
  standard deviation of $\frac{\sigma}{\sqrt{n}}$.

  \vfill

\end{frame}



\subsection{Examples}

\begin{frame}
  \frametitle{Example}

  The life time for a brand of batteries is normally distributed with
  a mean of 19.8 hours and a standard deviation of 1.3 hours. What is
  the probability that a sample size of six batteries will give a
  sample mean that is more than 21.1 hours?

  \vfill

  \only<2->{%
    First, organize the information: \\
    \begin{tabular}{l|l}
      Random variable & \action<3-|alert@3>{\color{red}Sample Mean} \\ 
      $\mu=19.8$   & \action<3-|alert@3>{\color{red}$\mu=19.8$} \\
      $\sigma=1.3$ & \action<3-|alert@3>{\color{red}$\sigma=\frac{1.3}{\sqrt{6}}$} \\
                   & \action<3-|alert@3>{\color{red}$n=6$}
    \end{tabular}
  }

  \vfill

  \note{
    \begin{itemize}
    \item Ans: 0.0071
    \item First organize the information.
    \item Make a rough sketch.
    \item Figure out the question that is asked.
    \item Figure out how to answer the question.
    \item Finally, carry out your plan.
    \end{itemize}
  }
\end{frame}


\begin{frame}
  \frametitle{Example}

  You want to estimate the start up costs for restaurants in an
  area. If the mean start up costs are \$235,000 with a standard
  deviation of \$45,000 what is the probability that a sample of
  twenty restaurants will be between \$260,000 and \$205,000?

  \vfill

  \only<2->{%
    First, organize the information: \\
    \begin{tabular}{l|l}
      Random variable & \action<3-|alert@3>{\color{red}Sample Mean} \\ 
      $\mu=\$235,000$   & \action<3-|alert@3>{\color{red}$\mu=\$235,000$} \\
      $\sigma=\$45,000$ & \action<3-|alert@3>{\color{red}$\sigma=\frac{\$45,000}{\sqrt{20}}$} \\
                   & \action<3-|alert@3>{\color{red}$n=20$}
    \end{tabular}
  }

  \vfill

  \note{
    \begin{itemize}
    \item Ans: 0.9920
    \item First organize the information.
    \item Make a rough sketch.
    \item Figure out the question that is asked.
    \item Figure out how to answer the question.
    \item Finally, carry out your plan.
    \end{itemize}
  }
\end{frame}


\begin{frame}
  \frametitle{Example}

  You want to estimate the start up costs for restaurants in an
  area. If the mean start up costs are \$235,000 with a standard
  deviation of \$45,000 how many restaurants should you poll so that
  the probability to be within \$1,000 of the mean is eighty percent?

  \vfill

  \only<2->{%
    Organize the information: \\
    \begin{tabular}{l|l}
      Random variable & \action<3-|alert@3>{\color{red}Sample Mean} \\ 
      $\mu=\$235,000$   & \action<3-|alert@3>{\color{red}$\mu=\$235,000$} \\
      $\sigma=\$45,000$ & \action<3-|alert@3>{\color{red}$\sigma=\frac{\$45,000}{\sqrt{N}}$} \\
                   & \action<3-|alert@3>{\color{red}$n=?$}
    \end{tabular}
  }

  \vfill

  \note{
    \begin{itemize}
    \item $N=3318$
    \item Note that we rounded up!
    \item First organize the information.
    \item Make a rough sketch.
    \item Figure out the question that is asked.
    \item Figure out how to answer the question.
    \item Finally, carry out your plan.
    \end{itemize}
  }

\end{frame}


\iftoggle{clicker}{%
  \begin{frame}
    \frametitle{Clicker Quiz}

    You want to estimate the start up costs for restaurants in an
    area. The mean start up costs are \$235,000 with a standard
    deviation of \$45,000. What is the probability that twenty samples
    will give a sample mean greater than \$245,000?


    \vfill

    \begin{tabular}{l@{\hspace{3em}}l@{\hspace{3em}}l@{\hspace{3em}}l}
      A: 0.1611  & B: 0.4021  & C: 0.4121 & D: 0.9999
    \end{tabular}

    \vfill
    \vfill
    \vfill

  \end{frame}
}



% LocalWords:  Clarkson pausesection hideallsubsections hideothersubsections
% LocalWords:  sectionstyle
