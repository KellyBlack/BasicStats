\documentclass[svgnames,table]{beamer}
%\documentclass[svgnames,table,handout,aspectratio=129]{beamer}
\usepackage{hhline}
\usepackage{etoolbox}
\usepackage{tikz}
\usepackage{mathtools}
\usepackage{amssymb}
%\usepackage{/usr/lib64/R/share/texmf/Sweave}


% %%%%%%%%%%%%%%%%%%%%%%%%%%%%%%%%%%%%%%%%%%%%%%%%%%%%%%%%%%%%%%%%%%%%%%%
% List of definitions that are used in the different pages for the
% notes


% %%%%%%%%%%%%%%%%%%%%%%%%%%%%%%%%%%%%%%%%%%%%%%%%%%%%%%%%%%%%%%%%%%%%%%%
% Basic definitions used throughout the notes

\newcommand{\half}{\mbox{$\frac{1}{2}$}}
\newcommand{\deltat}{\mbox{$\triangle t$}}
\newcommand{\deltax}{\mbox{$\triangle x$}}
\newcommand{\deltay}{\mbox{$\triangle y$}}

\newcommand{\deriv}[2]{\frac{d}{d#2}#1}
\newcommand{\derivTwo}[2]{\frac{d^2}{d#2^2}#1}

\newcommand{\lp}{\left(}
\newcommand{\rp}{\right)}


% %%%%%%%%%%%%%%%%%%%%%%%%%%%%%%%%%%%%%%%%%%%%%%%%%%%%%%%%%%%%%%%%%%%%%%
% Basic color additions
\definecolor{fuchsia}{RGB}{255,0.0,255}
\definecolor{garnet}{RGB}{136,0,0}
%\definecolor{clarksonGreen}{RGB}{0,71,28}
\definecolor{light-gray}{gray}{0.8}
\definecolor{clarksonGreen}{RGB}{0,52,21}


\newcommand{\redText}[1]{{\color{red}#1}}
\newcommand{\blueText}[1]{{\color{blue}#1}}
\newcommand{\greenText}[1]{{\color{green}#1}}
\newcommand{\fuchsiaText}[1]{{\color{fuchsia}#1}}



%%% Local Variables:
%%% mode: latex
%%% TeX-master: "IntroStats"
%%% End:



%\usetheme{Frankfurt}%
\usetheme{Warsaw}%
%\useoutertheme{smoothbars}

%\usecolortheme{seagull}
\usecolortheme{beaver}


\begin{document}

\begin{frame}
  \frametitle{Clicker Quiz}

  You want to monitor the travel costs for sales people in remote
  offices. It is estimated that the mean travel cost per person is
  \$5,500 per person per month with a standard deviation of
  \$1,200. Each week you will sample fifteen sales people at random
  and calculate a sample mean.

  \vfill

  What is the upper control limit using a $3\sigma$ cut-off?

  \vfill

  \begin{tabular}{l@{\hspace{3em}}l@{\hspace{3em}}l}
    A: \$5,809.80 & B: \$6430.50 & C: \$6,700.00
  \end{tabular}

  \vfill
  \vfill
  \vfill
  
\end{frame}

\begin{frame}
  \frametitle{Clicker Quiz}

  You are monitoring the travel costs for sales people in remote
  offices. It was estimated that the mean travel cost per person is
  \$5,500 per person per month with a standard deviation of \$1,200.
  Each week you sample fifteen sales people at random and calculate a
  sample mean. Due to increased fuel costs you suspect that the mean
  has shifted to be \$5,750. 

  \vfill

  What is the probability that this increase will be detected?

  \vfill

  \begin{tabular}{l@{\hspace{3em}}l@{\hspace{3em}}l}
    A: 0.014 & B: 0.050 & C: 0.45
  \end{tabular}

  \vfill
  \vfill
  \vfill
  
\end{frame}


\begin{frame}
  \frametitle{Clicker Quiz}

  You want to monitor the travel costs for sales people in remote
  offices. It is estimated that the mean travel cost per person is
  \$5,500 per person per month with a standard deviation of
  \$1,200. Each week you want to sample a set of sales people at
  random and calculate a sample mean.

  \vfill

  How many should you sample if the upper control limit using a
  $3\sigma$ cut-off will be \$5,700 and the lower control limit will
  be \$5,300.

  \vfill

  \begin{tabular}{l@{\hspace{3em}}l@{\hspace{3em}}l}
    A: 4 & B: 36 & C: 324
  \end{tabular}

  \vfill
  \vfill
  \vfill
  
\end{frame}


\begin{frame}
  \frametitle{Clicker Quiz}

  You want to monitor the travel costs for sales people in remote
  offices. It is estimated that the mean travel cost per person is
  \$5,500 per person per month with a standard deviation of
  \$1,200. Each week you want to sample a set of sales people at
  random and calculate a sample mean.

  \vfill

  How many should you sample if the upper control limit using a
  $3\sigma$ cut-off will be \$5,600 and the lower control limit will
  be \$5,400?

  \vfill

  \begin{tabular}{l@{\hspace{3em}}l@{\hspace{3em}}l}
    A: 324 & B: 1296 & C: 5184
  \end{tabular}

  \vfill
  \vfill
  \vfill
  
\end{frame}


\begin{frame}
  \frametitle{Clicker Quiz}

  You want to monitor the travel costs for sales people in remote
  offices. It is estimated that the travel costs per person is
  normally distributed with a mean of \$5,500 per person per month
  with a standard deviation of \$1,200. Each week you will sample
  fifty sales people at random. You will then determine the proportion
  of people whose costs are above \$5,800.

  \vfill

  What is the upper control limit using a $3\sigma$ cut-off?

  \vfill

  \begin{tabular}{l@{\hspace{3em}}l@{\hspace{3em}}l}
    A: 0.61  & B: 0.82  & C: 1.87
  \end{tabular}

  \vfill
  \vfill
  \vfill
  
\end{frame}


\end{document}


%%% Local Variables:
%%% mode: latex
%%% TeX-master: t
%%% End:
