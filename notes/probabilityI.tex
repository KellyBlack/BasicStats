
\lecture{Introduction to Probability}{intro-to-probability}
\section{Introduction to Probability}

\title{Probability}
\subtitle{What are the chances?}

%\author{Kelly Black}
%\institute{Clarkson University}
\date{12 January 2015}

\begin{frame}
  \titlepage
\end{frame}

\begin{frame}
  \frametitle{Outline}
  \tableofcontents[hideothersubsections,sectionstyle=show/hide]
\end{frame}


\subsection{Clicker Quiz}


\iftoggle{clicker}{%

  \begin{frame}
    \frametitle{Clicker Quiz}

    If I flip a fair coin ten times how many tails will I get?

    \begin{tabular}{l@{\hspace{3em}}l@{\hspace{3em}}l@{\hspace{3em}}l}
      A: 4 & B: 5 & C: 6 & D: I do not know
    \end{tabular}


  \end{frame}

}




\subsection{Example}

%\begin{frame}
%  \frametitle{Clicker Exercise}
%
%  Flip a coin 3 times. How many tails did you get?
%
%  \begin{tabular}{l@{\hspace{3em}}l@{\hspace{3em}}l@{\hspace{3em}}l}
%  A: 0 & B: 1 & C: 2 & D: 3
%  \end{tabular}
%
%  \only<2->%
%  {
%    Question: What is the probability that we get the same result if
%    we do it again?
%  }
%
%\end{frame}

\begin{frame}{Recall}

  Recall from last time:

  \begin{definition}[Probability]
    ``\textbf{Probability} is the measure of the likelihood of a random
    phenomena or chance behavior. Probability describes the long-term
    proportion with which a certain \textbf{outcome} will occur in
    situations with short-term uncertainty.'' (Page 223)
  \end{definition}

  
\end{frame}


\begin{frame}{Also Recall}

  \begin{description}
  \item[Event:] Something that \textit{\color{red}can} happen.
  \item[Outcome:] Something that \textit{\color{red}did} happen.
  \item[Experiment:] A structured activity that includes the
    measurement of the outcomes of the activity after predefined and
    {\color{blue}intentional changes to some aspect of the events}.
  \item[Observational Study:] A structured activity that includes the
    measurement of the outcomes of the activity {\color{blue}without
      intentional changes to the events}.
  \item[Expected Outcome:] The ``average'' of the possible outcomes.
  \item[{\color{red}Variation}:] {\color{red}Some measure of the
      spread of possible outcomes.}
  \end{description}
  
\end{frame}

\begin{frame}{Also Also Recall}

  \textbf{\color{blue}Probability} is an idealized notion. If we
    repeat the experiment an infinite number of times we ask what
    \textit{\color{red}might} happen.

  \vfill

  \textbf{\color{blue}Statistics} is the study of how to interpret
  data. We ask what \textit{\color{red}did} happen and what does it
  imply about the underlying probabilities?

  \vfill

\end{frame}


\subsection{Events}

\begin{frame}
  \frametitle{Definitions}

  \begin{definition}
    An \textbf{event} is a possible outcome from an experiment.
  \end{definition}

  \begin{definition}
    The \textbf{sample space} is the collection of all possible
    events.
  \end{definition}

\end{frame}


\begin{frame}
  \frametitle{Sample Space}

  There are two ways to visualize the sample space:
  \begin{itemize}
  \item Venn Diagram
  \item Tree Diagram
  \end{itemize}

\end{frame}

\subsection{Examples}

\begin{frame}{Example}

  We flip a coin two times. What are the events?

  \only<2-4>%
  {

    \begin{picture}(250,120)
      \put(5,90){HH}
      \put(5,75){HT}
      \put(5,60){TH}
      \put(5,45){TT}

      \put(1,10){All}
      \put(1,0){Events}


      \uncover<3->{%
        \put( 55,110){\line(50,0){75}}
        \put( 55, 35){\line(50,0){75}}
        \put( 55,72.5){\line(50,0){75}}
        \put( 55,110){\line(0,-50){75}}
        \put( 95,110){\line(0,-50){75}}
        \put(130,110){\line(0,-50){75}}
        \put( 71,85){HH}
        \put(103,85){HT}
        \put( 71,50){TH}
        \put(103,50){TT}

        \put(60,10){Venn Diagram}
      }

      \uncover<4->{%
        \put(155,72.5){\circle*{5}}
        \put(155,72.5){\line(1,1){20}}
        \put(155,72.5){\line(1,-1){20}}
        \put(195,95){\circle*{5}}
        \put(195,95){\line(1,1){20}}
        \put(195,95){\line(1,-1){20}}
        \put(180,92){H}
        \put(195,45){\circle*{5}}
        \put(195,45){\line(1,1){20}}
        \put(195,45){\line(1,-1){20}}
        \put(180,42){T}
        \put(217,112){H}
        \put(217,72){T}
        \put(217,59){H}
        \put(217,23){T}

        \put(165,10){Tree Diagram}
      }

    \end{picture}

  }

  \only<5->%
  {

    \begin{eqnarray*}
      p(\mathrm{first~T}) & = & ? \\
      p(TT) & = & ? \\
      p(TH) & = & ? \\
      p(\mathrm{second~H}) & = & ? \\
      p(\mathrm{one~T}) & = & ? \\
      p(\mathrm{two~H}) & = & ?
    \end{eqnarray*}

  }
  
\end{frame}

\begin{frame}{Clicker Quiz}
  A couple has a child. Two years later they have another child. What
  is the probability that they have one girl and one boy?

  \begin{tabular}{l@{\hspace{3em}}l@{\hspace{3em}}l@{\hspace{3em}}l}
    A: 0 & B: 1/4 & C: 1/2 & D: 1
  \end{tabular}


\end{frame}

\begin{frame}{Example}

  I roll a fair, six sided die twice. What is $p(\mathrm{sum}=5)$?
  
\end{frame}


\subsection{Properties of Probabilities}

\begin{frame}{Properties of Probabilities}

  Suppose that A is an event in the sample space, then
  \begin{eqnarray*}
    \begin{array}{rcccl}
      0 & \leq & P(A) & \leq & 1
    \end{array}
  \end{eqnarray*}

  Also note that
  \begin{eqnarray*}
    p(\mathrm{Sample~Space}) & = & 1.
  \end{eqnarray*}

  The notation matters! It is easy to get sloppy, but it makes it
  harder to understand what is going on if you do.
  
\end{frame}


\begin{frame}{}

  It is estimated that one in every three-hundred and fifty-two
  properties in Miami has been foreclosed by a
  bank\footnote{\url{http://www.realtytrac.com/statsandtrends/foreclosuretrends/fl/miami-dade-county}}. Three
  properties in Miami are chosen at random. What is the probability
  that one is a foreclosed property?

  \vfill
  
\end{frame}


% LocalWords:  Clarkson pausesection hideallsubsections
