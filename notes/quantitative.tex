
\lecture{Introduction to Bivariate Data}{introduction-to-bivariate-data}
\section{Introduction to Bivariate Data}

\title{Quantitative Data}
\subtitle{Methods to Assess Numerical Data}

%\author{Kelly Black}
%\institute{Clarkson University}
\date{4 February 2015}

\begin{frame}
  \titlepage
\end{frame}

\begin{frame}
  \frametitle{Outline}
  \tableofcontents[hideothersubsections,sectionstyle=show/hide]
\end{frame}


\subsection{Clicker Quiz}


\iftoggle{clicker}{%
  \begin{frame}
    \frametitle{Clicker Quiz}

    A normally distributed random variable has a mean of 1.2. The
    probability that it is less than 0.8 is 0.25. What is the standard
    deviation?
    
    \vfill

    \begin{tabular}{l@{\hspace{3em}}l@{\hspace{3em}}l@{\hspace{3em}}l}
      A: -0.67 & B: -0.40  & C: 0.60 & D: 0.67
    \end{tabular}

    \vfill

  \end{frame}
}


\subsection{Quantitative Data}

\begin{frame}{Quantitative Data}
  
  \begin{definition}{Discrete Data}

    A discrete random variable produces a countable number of
    values. The values are one of a fixed set of values. 

    \textit{{\color{red}Here the random variables return numbers.}}
    
  \end{definition}

  \begin{definition}{Continuous Data}

    The numbers produced come from a continuous range of values.
    
  \end{definition}

\end{frame}

\begin{frame}{Discrete Data Examples}

  \begin{itemize}
  \item Count the number of people who arrive in a line during a given
    time period.

  \item We report the number of bottles of a brand of soft drink that
    are sold in a given time period.

  \item We report the number of defects found in a mirror randomly
    chosen from a factory floor.

  \end{itemize}
  
\end{frame}


\begin{frame}{Quantitative Data}
  
  \begin{definition}[Univariate Data]

    Each observation consists of a single measurement, and there are
    no explicit relationships between them.

    \textit{{\color{red}All of the examples on the previous slide
        represent univariate data.}}
    
  \end{definition}

  \begin{definition}[Bivariate Data]

    Each observation consists of a pair of measurements.
    
  \end{definition}

  \begin{definition}[Multivariate Data]

    Each observation consists of more than two measurements. There is
    more than two measurements per group.
    
  \end{definition}


\end{frame}



\subsection{Bivariate Data}

\begin{frame}
  \frametitle{Bivariate Data}

  \begin{eqnarray*}
    \begin{array}{l|l@{\hspace{2em}}r}
      x      & y \\ \cline{1-2}
      x_1    & y_1 \\
      \only<1>{%
      x_2    & y_2 \\
      }
      \only<2>{%
      \color{red}{x_2} & \color{red}{y_2} \\
      }
      \only<3>{%
      \color{red}{x_2} & \color{red}{y_2} & \leftarrow \mathrm{~one~set~of~observations} \\
      }
      x_3    & y_3 \\
      \vdots & \vdots \\
      x_n    & y_n
    \end{array}
  \end{eqnarray*}
\end{frame}



\iftoggle{clicker}{%
  \begin{frame}
    \frametitle{Clicker Quiz}

    An experiment is run. Twenty different two hour periods are chosen
    at random. At the end of each time period the start time and the
    number of cars that stop at the sign are recorded.

    What kind of data is this?

    \vfill

    \begin{tabular}{l@{\hspace{3em}}l@{\hspace{3em}}l@{\hspace{3em}}l}
      A: univariate, discrete & B: univariate, continuous  \\
      C: bivariate, discrete  & D: bivariate, continuous
    \end{tabular}

    \vfill



  \end{frame}
}

\subsection{Discrete Quantitative Data}

\begin{frame}
  \frametitle{Discrete Quantitative Data}

  \begin{eqnarray*}
    5,~5,~3,~5,~4,~3,~4,~1,~6,~3,~2,~1
  \end{eqnarray*}

  This is no different from what we did with qualitative data in our
  last session. Just count the number of occurrences and work with
  frequencies.


  \only<2->%
  {
    \begin{tabular}{l|l}
      Value  & Frequency \\ \hline
      1 & 2 \\
      2 & 1 \\
      3 & 3 \\
      4 & 2 \\
      5 & 3 \\
      6 & 1 
    \end{tabular}
  }

  \only<3->%
  {
    \textit{relative frequency, blah blah blah}
  }

\end{frame}

\subsection{Continuous Data}

\begin{frame}
  \frametitle{Continuous Data}

  \begin{itemize}
  \item Every day I measure the amount of precipitation over the
    previous 24 hours.

  \item I observe the price of a particular stock at the end of each day.
    \uncover<2->{\textit{\color{red}Well... technically this is discrete, but we
        pretend it is continuous.}}

  \item I measure the blood alcohol levels for people who have been
    pulled over each night in Potsdam, NY.
  \end{itemize}
\end{frame}

\subsection{Graphical Techniques}

\begin{frame}{Graphical Techniques}

  \begin{itemize}
  \item Bar Plots
  \item Dot Plots
  \item Stem-Leaf Charts
  \item histograms
  \end{itemize}
  
\end{frame}


\begin{frame}{Bar Plots}

  \begin{eqnarray*}
    5,~5,~3,~5,~4,~3,~4,~1,~6,~3,~2,~1
  \end{eqnarray*}


  \begin{columns}
    \column{.25\textwidth}


    \begin{tabular}{l|l}
      Number  & Freq. \\ \hline
      1 & 2 \\
      2 & 1 \\
      3 & 3 \\
      4 & 2 \\
      5 & 3 \\
      6 & 1 
    \end{tabular}

    \column{.75\textwidth}

    \includegraphics[width=7cm]{img/barplotExample}


  \end{columns}
  
  
\end{frame}


\begin{frame}{Dot Charts}

  \begin{eqnarray*}
    5,~5,~3,~5,~4,~3,~4,~1,~6,~3,~2,~1
  \end{eqnarray*}


  \begin{columns}
    \column{.25\textwidth}


    \begin{tabular}{l|l}
      Number  & Frequency \\ \hline
      1 & 2 \\
      2 & 1 \\
      3 & 3 \\
      4 & 2 \\
      5 & 3 \\
      6 & 1 
    \end{tabular}

    \column{.75\textwidth}

    \only<1>{\includegraphics[width=7cm]{img/stripChart}}
    \only<2>{\includegraphics[width=7cm]{img/dotChart}}


  \end{columns}
  
  
\end{frame}


\begin{frame}{Stem-Leaf Plot}

  We can divide a number between its ``stem'' and ``leaf.''

  For example:
  \only<1>%
  {
    \begin{eqnarray*}
      17
    \end{eqnarray*}
  }
  \only<2>%
  {
    \begin{eqnarray*}
      \underbrace{\color{red}1}_{\mathrm{stem}}\underbrace{\color{blue}7}_{\mathrm{leaf}}
    \end{eqnarray*}
  }

  \only<3>%
  {
    Given a group of numbers we can organize the numbers with the same
    stem in the same rows. Then list each leaf in ascending order. The
    result is a ``stem-leaf plot.''
  }
  
  
\end{frame}

\begin{frame}{Example}

  \only<1>%
  {
    \begin{eqnarray*}
      47,~44,~29,~27,~29,~36,~24,~21,~36,~34
    \end{eqnarray*}
  }
  \only<2>%
  {
    \noindent
    {\color{red}{Identify the stems}}:
    \begin{eqnarray*}
      {\color{red}4}7,~{\color{red}4}4,~{\color{red}2}9,~{\color{red}2}7,
      ~{\color{red}2}9,~{\color{red}3}6,~{\color{red}2}4,~{\color{red}2}1,
      ~{\color{red}3}6,~{\color{red}3}4
    \end{eqnarray*}
  }\
  \only<3->%
  {
    \noindent
    {\color{red}{Identify the stems}} and 
    {\color{blue}{identify the leaves}}:
    \begin{eqnarray*}
      {\color{red}4}{\color{blue}7},~{\color{red}4}{\color{blue}4},
      ~{\color{red}2}{\color{blue}9},~{\color{red}2}{\color{blue}7},
      ~{\color{red}2}{\color{blue}9},~{\color{red}3}{\color{blue}6},
      ~{\color{red}2}{\color{blue}4},~{\color{red}2}{\color{blue}1},
      ~{\color{red}3}{\color{blue}6},~{\color{red}3}{\color{blue}4}
    \end{eqnarray*}
  }

  \only<4>%
  {
    Write out the unique stems in order: \\
    \begin{tabular}{l@{\hspace{1em}}|@{\hspace{1em}}l}
      {\color{red}2} & \\
      {\color{red}3} & \\
      {\color{red}4} & 
    \end{tabular}
  }

  \only<5>%
  {
    Write out every leaf in ascending order: \\
    \begin{tabular}{l@{\hspace{1em}}|@{\hspace{1em}}l}
      {\color{red}2} & {\color{blue}14799}\\
      {\color{red}3} & {\color{blue}466}\\
      {\color{red}4} & {\color{blue}47}
    \end{tabular}
  }

  
\end{frame}

\begin{frame}{Histograms}

  This is a generalization of the stem-leaf plot. Instead of dividing
  things in groups by the ``tens'' we instead use an arbitrary range
  of values. We calculate how many data points fall into the
  pre-determined ranges and then make a bar plot of the frequencies.
  
\end{frame}

\begin{frame}{Example}

  \begin{eqnarray*}
    2.7,~3.6,~3.8,~2.7,~2.7,~3.3,~3.3,~3.6,~3.0,~3.2
  \end{eqnarray*}


  \begin{columns}
    \column{.25\textwidth}

    \only<2>%
    {
      We (arbitrarily) break things up into a range of values: \\
      \begin{tabular}{l|l}
        2.6-2.8 \\
        2.8-3.0 \\
        3.0-3.2 \\
        3.2-3.4 \\
        3.4-3.6 \\
        3.6-3.8
      \end{tabular}
    }

    \only<3->%
    {
      We (arbitrarily) break things up into a range of values: \\
      \begin{tabular}{l|l}
        2.6-2.8 & III\\
        2.8-3.0 & I \\
        3.0-3.2 & I \\
        3.2-3.4 & II \\
        3.4-3.6 & II \\
        3.6-3.8 & I
      \end{tabular}
    }
    
    \column{.75\textwidth}

    \only<4>{\includegraphics[width=6cm]{img/histogramExample}}

  \end{columns}

  
\end{frame}

\begin{frame}{Histograms}

  \begin{itemize}
  \item Overall shape 
  \item Skew
  \item Center (balance point, middle of area)
  \item Spread
  \end{itemize}
  
\end{frame}

% LocalWords:  Clarkson pausesection hideallsubsections Bivariate
