
\lecture{Confidence Intervals for the Sample Proportion}{confidence-interval-sample-proportions}
\section{Confidence Intervals for the Sample Proportion}

\title{Confidence Intervals for the Sample Proportion}
\subtitle{What percentage?}

%\author{Kelly Black}
%\institute{Clarkson University}
\date{13 March 2013}

\begin{frame}
  \titlepage
\end{frame}

\begin{frame}
  \frametitle{Outline}
  \tableofcontents[hideothersubsections,sectionstyle=show/hide]
\end{frame}


\subsection{Clicker Quiz}


\iftoggle{clicker}{%
  \begin{frame}
    \frametitle{Clicker Quiz}

    Requests are made to determine the cost of including advertisements
    on web pages.  Twenty-three responses are received, and the sample
    mean is 1,300\$ with a sample standard deviation of 320\$. What is
    the 95\% confidence interval for the mean?

    \vfill

    \begin{tabular}{l@{\hspace{3em}}l@{\hspace{3em}}l@{\hspace{3em}}l}
      A: 1169\$ to 1431\$ & B: 1190\$ to 1410\$ \\
      C: 1162\$ to 1438\$ & D: 1185\$ to 1415\$
    \end{tabular}

    \vfill
    \vfill
    \vfill

  \end{frame}
}


\subsection{Sample Proportion}

\begin{frame}{The Sample Proportion}

  You conduct $n$ trials. Each trial has a probability $p$ for
  ``success'' and a probability of $1-p$ for a different
  result. Define the sample proportion to be 
  \begin{eqnarray*}
    \hat{p} & = & \frac{\mathrm{number~of~successes}}{\mathrm{number~of~trials}}.
  \end{eqnarray*}

  \begin{itemize}
  \item The sample proportion has a mean of $p$, and a standard
    deviation of $\sqrt{\frac{p(1-p)}{n}}$.
  \item Check to see if $p\cdot n \geq 5$, $(1-p)\cdot n \geq 5$ and
    $n\geq 20$. If \textit{all} of these things are true
    \begin{itemize}
    \item $\hat{p}$ is approximately normally distributed
    \item The mean of $\hat{p}$ is mean $p$.
    \item The standard deviation of $\hat{p}$ is
      $\sqrt{\frac{p(1-p)}{n}}$
    \item A $z$-statistic can be approximated using
      \begin{eqnarray*}
        z & = & \frac{\hat{p}-p}{\sqrt{\frac{p(1-p)}{n}}}.
      \end{eqnarray*}
    \end{itemize}

  \end{itemize}



  
\end{frame}


\begin{frame}
  \frametitle{Notation}


  \begin{definition}[The Sample Proportion]

    You conduct an experiment. There are $N$ trials. The number of
    ``successes'' is $x$.  The sample proportion is defined to be 
    \begin{eqnarray*}
      \hat{p} & = & \frac{x}{N}.
    \end{eqnarray*}

    \only<2->%
    {

      If $Np\geq 5$ and $N(1-p)\geq 5$ and $N\geq 20$ then $\hat{p}$
      can be approximated by a normal distribution with mean $p$ and
      standard deviation $\sqrt{p(1-p)/N}$.

    }

    \only<3->%
    {

      Once we have the distribution then we can do all the stuff we do
      with normal distributions. (i.e. We can find confidence intervals!)

    }

  \end{definition}
  
\end{frame}

\subsection{Examples}

\begin{frame}
  \frametitle{Example}

  I conduct a poll to determine what proportion of people are
  satisfied with the company's benefits program. Forty-five people
  respond and thirty-one people say they are satisfied.  What is the
  95\% confidence interval for the proportion of people satisfied with
  their benefits?

  \only<2->%
  {

    In this case our estimate for $\hat{p}$ is 
    \begin{eqnarray*}
      \hat{p} & = & \frac{31}{45}.
    \end{eqnarray*}
    We use this as our estimate for $p$.

  }

  \only<3->%
  {

    {\color{red}
      The 95\% confidence interval is from 0.554 to 0.824 assuming a
      normal approximation with 45 samples.
    }

  }
  
\end{frame}


\begin{frame}
  \frametitle{Example}

  We want to determine how many people in a company have a commute
  that takes more than one hour one-way. One-hundred and twenty people
  respond, and forty-two say that their commute is more than one
  hour. What is the 95\% confidence interval for the proportion of
  people whose commute is more than one hour?

  \only<2->%
  {

    In this case our estimate for $\hat{p}$ is 
    \begin{eqnarray*}
      \hat{p} & = & \frac{42}{120}.
    \end{eqnarray*}
    We use this as our estimate for $p$.

  }

  \only<3->%
  {

    {\color{red}
      The 95\% confidence interval is from 0.265 to 0.435 assuming a
      normal approximation with 120 samples.
    }

  }
  
\end{frame}


\iftoggle{clicker}{%
  \begin{frame}
    \frametitle{Clicker Quiz}

    We want to determine the percentage of stocks in a particular
    market whose price rises on a given day. A sample of fifty-seven
    stocks is made, and the price for eighteen rise.  What is the 95\%
    confidence interval for the proportion of stocks whose price
    increased that day?

    \vfill

    \begin{tabular}{l@{\hspace{3em}}l@{\hspace{3em}}l@{\hspace{3em}}l}
      A: .195 to .437 & B: .214 to .418 
    \end{tabular}

    \vfill
    \vfill
    \vfill

  \end{frame}
}

\begin{frame}
  \frametitle{Example}

    We want to determine the percentage of stocks in a particular
    market whose price rises on a given day. We want the error in the
    confidence interval to be $\pm 1.5\%$ for the 95\% confidence
    interval. How many stocks should we sample?

    \only<2->%
    {

      We do not know $p$ and assume the value of $p$ that will make
      the standard deviation as large as possible. Since
      \begin{eqnarray*}
        \sigma_{\hat{p}} & = & \sqrt{\frac{p(1-p)}{N}},
      \end{eqnarray*}
      then this is largest when $p=\half$.

    }

  \only<3->%
  {

    statement here.

  }

  
\end{frame}




%%% Local Variables: 
%%% mode: latex
%%% TeX-master: "IntroStats"
%%% End: 

% LocalWords:  hideothersubsections sectionstyle pausesection
