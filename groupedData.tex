
\lecture{Grouped Data}{grouped-data}
\section{Grouped Data}


\title{Grouped Data}
\subtitle{Working with Grouped Data}

%\author{Kelly Black}
%\institute{Clarkson University}
\date{18 February 2013}

\begin{frame}
  \titlepage
\end{frame}

\begin{frame}
  \frametitle{Outline}
  \tableofcontents[pausesection,hideothersubsections,sectionstyle=show/hide]
\end{frame}


\iftoggle{clicker}{%
  \subsection{Clicker Quiz}


  \begin{frame}
    \frametitle{Clicker Quiz}

    \vfill
    Find the median of the following data set:\\
    \begin{tabular}{llllll}
      28, & 17, & 12, & 20, & 21, & 18
    \end{tabular}

    \vfill
    
    \begin{tabular}{l@{\hspace{3em}}l@{\hspace{3em}}ll@{\hspace{3em}}l}
      A: 17 & B: 18 & C: 19 & D: 20
    \end{tabular}

    \vfill

  \end{frame}

}



\subsection{Grouped Data - Discrete Data}

\begin{frame}{Grouped Data}

  \begin{itemize}
  \item Often times you do not get raw data.

  \item What do you do when you have to react to data in a report or a
    publication?

  \item Here we assume that we are given frequencies from summaries of
    data. We want to know how to estimate the center and spread of the
    data.
  \end{itemize}
\end{frame}

\begin{frame}{Frequency Tables - Discrete Data}

  Suppose we have this data:
  \begin{eqnarray*}
    5,~5,~3,~5,~4,~3,~4,~1,~6,~3,~2,~1
  \end{eqnarray*}

  We can count the number of occurrences and work with frequencies.

  \only<2->%
  {
    \begin{tabular}{l|l}
      Number  & Frequency \\ \hline
      1 & 2 \\
      2 & 1 \\
      3 & 3 \\
      4 & 2 \\
      5 & 3 \\
      6 & 1 
    \end{tabular}
  }

  \only<3->%
  {
    What if we only have the table? How do we calculate the sample
    mean?
  }

  
\end{frame}


\begin{frame}{Frequency Tables - Discrete Data}

  Suppose we have this data: \\
    \begin{tabular}{l|l}
      Number  & Frequency \\ \hline
      1 & {\color{red}2} \\
      2 & {\color{red}1} \\
      3 & {\color{red}3} \\
      4 & {\color{red}2} \\
      5 & {\color{red}3} \\
      6 & {\color{red}1}
    \end{tabular}

    \only<2->%
    {

      There are {\color{red}2} ones, {\color{red}1} two,
      {\color{red}3} threes, {\color{red}2} fours, {\color{red}3}
      fives, and {\color{red}1}
      six. The sum of the numbers is  \\
      \begin{center}
        {\color{red}2}*one + {\color{red}1}*two + {\color{red}3}*three
        + {\color{red}2}*four + {\color{red}3}*five + {\color{red}1}*six.
      \end{center}

      \only<3->%
      {
        The sum of the data is 42.
      }
    }
  
\end{frame}


\begin{frame}{Frequency Tables - Discrete Data}

  Suppose we have this data: \\
    \begin{tabular}{l|l}
      Number  & Frequency \\ \hline
      1 & {\color{red}2} \\
      2 & {\color{red}1} \\
      3 & {\color{red}3} \\
      4 & {\color{red}2} \\
      5 & {\color{red}3} \\
      6 & {\color{red}1}
    \end{tabular}

    We now need to divide by the total number of data points.

    \only<2->%
    {

      There are {\color{red}2} ones, {\color{red}1} two,
      {\color{red}3} threes, {\color{red}2} fours, {\color{red}3}
      fives, and {\color{red}1}
      six. The total number of data points is  \\
      \begin{center}
        {\color{red}2} + {\color{red}1} + {\color{red}3}
        + {\color{red}2} + {\color{red}3} + {\color{red}1}.
      \end{center}

      \only<3->%
      {
        There are 12 data points.
      }
    }
  
\end{frame}


\begin{frame}{Frequency Tables - Discrete Data}

  Suppose we have this data: \\
    \begin{tabular}{l|l}
      Number  & Frequency \\ \hline
      1 & {\color{red}2} \\
      2 & {\color{red}1} \\
      3 & {\color{red}3} \\
      4 & {\color{red}2} \\
      5 & {\color{red}3} \\
      6 & {\color{red}1}
    \end{tabular}

    The sample mean is the sum of the data points divided by the total
    number of data points.

    \only<2->%
    {
      \begin{eqnarray*}
        \bar{x} & = & \frac{42}{12}.
      \end{eqnarray*}
    }
  
\end{frame}


\begin{frame}{Frequency Tables - Discrete Data}

  In general we have the following data:
  \begin{eqnarray*}
    \begin{array}{l|l}
      \mathrm{Number}  & \mathrm{Frequency} \\ \hline
      x_1 & {\color{red}f_1} \\
      x_2 & {\color{red}f_2} \\
      \vdots & \vdots \\
      x_N & {\color{red}f_n}
    \end{array}
  \end{eqnarray*}

  The sum of the data is 
  \begin{eqnarray*}
    x_1\cdot {\color{red}f_1} + x_2 \cdot {\color{red}f_2} + x_3 \cdot {\color{red}f_3} + 
    \cdots + x_N \cdot {\color{red}f_N}.
  \end{eqnarray*}
  The total number of data points is
  \begin{eqnarray*}
    {\color{red}f_1} + {\color{red}f_2} + {\color{red}f_3} + 
    \cdots + {\color{red}f_N}.
  \end{eqnarray*}
  The sample mean is
  \begin{eqnarray*}
    \bar{x} & = & \frac{x_1\cdot {\color{red}f_1} + x_2 \cdot {\color{red}f_2} + x_3 \cdot {\color{red}f_3} + 
    \cdots + x_N \cdot {\color{red}f_N}}{{\color{red}f_1} + {\color{red}f_2} + {\color{red}f_3} + 
    \cdots + {\color{red}f_N}}.
  \end{eqnarray*}

  
\end{frame}



\iftoggle{clicker}{%
  \subsection{Clicker Quiz}


  \begin{frame}
    \frametitle{Clicker Quiz}

    \vfill
    Find the sample mean of the following data set:\\
    \begin{eqnarray*}
      5.1,~5.4,~3.6,~5.8,~4.1,~3.1,~4.5,~1.3,~6.0,~3.3,~2.8,~1.7
    \end{eqnarray*}

    \vfill
    
    \begin{tabular}{l@{\hspace{3em}}l@{\hspace{3em}}ll@{\hspace{3em}}l}
      A: 3.85 & B: 3.89 & C: 4.7 
    \end{tabular}

    \vfill

  \end{frame}

}


\begin{frame}{Frequency Tables - Continuous Data}

  Suppose we have this data:
  \begin{eqnarray*}
    5.1,~5.4,~3.6,~5.8,~4.1,~3.1,~4.5,~1.3,~6.0,~3.3,~2.8,~1.7
  \end{eqnarray*}

  Doing the calculations straight from the data we get
  \begin{eqnarray*}
    \bar{x} & \approx & \\ %3.89 \\
    s       & \approx & 1.54.
  \end{eqnarray*}

  
\end{frame}


\begin{frame}{Frequency Tables - Continuous Data}

  Suppose we have this data:
  \begin{eqnarray*}
    5.1,~5.4,~3.6,~5.8,~4.1,~3.1,~4.5,~1.3,~6.0,~3.3,~2.8,~1.7
  \end{eqnarray*}

  We can count the number of occurrences in certain ranges, and
  suppose we are given this summary instead of the original data:
    \begin{eqnarray*}
      \begin{array}{l|l}
        \mathrm{Range}   & \mathrm{Frequency} \\ \hline
        1<x\leq 3 & 3 \\
        3<x\leq 5 & 5 \\
        5<x\leq 7 & 4 \\
      \end{array}
    \end{eqnarray*}

  \only<2->%
  {
    How do we approximate the sample mean and sample standard deviation?
  }

  
\end{frame}


\begin{frame}{Frequency Tables - Continuous Data}

  Suppose we have this data:
  \begin{eqnarray*}
    5.1,~5.4,~3.6,~5.8,~4.1,~3.1,~4.5,~1.3,~6.0,~3.3,~2.8,~1.7
  \end{eqnarray*}

  We can count the number of occurrences in certain ranges, and
  suppose we are given this summary instead of the original data:
    \begin{eqnarray*}
      \begin{array}{l|l}
        \mathrm{Range}   & \mathrm{Frequency} \\ \hline
        1<x\leq 3 & {\color{red}3} \\
        3<x\leq 5 & {\color{red}5} \\
        5<x\leq 7 & {\color{red}4} \\
      \end{array}
    \end{eqnarray*}

  \only<2->%
  {
    Assume that the values within each range are close to the median value of the range.

    So there are {\color{red}3} values that are close to two, there
    are {\color{red}5} values close to four, and there are
    {\color{red}4} values close to six.

    It is now just like the discrete case.

  }

  
\end{frame}

\begin{frame}{Frequency Tables - Continuous Data}

  Suppose we have this data:
  \begin{eqnarray*}
    5.1,~5.4,~3.6,~5.8,~4.1,~3.1,~4.5,~1.3,~6.0,~3.3,~2.8,~1.7
  \end{eqnarray*}

  We can count the number of occurrences in certain ranges, and
  suppose we are given this summary instead of the original data:
    \begin{eqnarray*}
      \begin{array}{l|l}
        \mathrm{Range}   & \mathrm{Frequency} \\ \hline
        1<x\leq 3 & {\color{red}3} \\
        3<x\leq 5 & {\color{red}5} \\
        5<x\leq 7 & {\color{red}4} \\
      \end{array}
    \end{eqnarray*}

    \begin{eqnarray*}
      \bar{x} & \approx & \frac{3\cdot 2 + 5 \cdot 4 + 4 \cdot 6}{3+5+4}, \\
      & \approx & 4.17.
    \end{eqnarray*}

  
\end{frame}


\begin{frame}{Frequency Tables - Continuous Data}

  Suppose we have this data:
  \begin{eqnarray*}
    5.1,~5.4,~3.6,~5.8,~4.1,~3.1,~4.5,~1.3,~6.0,~3.3,~2.8,~1.7
  \end{eqnarray*}

  We can count the number of occurrences in certain ranges, and
  suppose we are given this summary instead of the original data:
    \begin{eqnarray*}
      \begin{array}{l|l}
        \mathrm{Range}   & \mathrm{Frequency} \\ \hline
        1<x\leq 3 & {\color{red}3} \\
        3<x\leq 5 & {\color{red}5} \\
        5<x\leq 7 & {\color{red}4} \\
      \end{array}
    \end{eqnarray*}

    To get an approximation for the variance we approximate the
    deviations for the median of each range, and multiply by the
    number of deviations represented in each range:
    \begin{eqnarray*}
      s^2 & \approx & \frac{3\cdot (2-\bar{x})^2 + 
                            5 \cdot (4-\bar{x})^2 + 
                            4 \cdot (6-\bar{x})^2}{3+5+4-1}, \\
      & \approx & 2.52, \\
      s & \approx & 1.59.
    \end{eqnarray*}

  
\end{frame}


\begin{frame}{Frequency Tables - Continuous Data}

  In general we have
    \begin{eqnarray*}
      \begin{array}{l|l}
        \mathrm{Range}   & \mathrm{Frequency} \\ \hline
        x_1<x\leq x_2 & {\color{red}f_1} \\
        x_2<x\leq x_3 & {\color{red}f_2} \\
        \vdots & \vdots \\
        x_{n-1}<x\leq x_n & {\color{red}f_n} \\
      \end{array}
    \end{eqnarray*}

    \begin{eqnarray*}
      \bar{x_i} & = & \frac{x_i+x_{i+1}}{2}, \\
      \bar{x} & \approx & \frac{\bar{x}_1\cdot f_1 +  \bar{x}_2\cdot f_2 + \cdots + \bar{x}_n \cdot f_n}{
        f_1+f_2+\cdots+f_n}, \\
      s^2 & \approx & \frac{(\bar{x}_1-\bar{x})^2\cdot f_1 +  (\bar{x}_2-\bar{x})^2\cdot f_2 + \cdots +
        (\bar{x}_n-\bar{x})^2 \cdot f_n}{
        f_1+f_2+\cdots+f_n-1}, \\
    \end{eqnarray*}

  
\end{frame}

\subsection{Weighted Means}

\begin{frame}{Weighted Means}

  How do we calculate the final grade in this course? \\
  The final grades are calculated using the following distribution:
  \begin{tabular}[t]{rl}
    60\% & Three Exams, \\
    17\% & Quizzes, \\
    17\% & Webworks, \\
    6\%  & Attendance/Clicker Response \\
  \end{tabular}

  How does that work?

  
\end{frame}


\begin{frame}{Weighted Means}

  The final grades are calculated using the following distribution:
  \begin{tabular}[t]{rl}
    60\% & Three Exams, \\
    17\% & Quizzes, \\
    17\% & Webworks, \\
    6\%  & Attendance/Clicker Response \\
  \end{tabular}

  Suppose that your mean test score is 75\%, your mean quiz score is
  80\%, your mean webworks score is 90\%, and your mean clicker
  response score is 85\%, then the final grade is
  \begin{eqnarray*}
    \bar{x} & = & 0.60\cdot 75 + 0.17\cdot 80 + 0.17\cdot 90 + 0.06\cdot 85, \\
    & = & 79.
  \end{eqnarray*}

  
\end{frame}

\begin{frame}{What Do I Need?}

  The final grades are calculated using the following distribution:
  \begin{tabular}[t]{rl}
    60\% & Three Exams, \\
    17\% & Quizzes, \\
    17\% & Webworks, \\
    6\%  & Attendance/Clicker Response \\
  \end{tabular}

  Suppose that your mean quiz score is 80\%, your mean webworks score
  is 90\%, and your mean clicker response score is 85\%, and you want to achieve a 
  final grade of 70\%. What do you need for an average test score?
  \begin{eqnarray*}
    70 & = & 0.60\cdot \mathrm{Exam} + 0.17\cdot 80 + 0.17\cdot 90 + 0.06\cdot 85, \\
    70 & = &  0.60\cdot \mathrm{Exam} + 34, \\
    36 & = &  0.60\cdot \mathrm{Exam}, \\
    \mathrm{Exam} & = & \frac{36}{0.6}, \\
    & = & 60.
  \end{eqnarray*}

  
\end{frame}


\begin{frame}{Weighted Mean in General}

  \begin{eqnarray*}
    \begin{array}{l|l}
      \mathrm{Value}  & \mathrm{Weight} \\ \hline
      x_1 & {\color{red}w_1} \\
      x_2 & {\color{red}w_2} \\
      \vdots & \vdots \\
      x_N & {\color{red}w_n}
    \end{array}
  \end{eqnarray*}

  The weighted mean is 
  \begin{eqnarray*}
    \bar{x} & = & \frac{x_1\cdot{\color{red}w_1} + x_2\cdot{\color{red}w_2} +
      \cdots + x_n\cdot{\color{red}w_n}}{
      {\color{red}w_1}+{\color{red}w_2}+\cdots+{\color{red}w_n}}.
  \end{eqnarray*}
  
\end{frame}




% LocalWords:  Clarkson pausesection hideallsubsections hideothersubsections
% LocalWords:  sectionstyle Webworks webworks
