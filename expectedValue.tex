
\lecture{Expected Value}{expected-value}
\section{Expectation of Random Variables}

\title{Expectation of Random Variables}
\subtitle{Defining the Mean of a Random Variable}

%\author{Kelly Black}
%\institute{Clarkson University}
\date{9 September 2013}

\begin{frame}
  \titlepage
\end{frame}

\begin{frame}
  \frametitle{Outline}
  \tableofcontents[hideothersubsections,sectionstyle=show/hide]
\end{frame}


\subsection{Clicker Quiz}


\iftoggle{clicker}{%
  \begin{frame}
    \frametitle{Clicker Quiz}

    A random variable, $X$, has a probability density function given by
    \begin{eqnarray*}
      f(x) & = & \left\{
        \begin{array}{l@{\hspace{2em}}l}
          \frac{x}{18}, & 0\leq x \leq 6, \\
          0 & \mathrm{otherwise.}
        \end{array}
      \right.
    \end{eqnarray*}
    What is the probability that $X$ is between one and three? 

    \vfill
    
    \begin{tabular}{l@{\hspace{3em}}l@{\hspace{3em}}l@{\hspace{3em}}l}
      A: $\frac{1}{36}$ & B: $\frac{7}{36}$ & C: $\frac{8}{36}$  & D: $\frac{9}{36}$
    \end{tabular}

    \vfill
    \vfill
    \vfill

  \end{frame}
}


\subsection{Expected Value for Discrete Random Variables}


 \begin{frame}
 \frametitle{The Expected Value}
   \begin{columns}
     \column{.15\textwidth}
     \begin{eqnarray*}
       \begin{array}{r|l}
         \mathrm{X} & p \\ \hline
          x_1 & p_1 \\ [5pt]
          x_2 & p_2 \\ [5pt]
          x_3 & p_3 \\ [5pt]
          x_4 & p_4 \\
          \vdots & \vdots \\
          x_n & p_n
       \end{array}
     \end{eqnarray*}

     \column{.90\textwidth}
     \uncover<2->
     {
       \begin{eqnarray*}
         \mu_X \lp\mathrm{total~mass}\rp & = & x_1 \cdot p_1 + x_2\cdot p_2 + 
                 x_3 \cdot p_3 + \cdots + x_n \cdot p_n. \\
         \mu_X  & = & x_1 \cdot p_1 + x_2 \cdot p_2 + x_3 \cdot p_3 + \cdots + x_n \cdot p_n.
       \end{eqnarray*}
     }

   \end{columns}

 \end{frame}


 \begin{frame}
 \frametitle{The Expected Value}

 \begin{definition}
   The \redText{expected value} of a discrete random variable, $X$, is
   \begin{eqnarray*}
     E[X] & = & x_1 \cdot p_1 + x_2 \cdot p_2 + x_3 \cdot p_3 + \cdots + x_n \cdot p_n.
   \end{eqnarray*}
 \end{definition}

 \end{frame}


% \begin{frame}{Example}
%   \begin{columns}
%     \column{.15\textwidth}
%     \begin{eqnarray*}
%       \begin{array}{r|l}
%         \mathrm{X} & p \\ \hline
%          0 & \frac{1}{8} \\ [5pt]
%          1 & \frac{1}{8} \\ [5pt]
%          2 & \frac{4}{8} \\ [5pt]
%          3 & \frac{2}{8}
%       \end{array}
%     \end{eqnarray*}

%     \column{.90\textwidth}
%     \uncover<2->
%     {
%       \begin{eqnarray*}
%         \mu_X & = & 0 \cdot \frac{1}{8} + 1 \cdot \frac{1}{8} + 2 \cdot \frac{4}{8} + 3 \frac{2}{8}, \\
%         & = & \frac{15}{8}.
%       \end{eqnarray*}
%     }

%   \end{columns}

%     \uncover<3->
%     {
%         \begin{eqnarray*}
%           \sigma^2_X & = & \lp 0-\frac{15}{8}\rp^2 \frac{1}{8} + 
%           \lp 1-\frac{15}{8}\rp^2 \frac{1}{8} + \lp 2-\frac{15}{8}\rp^2 \frac{4}{8} + 
%           \lp 3-\frac{15}{8}\rp^2 \frac{2}{8}, \\
%           & = & \frac{55}{64}.
%         \end{eqnarray*}
%     }

%     \uncover<4->
%     {
%       \begin{eqnarray*}
%         \sigma_X & = & \sqrt{\frac{55}{64}}.
%       \end{eqnarray*}
%       \vfill
%     }

    

% \end{frame}



\subsection{Examples}

\begin{frame}{Examples}

  Flip a fair coin three times and report the number of tails. What is
  the expected value?
  
\end{frame}

\begin{frame}{Examples}

   \begin{columns}
     \column{.20\textwidth}
     \includegraphics[width=7em]{img/dr_evil_one_million_dollars}
     \column{.75\textwidth}

     I play the lottery, and it costs one dollar each time. If I match
     seven random numbers between 0 and 9, inclusive, then I win
     \textbf{one million dollars}. What is the expected earnings?
   \end{columns}
  
\end{frame}

\subsection{Expected Value for Continuous Random Variables}


 \begin{frame}
 \frametitle{The Expected Value}
   \begin{columns}
     \column{.5\textwidth}
     \begin{eqnarray*}
       p(a \leq x \leq b) & = & \int^b_a f(x) ~ dx.
     \end{eqnarray*}

     \column{.50\textwidth}
     \uncover<2->
     {
       \begin{eqnarray*}
         \mu_X \lp\mathrm{total~mass}\rp & = & \int^\infty_{-\infty} x f(x) ~ dx, \\
         \mu_X  & = & \int^\infty_{-\infty} x f(x) ~ dx.
       \end{eqnarray*}
     }

   \end{columns}

 \end{frame}


 \begin{frame}
 \frametitle{The Expected Value}

 \begin{definition}
   The \redText{expected value} of a continuous random variable, $X$, is
   \begin{eqnarray*}
     E[X] & = & \int^\infty_{-\infty} x f(x) ~ dx.
   \end{eqnarray*}
 \end{definition}

 \end{frame}


\subsection{Examples}


\iftoggle{clicker}{%
  \begin{frame}
    \frametitle{Clicker Quiz}

    A uniformly distributed random variable takes on values between
    one and five. What is the probability density function?


    \vfill
    
    \begin{tabular}{l@{\hspace{3em}}l@{\hspace{3em}}l@{\hspace{3em}}l}
      A: $\frac{1}{5}$ & B: $\frac{1}{4}$ & C: $1$  & D: $5$
    \end{tabular}

    \vfill
    \vfill
    \vfill

  \end{frame}
}


\begin{frame}{Example}

  A uniformly distributed random variable takes on values between one
  and five. What is the expected value?
  
\end{frame}

\begin{frame}{Example}

  The probability density function for a random variable is 
  \begin{eqnarray*}
    f(x) & = & \left\{
      \begin{array}{l@{\hspace{2em}}l}
        \frac{1}{4}-\frac{x}{32}, & 0\leq x \leq 8, \\
        0 & \mathrm{otherwise.}
        \end{array}
      \right.
  \end{eqnarray*}
  
  \begin{enumerate}
  \item What is $p(5\leq x \leq 6)$?
  \item What is the expected value?
  \end{enumerate}
  
\end{frame}


% LocalWords:  Clarkson pausesection hideallsubsections sectionstyle
%  LocalWords:  hideothersubsections
