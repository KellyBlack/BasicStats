
\lecture{Review of Discrete Distributions}{discrete-reviews}
\section{Review of Discrete Distributions}

\title{Review of Discrete Distributions}
\subtitle{Which one is which?}

%\author{Kelly Black}
%\institute{Clarkson University}
\date{26 September 2014}

\begin{frame}
  \titlepage
\end{frame}

\begin{frame}
  \frametitle{Outline}
  \tableofcontents[hideothersubsections,sectionstyle=show/hide]
\end{frame}


\subsection{Clicker Quiz}


\iftoggle{clicker}{%
  \begin{frame}
    \frametitle{Clicker Quiz}

    The number of cars going through a toll booth in a given hour has
    a mean of 97. What is the probability that 95 cars will go through
    the toll booth in a given hour?

    \vfill

    \begin{tabular}{l@{\hspace{3em}}l@{\hspace{3em}}l@{\hspace{3em}}l}
      A: $\frac{e^{-95}(95)^{97}}{97!}$ & 
      B: $\frac{e^{-97}(97)^{95}}{95!}$ &
      C: ${97 \choose 95} (.5)^{95}(.5)^{2}$ & 
      D: $0.5$
    \end{tabular}

    \vfill
    \vfill
    \vfill


  \end{frame}
}


\subsection{Discrete Random Variables}

\begin{frame}{Discrete Random Variables}

  \begin{columns}
    \column{.20\textwidth}
    Probability Mass Function
    \begin{eqnarray*}
      \begin{array}{l|l}
        \mathrm{X} & p \\ \hline
        x_1 & p_1 \\
        x_2 & p_2 \\
        x_3 & p_3 \\
        \vdots & \vdots \\
        x_n & p_n
      \end{array}
    \end{eqnarray*}

    \column{.80\textwidth}
    Properties:
    \begin{eqnarray*}
      \begin{array}{rcccl}
        0 & \leq & p_i & \leq 1
      \end{array}
      \\
      p_1 + p_2 + p_3 + \cdots + p_n & = & 1.
    \end{eqnarray*}

   The \redText{expected value} of a discrete random variable, $X$, is
   \begin{eqnarray*}
     E[X] & = & x_1 \cdot p_1 + x_2 \cdot p_2 + x_3 \cdot p_3 + \cdots + x_n \cdot p_n.
   \end{eqnarray*}


  \end{columns}
  
\end{frame}



\begin{frame}
  \frametitle{Variance}

    \begin{definition}
      The \redText{variance} of a random variable is defined to be
      \begin{eqnarray*}
        \sigma^2 & = & E\left[\lp X - E[X] \rp^2\right], \\
        & = & E\left[X^2\right] - \lp E\left[X\right]\rp^2.
      \end{eqnarray*}
    \end{definition}

    \begin{definition}
      The \redText{standard deviation} of a random variable is defined to be
      \begin{eqnarray*}
        \sigma   & = & \sqrt{\sigma^2}, \\
                 & = & \sqrt{E\left[\lp X - E[X] \rp^2\right]}.
      \end{eqnarray*}
    \end{definition}

\end{frame}


\subsection{Bernoulli Distribution}

\begin{frame}
  \frametitle{Bernoulli Distribution}

  \begin{columns}
    \column{.50\textwidth}
    \begin{definition}
      A random variable is a \redText{Bernoulli distribution} if it
      has
      a probability mass function that can written in the following form:\\
      \begin{tabular}{r|c}
        X & $p$ \\ \hline
        0 & $1-p$ \\
        1 & $p$
      \end{tabular}
    \end{definition}
    \column{.50\textwidth}
    \begin{itemize}
    \item Only two outcomes, 0 or 1.
    \item A single trial.
    \item One parameter, $p$.
    \item $E[X]=p$
    \item $\mathrm{Var}[X]=p(1-p)$.
    \end{itemize}
  \end{columns}
\end{frame}




\subsection{Binomial Distribution}

\begin{frame}{Binomial Distribution}

  \begin{columns}
    \column{.50\textwidth}
    \begin{definition}[Binomial Distribution]
      If $X_1$, $X_2$, $\ldots$, $X_n$ are independent Bernoulli
      random variables with the same parameter, $p$, then define $Y$
      to be 
      \begin{eqnarray*}
        Y & = & X_1 + X_2 + X_3 + \cdots + X_n.
      \end{eqnarray*}
      The random variale $Y$ has a \redText{binomial distribution}.
    \end{definition}
    \column{.50\textwidth}
    \begin{itemize}
    \item $N$ experiments, $X_1$, $X_2$, $\ldots$, $X_n$.
    \item Each experiment has only two possible outcomes (0/1).
    \item Each experiment has a probability $p$ of a ``1'' outcome.
    \item Each experiment is independent of the others.
    \item Add up the results. (Total number of ``yeses.'')
    \end{itemize}
  \end{columns}
\end{frame}


\begin{frame}{Binomial Distribution}

  \vfill
  
    The probability of $k$ ``yes'' outcomes for a random variable that
    follows a binomial distribution is
    \begin{eqnarray*}
      p(Y=k) & = & C^N_k \cdot p^k (1-p)^{N-k}.
    \end{eqnarray*}

    \vfill

    The expectation and variation of a Binomial random variable are
    the following:
    \begin{eqnarray*}
      E[Y] & = & N\cdot p, \\
      \mathrm{Var}[Y] & = & N \cdot p \cdot (1-p).
    \end{eqnarray*}

\end{frame}



\subsection{Geometric Distribution}

\begin{frame}
  \frametitle{Geometric Distribution}

  \begin{columns}
    \column{.50\textwidth}
    \begin{definition}
      A \redText{geometric} random variable returns the number of
      times that a sequence of independent Bernoulli random variables,
      each with parameter $p$, are sampled until the first ``yes''
      is found. Define $Y$ to be the total number of trials taken.
    \end{definition}
    \column{.50\textwidth}
      \begin{itemize}
      \item A sequence of Bernoulli Trials, $X_1$, $X_2$, $X_3$,
        $\ldots$
      \item Each Bernoulli trial is independent of the others.
      \item Each one has a probability of $p$ of a ``yes.''
      \item Test each one in order starting at the first.
      \item You report the number of tests you conduct when you get
        the first ``1'' (i.e. a ``yes'').
      \end{itemize}
  \end{columns}

\end{frame}

\begin{frame}{Properties}

  The probability mass function for a geometric distribution with
  probability $p$ for a ``yes'' in any one trial is
  \begin{eqnarray*}
    p(Y=k) & = & \lp 1-p\rp^{k-1} p.
  \end{eqnarray*}

  The expectation and variance of a geometric distribution are
  \begin{eqnarray*}
    E[Y] & = & \frac{1}{p}, \\
    \mathrm{Var}[Y] & = & \frac{1-p}{p^2}.
  \end{eqnarray*}

\end{frame}

\subsection{Hypergeometric Distribution}

\begin{frame}{Hypergeometric Distribution}

  \begin{columns}
    \column{.50\textwidth}
    \begin{definition}
      Choose $n$ items without replacement from a group of $N$
      items. There are $r$ objects that are ``special.'' Return the
      number of ``special'' items that were chosen. Define $Y$ to be
      the number of ``special'' items selected.
    \end{definition}
    \column{.50\textwidth}
    \begin{itemize}
    \item There are $N$ objects.
    \item There are $r$ of the objects that are different.
    \item I pick $n$ objects at random \textbf{without} replacement.
    \item I report the number of objects ($k$) that are from the group
      of ``different'' items that were chosen.
    \end{itemize}
  \end{columns}
  \vfill


\end{frame}

\begin{frame}{Hypergeometric Distribution}


    If $Y$ is a hypergeometric distribution then
    \begin{eqnarray*}
      p(Y=k) & = & \frac{C^r_k \cdot C^{N-r}_{n-k}}{C^N_n}, \\
      E[Y] & = & \frac{nr}{N}, \\
      \mathrm{Var}[Y] & = & \frac{N-n}{N-1} \cdot n \cdot \frac{r}{N} \cdot \lp 1-\frac{r}{N}\rp.
    \end{eqnarray*}

\end{frame}


\subsection{Poisson Distribution}

\begin{frame}
  \frametitle{Poisson Distribution}

  \begin{columns}
    \column{.50\textwidth}
    \begin{definition}
      A \redText{Poisson} random variable returns the number of events
      that occur within some constraint.
    \end{definition}
    \column{.50\textwidth}
      \begin{itemize}
      \item Count of the total number of times some event occurs,
      \item The count occurs under some limit or per item or some sort
        of constraint.
      \item There is no explicit limit on the total number of times
        that the event can occur.
      \item Has one parameter, $\lambda$.
      \end{itemize}
  \end{columns}

\end{frame}

\begin{frame}{Properties}

  The probability mass function for a Poisson distribution with
  parameter $\lambda$ for the number of times an event can occur
  within the constraint is the following:
  \begin{eqnarray*}
    p(X=k) & = & \frac{e^{-\lambda}\lambda^k}{k!}.
  \end{eqnarray*}

  The expectation and variance of a Poisson distribution are
  \begin{eqnarray*}
    E[X] & = & \lambda, \\
    \mathrm{Var}[X] & = & \lambda.
  \end{eqnarray*}

\end{frame}


\subsection{Examples}

\begin{frame}{Good Problems in the Book}

  \vfill 
  Pages 184-185:
  \begin{itemize}
  \item 3.8.1-3.8.5
  \item 3.8.7-3.8.9
  \item 3.8.10 (a,c), 3.8.13 (a,c)
  \item 3.8.11-3.8.12
  \item 3.8.14
  \end{itemize}
  \vfill
  
\end{frame}


\iftoggle{clicker}{%
  \begin{frame}{Clicker Quiz}

    \vfill

    A shelf in a store has ten items on it. Three of the items are out
    of date. A customer picks four items at random. What is the
    probability that the customer gets at most one out of date item?
  
    \vfill

    What kind of distribution is this?

    \vfill

    \begin{tabular}{l@{\hspace{3em}}l@{\hspace{3em}}l@{\hspace{3em}}l}
      A: Binomial  & B: Geometric \\ [12pt] C: Hypergeometric & D: Poisson
    \end{tabular}

    \vfill
  
  \end{frame}
}

\begin{frame}{Example}

  \vfill

    A shelf in a store has ten items on it. Three of the items are out
    of date. A customer picks four items at random. What is the
    probability that the customer gets at most one out of date item?


  \vfill

  (0.6667)
  
\end{frame}

\iftoggle{clicker}{%
  \begin{frame}{Clicker Quiz}

    \vfill

    A person will use a pen until they run into a problem and will
    then throw it out. There is a probability of 0.02 that the item
    will have a problem in any one use. What is the probability that
    the person will use the item at least ten times?
  
    \vfill

    What kind of distribution is this?

    \vfill

    \begin{tabular}{l@{\hspace{3em}}l@{\hspace{3em}}l@{\hspace{3em}}l}
      A: Binomial  & B: Geometric \\ [12pt] C: Hypergeometric & D: Poisson
    \end{tabular}

    \vfill
  
  \end{frame}
}

\begin{frame}{Example}

  \vfill

    A person will use a pen until they run into a problem and will
    then throw it out. There is a probability of 0.02 that the item
    will have a problem in any one use. What is the probability that
    the person will use the item at least ten times?

    \vfill

    (.8171)
  
\end{frame}

\iftoggle{clicker}{%
  \begin{frame}{Clicker Quiz}

    \vfill

    A paper mill has a roller that is expected to experience a mean of
    1.3 failures per day. What is the probability of having less than
    three failures in a given day?
  
    \vfill

    What kind of distribution is this?

    \vfill

    \begin{tabular}{l@{\hspace{3em}}l@{\hspace{3em}}l@{\hspace{3em}}l}
      A: Binomial  & B: Geometric \\ [12pt] C: Hypergeometric & D: Poisson
    \end{tabular}

    \vfill
  
  \end{frame}
}

\begin{frame}{Example}

  \vfill

    A paper mill has a roller that is expected to experience a mean of
    1.3 failures per day. What is the probability of having less than
    three failures in a given day?

  \vfill

  (0.8571)
  
\end{frame}

\iftoggle{clicker}{%
  \begin{frame}{Clicker Quiz}

    \vfill

    A signal is received by a listening circuit once every 0.003
    seconds. There is a probability of 0.012 that the signal is not
    correct. What is the probability that there are more than two
    errors in a 0.030 second time period?
  
    \vfill

    What kind of distribution is this?

    \vfill

    \begin{tabular}{l@{\hspace{3em}}l@{\hspace{3em}}l@{\hspace{3em}}l}
      A: Binomial  & B: Geometric \\ [12pt] C: Hypergeometric & D: Poisson
    \end{tabular}

    \vfill
  
  \end{frame}
}

\begin{frame}{Example}

  \vfill

    A signal is received by a listening circuit once every 0.003
    seconds. There is a probability of 0.012 that the signal is not
    correct. What is the probability that there are more than two
    errors in a 0.030 second time period?

  \vfill

  (.00019)
  
\end{frame}

\iftoggle{clicker}{%
  \begin{frame}{Clicker Quiz}

    \vfill

    A signal is received by a listening circuit. It is expected to
    receive 1.8 errors per hour. What is the probability that there
    will be more than two errors in a given hour.
  
    \vfill

    What kind of distribution is this?

    \vfill

    \begin{tabular}{l@{\hspace{3em}}l@{\hspace{3em}}l@{\hspace{3em}}l}
      A: Binomial  & B: Geometric \\ [12pt] C: Hypergeometric & D: Poisson
    \end{tabular}

    \vfill
  
  \end{frame}
}

\begin{frame}{Example}

  \vfill

    A signal is received by a listening circuit. It is expected to
    receive 1.8 errors per hour. What is the probability that there
    will be more than two errors in a given hour.

  \vfill

  (.2693)
  
\end{frame}


%%% Local Variables: 
%%% mode: latex
%%% TeX-master: t
%%% End: 

%  LocalWords:  Hypergeometric hypergeometric
